\section{\'Economie d'énergie}

Le tableau ci-dessous présente 3 types de lampes fournissant un même flux lumineux.
\begin{center}
	
\begin{tabular}{|@{\ }l@{\ }|@{\ }c@{\ }|@{\ }c@{\ }|@{\ }c@{\ }|}
	\hline
	& \textbf{Halogène} & \textbf{Fluocompacte} & \textbf{LED}   \\ \hline
	\textbf{Puissance} (en W)    & 85                & 24                    & 21             \\ \hline
	\textbf{Durée de vie} (en h) & \num{2000}        & \num{10000}           & \num{20000} \\ \hline
	\textbf{Prix} (en €)         & 7                 & 13                    & 19             \\ \hline
	\textbf{Classe énergétique}  & D                 & B                     & A+             \\ \hline
\end{tabular}
\end{center}

\begin{questions}
	\question Calculer l'énergie, en kWh, consommée en une année par chaque lampe, si l'on suppose qu'elles fonctionnent 4 heures par jour.
	
	\fillwithdottedlines{5cm}
	
	\question Calculer alors le cout de fonctionnement de chaque lampe en une année, en €, sachant que le prix moyen du  kWh est d'environ \num{0.15} €.
	\fillwithdottedlines{5cm}
	
	\question Calculer, pour chaque type de lampe, son cout d'utilisation pendant 10 ans.
	
	\fillwithdottedlines{5cm}
\end{questions}