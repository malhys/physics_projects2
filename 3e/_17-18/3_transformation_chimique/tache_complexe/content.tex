	\begin{center}
		{\LARGE \textbf{Proposition de correction de l'exercice 32 p 33}}
	\end{center}
	
	
	D'après le \textbf{doc. 2}, la combustion du bois produit du dioxyde d e carbone ($CO_2$) qui est ensuite consommé par les arbres. Pour que l'utilisation du bois de chauffage ait un bilan carbone neutre, il faut que les arbres consomment autant de molécules de $CO_2$ que la combustion du bois de chauffage.	
	D'après le \textbf{doc. 3}, le bois est essentiellement composé d'un dérivé du glucose ; la combustion d'une molécule de glucose produit 6 molécules de $CO_2$. 
	D'après le \textbf{doc. 1}, la photosynthèse permet aux végétaux chlorophylliens et donc aux arbres de produire du glucose à partir de $CO_2$. Produire une molécule de glucose consomme 6 molécules de $CO_2$.
	
	Si un nouvel arbre est planté pour chaque arbre coupé, alors pour chaque molécule de glucose consommée, le $CO_2$ produit sera à son tour consommé pour produire une nouvelle molécule de glucose. Donc, l'utilisation de bois de chauffage produit dans ces conditions affiche un bilan carbone neutre.