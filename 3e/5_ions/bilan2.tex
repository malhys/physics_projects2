\begin{mybilan}
	
	\begin{itemize}
		\item Pour identifier les ions présents dans une solution on utilise un \kw{test reconnaissance} par précipitation.\pause
		
		\item On verse quelques gouttes de réactif dans la solution à tester (voir schéma de la partie 1 page 22 du cahier d'activité):\pause
			\begin{itemize}
				\item Si il y a une \kw{réaction chimique} entre les ions du réactif et de la solution, un \kw{précipité} est formé. Le test est \kw{positif}.
				
				\item S'il n'y a \kw{pas de précipité}, le test est \kw{négatif}.\pause
			\end{itemize} 
		
		\item Voir la fiche méthode 6 page 123 pour les tests d'identification des ions métalliques.
	\end{itemize}

	
		

\end{mybilan}