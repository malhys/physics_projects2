\section{Ions et santé}

On attribue plusieurs vertus au bicarbonates de sodium. On l'emploie notamment pour pour l'hygiène dentaire ou contre les maux d'estomac.

\begin{questions}
	\question L'ion Sodium est engendré par un atome de sodium lorsqu'il perd un électron.	
	\begin{parts}
		\part Combien de charges positives compte le noyau de l'ion sodium ?
		\begin{solution}
			L'ion sodium possède 17 protons dans son noyau donc 17 charges positives.
		\end{solution}
		
		\part Combien d'électrons composent cet ion ?
		\begin{solution}
			L'atome de sodium contient 17 électrons, donc l'ion sodium qui en a perdu un, en contient 16.
		\end{solution}
		
		\part \'Ecrire la formule chimique de cet ion.		
		\begin{solution}
			La formule chimique de l'ion sodium est $Na^+$.
		\end{solution}
	\end{parts}

	\question La formule chimique de l'ion bicarbonate s'écrit $HCO_3^-$
	\begin{parts}
		\part S'agit-il d'un cation ou d'un anion ?
		\begin{solution}
			L'ion bicarbonate est chargé négativement, c'est donc un anion.
		\end{solution}
	
		\part Combien d'atomes de chaque élément composent cet ion ?
		\begin{solution}
			Cet ion est composé d'un atome d'hydrogène, d'un de carbone et de 3 d'oxygène.
		\end{solution}
		
		\part Ce groupe d'atomes a perdu ou gagné un ou des électrons pour devenir un ion. Combien ?
		\begin{solution}
			C'est un anion donc il a gagné un électron.
		\end{solution}
	
	\end{parts}
\end{questions}
