\section{Identification de solutions}

Au laboratoire, Enzo a trouvé un flacon sans étiquette, qui contient une solution incolore. 

\begin{center}
	\includegraphics[scale=0.6]{img/docs}
\end{center}

\begin{questions}
	\question La solution inconnue est l'une de celles présentes dans le doc. 1. Quels tests doit-il faire pour l'identifier ? Décrire le protocole expérimental.
	\begin{solution}
		Pour identifier la solution inconnue, Enzo devra :
		\begin{itemize}
			\item verser quelques millilitres de la solution dans deux tubes à essais;
			\item ajouter quelques goutes de nitrate d'argent dans le premier tube;
			\item ajouter quelques goutes de soude dans le second tube;
			\item observer les résultats.
		\end{itemize}
	\end{solution}
	
	\question Représenter un de ces tests à l'aide d'un schéma.
	\begin{solution}
		\includegraphics[scale=0.2]{img/ajout-soude}
	\end{solution}
	
	\question D'après les résultats obtenus présentés dans le doc. 3, quels ions ont été identifiés par les tests ? 
	\begin{solution}
		La solution à réagit avec la soude en formant un précipité vert et avec le nitrate d'argent en formant un précipité blanc. Elle contient donc des ions Fer II ($Fe^{2+}$)et Chlorure ($Cl^-$).
	\end{solution}
	
	\question Quelle est la solution contenue dans le flacon ?
	\begin{solution}
		La solution contenue dans le flacon est donc un mélange de chlorure de sodium et de sulfate de fer (II).
	\end{solution}
	
	\question Détailler la composition de chacun des ions présents dans la solution. (nombre de protons, nombre d'électrons, nombre de charges).
	\begin{solution}
		Cette solution contient des ions chlorure ($Cl^-$), sodium ($Na^+$), sulfate ($SO_4^{2-}$) et fer II ($Fe^{2+}$).
		\begin{itemize}
			\item Le numéro atomique de l'atome de chlore est 17, l'ion chlorure contient donc 17 protons, 18 électrons et 1 charge négative;
			
			\item Le numéro atomique de l'atome de sodium est 11, l'ion chlorure contient donc 11 protons, 10 électrons et 1 charge positive;
			
			\item L'ion sulfate est composé d'un atome de soufre et de 4 atomes d'oxygène. Le numéro atomique de l'atome de soufre est 16, celui de l'atome d'oxygène est 8. Donc l'ion sulfate compte 48 protons et 50 électrons, soit deux charges négatives. 
			
			\item Le numéro atomique de l'atome de fer est 26, l'ion fer II contient donc 26 protons, 24 électrons et 2 charges positives.
			
			.
		\end{itemize}
		
		
		
	\end{solution}
	
\end{questions}