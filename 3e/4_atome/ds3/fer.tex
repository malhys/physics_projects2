\section{L'atome de Fer (4 points)}

\begin{multicols}{2}
	
	
	Le métal fer est un cristal, ce qui veut dire que ses atomes sont organisés selon une structure bien particulière appelée maille élémentaire. Sur l'Atomium à Bruxelles, chaque sphère de 18 m de diamètre représente un atome de fer agrandi 64 milliards de fois.
	
	\includegraphics[scale=0.5]{img/atomium}
\end{multicols}


\begin{questions}
	\question[2] Calculer le diamètre d'un atome de fer.
	%\fillwithdottedlines{2cm}
	\begin{solution}
		64 milliards  = $64 \times 10^9$.
		\begin{eqnarray*}
			\dfrac{18}{64 \times 10^9}& = & \num{2.8125} \times 10^{-10} m \\
			& = & \num{281.25} \times 10^{-12} m \\
		\end{eqnarray*}
	Un atome de fer a un diamètre de $\num{281.25} pm$.
	\end{solution}
	  
	
	\question[1] Combien d'électrons contient-il ?
	%\fillwithdottedlines{2cm}
	\begin{solution}
		Le numéro atomique de l'atome de fer est 26, il contient 26 protons, et donc 26 électrons.
	\end{solution}
	
	\question[1] Quel est le diamètre du noyau d'un atome de fer ?
	%\fillwithdottedlines{3cm}
	\begin{solution}
		Le diamètre d'un atome est \num{100000} fois plus grand que celui de son noyau.
		
		\begin{eqnarray*}
			\dfrac{\num{2.8125} \times 10^{-10}}{10^5}& = & \num{2.8125} \times 10^{-15} m
		\end{eqnarray*}
	
	Donc le noyau d'un atome de fer a un diamètre de \num{2.8125} fm.
	\end{solution}
\end{questions}

