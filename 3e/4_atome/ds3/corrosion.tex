
\section{La corrosion du fer}

Au contact du dioxygène $O_2$ et en présence d'eau $H_2O$, le fer $Fe$ se corrode en rouille $Fe_2O_3$. L'eau est indispensable pour ce processus, mais lors de cette transformation la quantité totale d'eau reste la même. On propose une équation pour modéliser cette réaction : $4 Fe + 4 O_2 \rightarrow 2 Fe_2O_3$.


\begin{questions}
	\question Expliquer pourquoi l'eau n'est ni un réactif, ni un produit dans cette transformation.
	\fillwithdottedlines{1.5cm}
	
	\question Compter le nombre d'atomes de fer dans les réactifs puis dans les produits de l'équation.
	\fillwithdottedlines{1.5cm}
	
	\question Faire de même pour les atomes d'oxygène.
	\fillwithdottedlines{1.5cm}
	
	\question Indiquer à l'aide des réponses précédentes, si l'équation de réaction est équilibrée. Si ce n'est pas le cas, proposer une correction de l'équation.
	\fillwithdottedlines{2.5cm}
\end{questions}
