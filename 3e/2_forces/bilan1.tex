\begin{mybilan}
	\begin{itemize}
		
		\item Si le mouvement d'un corps dépend du mouvement d'un autre, alors ces deux corps sont en \kw{interaction}.
		
		\item Pour modéliser une interaction, on utilise \kw{une force}.
		
		\item 'Une force est définie par :
			\begin{itemize}
				\item sa \kw{direction} (verticale, horizontale ou oblique);
				\item son \kw{sens} (vers les haut/le bas, vers la gauche / la droite);
				\item son \kw{point d'application};
				\item sa \kw{valeur} (exprimée en Newton, de symbole $N$).
			\end{itemize}
		
		\item Sur un schéma, une force est représentée par une flèche. Ses caractéristiques sont les mêmes que celles de la force et sa longueur est proportionnelle à sa valeur.
		
		\item \kw{Deux forces} exercées sur le même corps avec la même direction, la même valeur, et des sens opposés \kw{se compensent} (elles s'annulent).
	\end{itemize}
\end{mybilan}