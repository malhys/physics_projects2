\section{Des actions qui se compensent (3 points)}\label{actions}

Un corps est soumis à des forces qui se compensent.

\begin{questions}
	\question Dans quels cas des forces se compensent-elles ?
	\begin{solution}
		Des forces se compensent si elles ont même direction et même valeur mais des sens opposés.
	\end{solution}
	\question Ce corps est au repos :
	
	\begin{parts}
		\part Son état de repos va-t-il être modifié ?
			\begin{solution}
				Si un corps est au repos et soumis a des actions qui se compensent, alors il restera dans son état de repos.
			\end{solution}
		
		\part Que faut-il pour que l'état soit modifié ?
			\begin{solution}
				L'état de repos sera modifié si une (des) force(s) s'exerce(nt) sur le corps sans se compenser.
			\end{solution}
	\end{parts}
	
\end{questions}