
\section{La corrosion du fer (3 points +2 bonus)}\label{ex:corrosion}

Au contact du dioxygène $O_2$ et en présence d'eau $H_2O$, le fer $Fe$ se corrode en rouille $Fe_2O_3$. L'eau est indispensable pour ce processus, mais lors de cette transformation la quantité totale d'eau reste la même. On propose une équation pour modéliser cette réaction : 

\begin{center}
	\ce{4Fe + 4O2 -> 2Fe2O3}
\end{center}
%\begin{equation}
%	4 Fe + 4 O_2 \rightarrow 2 Fe_2O_3
%\end{equation}


\begin{questions}
	\question[1] Expliquer pourquoi l'eau n'est ni un réactif, ni un produit dans cette transformation.
	%\fillwithdottedlines{2cm}
	\begin{solution}
		Le texte indique que dans cette réaction, la quantité d'eau ne change pas donc il n'est pas nécessaire de la faire apparaître dans l'équation de réaction.
	\end{solution}
	
	\question[1] Compter le nombre d'atomes de fer dans les réactifs puis dans les produits de l'équation.
	%\fillwithdottedlines{2cm}
	\begin{solution}
		Dans les réactifs il y a 4 atomes de fer et dans les produits il y en 4 aussi.
	\end{solution}
	
	\question[1] Faire de même pour les atomes d'oxygène.
	%\fillwithdottedlines{2cm}
	\begin{solution}
		Dans les réactifs il y a 8 atomes d'oxygène et dans les produits il y en a 6.
	\end{solution}
	
	\question[2] Indiquer à l'aide des réponses précédentes, si l'équation de réaction est équilibrée. Si ce n'est pas le cas, proposer une correction de l'équation. \textbf{(bonus)}
	%\fillwithdottedlines{3cm}
	\begin{solution}
		Il n'y a pas le même nombre d'atomes d'oxygène dans les réactifs et les produits donc l'équation de réaction n'est pas équilibrée. 
		
		Dans l'équation suivante il y a autant d'atomes de fer et d'oxygène dans les réactifs que dans les produits, elle est donc équilibrée :
		\begin{center}
			\ce{4Fe + 3O2 -> 2Fe2O3}
		\end{center}
	\end{solution}
\end{questions}
