\documentclass[a4paper,11pt]{exam}
\printanswers % pour imprimer les réponses (corrigé)
%\noprintanswers % Pour ne pas imprimer les réponses (énoncé)
\addpoints % Pour compter les points
% \noaddpoints % pour ne pas compter les points
%\qformat{\textbf{\thequestion ) } }
%\qformat{\textbf{\thequestion )}} % Pour définir le style des questions (facultatif)
\usepackage{color} % définit une nouvelle couleur
\shadedsolutions % définit le style des réponses
% \framedsolutions % définit le style des réponses
\definecolor{SolutionColor}{rgb}{0.8,0.9,1} % bleu ciel
\renewcommand{\solutiontitle}{\noindent\textbf{Solution:}\par\noindent} % Définit le titre des solutions




\makeatletter

\def\maketitle{{\centering%
	\par{\huge\textbf{\@title}}%
	\par{\@date}%
	\par}}


\renewcommand{\thesubsection}{\Alph{subsection}.}   

\makeatother

\lhead{NOM Pr\'enom :}
\rhead{\textbf{Les r\'eponses doivent \^etre justifi\'ees et r\'edig\'ees}}
\cfoot{\thepage / \pageref{LastPage}}


%\usepackage{../../pas-math}
%\usepackage{../../moncours}


%\usepackage{pas-cours}
%-------------------------------------------------------------------------------
%          -Packages nécessaires pour écrire en Français et en UTF8-
%-------------------------------------------------------------------------------
\usepackage[utf8]{inputenc}
\usepackage[frenchb]{babel}
%\usepackage{numprint}
\usepackage[T1]{fontenc}
%\usepackage{lmodern}
\usepackage{textcomp}
\usepackage[french, boxed]{algorithm2e}
\usepackage{hyperref}


%-------------------------------------------------------------------------------

%-------------------------------------------------------------------------------
%                          -Outils de mise en forme-
%-------------------------------------------------------------------------------
\usepackage{hyperref}
\hypersetup{pdfstartview=XYZ}
%\usepackage{enumerate}
\usepackage{graphicx}
\usepackage{multicol}
\usepackage{tabularx}
\usepackage{multirow}
\usepackage{color}
\usepackage{eurosym}


\usepackage{anysize} %%pour pouvoir mettre les marges qu'on veut
%\marginsize{2.5cm}{2.5cm}{2.5cm}{2.5cm}

\usepackage{indentfirst} %%pour que les premier paragraphes soient aussi indentés
\usepackage{verbatim}
\usepackage{enumitem}
\usepackage{booktabs}
\usepackage[usenames,dvipsnames,svgnames,table]{xcolor}

\usepackage{variations}

%-------------------------------------------------------------------------------


%-------------------------------------------------------------------------------
%                  -Nécessaires pour écrire des mathématiques-
%-------------------------------------------------------------------------------
\usepackage{amsfonts}
\usepackage{amssymb}
\usepackage{amsmath}
\usepackage{amsthm}
\usepackage{tikz}
\usepackage{xlop}
\usepackage[output-decimal-marker={,}]{siunitx}
%-------------------------------------------------------------------------------

%-------------------------------------------------------------------------------
%                  -Nécessaires pour écrire des formules chimiquess-
%-------------------------------------------------------------------------------

\usepackage[version=4]{mhchem}

%-------------------------------------------------------------------------------
% Pour pouvoir exploiter les fichiers directement dans beamer
\newcommand{\pause}{\ }
%-------------------------------------------------------------------------------
%                    - Mise en forme avancée
%-------------------------------------------------------------------------------

\usepackage{ifthen}
\usepackage{ifmtarg}


\newcommand{\ifTrue}[2]{\ifthenelse{\equal{#1}{true}}{#2}{$\qquad \qquad$}}

%\newcommand{\kword}[1]{\textcolor{red}{\underline{#1}}}
%-------------------------------------------------------------------------------

%-------------------------------------------------------------------------------
%                     -Mise en forme d'exercices-
%-------------------------------------------------------------------------------
%\newtheoremstyle{exostyle}
%{\topsep}% espace avant
%{\topsep}% espace apres
%{}% Police utilisee par le style de thm
%{}% Indentation (vide = aucune, \parindent = indentation paragraphe)
%{\bfseries}% Police du titre de thm
%{.}% Signe de ponctuation apres le titre du thm
%{ }% Espace apres le titre du thm (\newline = linebreak)
%{\thmname{#1}\thmnumber{ #2}\thmnote{. \normalfont{\textit{#3}}}}% composants du titre du thm : \thmname = nom du thm, \thmnumber = numéro du thm, \thmnote = sous-titre du thm

%\theoremstyle{exostyle}
%\newtheorem{exercice}{Exercice}
%
%\newenvironment{questions}{
%\begin{enumerate}[\hspace{12pt}\bfseries\itshape a.]}{\end{enumerate}
%} %mettre un 1 à la place du a si on veut des numéros au lieu de lettres pour les questions 
%-------------------------------------------------------------------------------

%-------------------------------------------------------------------------------
%                    - Mise en forme de tableaux -
%-------------------------------------------------------------------------------

\renewcommand{\arraystretch}{1.7}

\setlength{\tabcolsep}{1.2cm}

%-------------------------------------------------------------------------------



%-------------------------------------------------------------------------------
%                    - Racourcis d'écriture -
%-------------------------------------------------------------------------------
%Droites
\newcommand{\dte}[1]{$(#1)$}
\newcommand{\fig}[1]{figure $#1$}
\newcommand{\sym}{symétrique}
\newcommand{\syms}{symétriques}
\newcommand{\asym}{axe de symétrie}
\newcommand{\asyms}{axes de symétrie}
\newcommand{\seg}[1]{$[#1]$}
\newcommand{\monAngle}[1]{$\widehat{#1}$}
\newcommand{\bissec}{bissectrice}
\newcommand{\mediat}{médiatrice}
\newcommand{\ddte}[1]{$[#1)$}


% Angles orientés (couples de vecteurs)
\newcommand{\aopp}[2]{(\vec{#1}, \vec{#2})} %Les deuc vecteurs sont positifs
\newcommand{\aopn}[2]{(\vec{#1}, -\vec{#2})} %Le second vecteur est négatif
\newcommand{\aonp}[2]{(-\vec{#1}, \vec{#2})} %Le premier vecteur est négatif
\newcommand{\aonn}[2]{(-\vec{#1}, -\vec{#2})} %Les deux vecteurs sont négatifs

%Ensembles mathématiques
\newcommand{\naturels}{\mathbb{N}} %Nombres naturels
\newcommand{\relatifs}{\mathbb{Z}} %Nombres relatifs
\newcommand{\rationnels}{\mathbb{Q}} %Nombres rationnels
\newcommand{\reels}{\mathbb{R}} %Nombres réels
\newcommand{\complexes}{\mathbb{C}} %Nombres complexes


%Intégration des parenthèses aux cosinus
\newcommand{\cosP}[1]{\cos\left(#1\right)}
\newcommand{\sinP}[1]{\sin\left(#1\right)}


%Probas stats
\newcommand{\stat}{statistique}
\newcommand{\stats}{statistiques}


\newcommand{\homo}{homothétie}
\newcommand{\homos}{homothéties}


\newcommand{\mycoord}[3]{(\textcolor{red}{\num{#1}} ; \textcolor{Green}{\num{#2}} ; \textcolor{blue}{\num{#3}})}
%-------------------------------------------------------------------------------

%-------------------------------------------------------------------------------
%                    - Mise en page -
%-------------------------------------------------------------------------------

\newcommand{\twoCol}[1]{\begin{multicols}{2}#1\end{multicols}}


\setenumerate[1]{font=\bfseries,label=\textit{\alph*})}
\setenumerate[2]{font=\bfseries,label=\arabic*)}


%-------------------------------------------------------------------------------
%                    - Elements cours -
%-------------------------------------------------------------------------------

%Correction d'exercice
\newcommand{\exoSec}[2]{\subsection*{Exercice #1 page #2}}
%-------------------------------------------------------------------------------
%                    - raccourcis d'écriture -
%-------------------------------------------------------------------------------

%Mise en évidence de termes clés
\newcommand{\mykw}[1]{\textcolor{red}{\underline{\textbf{#1}}}}

%Exercices
\newcommand{\exo}[2]{exercice #1 page #2}
\newcommand{\Exo}[2]{Exercice #1 page #2}

\renewcommand{\pause}{\ }


%\usepackage{fullpage}
\author{\ }
\date{15 Mars 2018}
\title{Sciences Physiques : DS n° 3}


\begin{document}
%	\usepackage{fancyhdr}
%	
%	\pagestyle{fancy}
%	\fancyhf{}
	%\rhead{Share\LaTeX}

	\maketitle
	
\begin{small}
	\begin{center}
		\begin{tabular}{|@{\ }l@{}|@{\ }c@{\ }|}
			\hline
			\textbf{Compétence} & \textbf{Maitrise} \\
			\hline
		Notions de molécules, atomes, ions. \ \ &  \ \ \ \\
			\hline
			%Conservation de la masse lors d’une transformation chimique. &  \\
			%\hline			
			Associer leurs symboles aux éléments à l’aide de la classification périodique. \ &  \\
			\hline
			Interpréter une formule chimique en termes atomiques. &  \\
			\hline
		\end{tabular}
	\end{center}
\end{small}	
	
	
%\vspace*{-0.5cm}	

Seul l'\ref{ex:structure} est à faire sur le sujet. Le soin et la qualité de rédaction sont pris en compte dans la notation.


%\section{\'Equations de réaction}

Ajuster les équations de réactions suivantes :
\begin{questions}
	\question $CH_4 + ....O_2 \rightarrow ....CO_2 + ....H_2O$
	
	\question $C_7H_{16} + ....O_2 \rightarrow ....CO_2 + ....H_2O$	
	
	\question $C_6H_{2}O + ....O_2 \rightarrow ....CO_2 + ....H_2O$
\end{questions}

%
%\section{À chaque modèle sa formule}
\begin{questions}
	\question \'A partir de ces dessins de modèles, donner la formule des molécules suivantes.

	\begin{center}
		\includegraphics[scale=0.6]{img/exemples}
	\end{center}
	\fillwithdottedlines{2cm}
	
\end{questions}

%\section{Composition des molécules (4 points)}


\begin{questions}
	\question[4] Donner la composition des molécules suivantes :
	
	\begin{parts}
		\part[1] l'éthylène $C_2H_4$	
		\fillwithdottedlines{1cm}
		\begin{solution}
			2 atomes de carbone et 4 d'hydrogène
		\end{solution}
		
		\part[1] le monoxyde d'azote $NO$	
		\fillwithdottedlines{1cm}
		\begin{solution}
			1 atome d'azote et 1 d'oxygène.
		\end{solution}
	
		\part[1] l'ozone $O_3$	
		\fillwithdottedlines{1cm}	
		\begin{solution}
			3 atomes d'oxygène
		\end{solution}
		
		\part[1] l'eau oxygénée $H_2O_2$	
		\fillwithdottedlines{1cm}
		\begin{solution}
			2 atomes d'hydrogène et 2 d'oxygène
		\end{solution}
	\end{parts}

\end{questions}


%\newpage





%\section{Compléter les phrases (2 points)}

Recopier et compléter les phrases suivantes :
\begin{questions}
	\question Le gaz d'une boisson gazeuse peut être extrait en $.....$ ou en $.....$ la boisson.
	\begin{solution}
		Le gaz d'une boisson gazeuse peut être extrait en \textbf{agitant} ou en \textbf{chauffant} la boisson.
	\end{solution}
	
	\question Le gaz dissous dans une boisson gazeuse est \textbf{le dioxyde de carbone}.
	
	
	\question Pour identifier ce gaz, on utilise de \textbf{l'eau de chaux}.
\end{questions}






\section{La corrosion du fer (5 points)}

Au contact du dioxygène $O_2$ et en présence d'eau $H_2O$, le fer $Fe$ se corrode en rouille $Fe_2O_3$. L'eau est indispensable pour ce processus, mais lors de cette transformation la quantité totale d'eau reste la même. On propose une équation pour modéliser cette réaction : 

\begin{center}
	\ce{4Fe + 4O2 -> 2Fe2O3}
\end{center}
%\begin{equation}
%	4 Fe + 4 O_2 \rightarrow 2 Fe_2O_3
%\end{equation}


\begin{questions}
	\question[1] Expliquer pourquoi l'eau n'est ni un réactif, ni un produit dans cette transformation.
	%\fillwithdottedlines{2cm}
	\begin{solution}
		Le texte indique que dans cette réaction, la quantité d'eau ne change pas donc il n'est pas nécessaire de la faire apparaître dans l'équation de réaction.
	\end{solution}
	
	\question[1] Compter le nombre d'atomes de fer dans les réactifs puis dans les produits de l'équation.
	%\fillwithdottedlines{2cm}
	\begin{solution}
		Dans les réactifs il y a 4 atomes de fer et dans les produits il y en 4 aussi.
	\end{solution}
	
	\question[1] Faire de même pour les atomes d'oxygène.
	%\fillwithdottedlines{2cm}
	\begin{solution}
		Dans les réactifs il y a 8 atomes d'oxygène et dans les produits il y en a 6.
	\end{solution}
	
	\question[2] Indiquer à l'aide des réponses précédentes, si l'équation de réaction est équilibrée. Si ce n'est pas le cas, proposer une correction de l'équation.
	%\fillwithdottedlines{3cm}
	\begin{solution}
		Il n'y a pas le même nombre d'atomes d'oxygène dans les réactifs et les produits donc l'équation de réaction n'est pas équilibrée. 
		
		Dans l'équation suivante il y a autant d'atomes de fer et d'oxygène dans les réactifs que dans les produits, elle est donc équilibrée :
		\begin{center}
			\ce{4Fe + 3O2 -> 2Fe2O3}
		\end{center}
	\end{solution}
\end{questions}


%\newpage

\section{Structure des atomes}

\begin{questions}
	\question Compléter le tableau \\
	
	\begin{tabular}{|@{\ }l@{\ }|@{\ }c@{\ }|@{\ }c@{\ }|@{\ }c@{\ }|@{\ }c@{\ }|@{\ }c@{\ }|}
			\hline
		\textbf{Nom de l'atome}                  &  $\qquad\quad\qquad$     & Chlore &  $\qquad\quad\qquad$  &  $\qquad\quad\qquad$  & $\qquad\quad\qquad$    \\ \hline
		\textbf{Symbole de l'atome}              & $He$ &        &    &    & $H$ \\ \hline
		\textbf{Nombre de protons dans le noyau} &      &        & 26 &    &     \\ \hline
		\textbf{Nombre d'électrons}              &      &        &    & 79 &    \\ \hline
	\end{tabular}
\end{questions}


%\newpage 

\section{L'atome de Fer (4 points)}

\begin{multicols}{2}
	
	
	Le métal fer est un cristal, ce qui veut dire que ses atomes sont organisés selon une structure bien particulière appelée maille élémentaire. Sur l'Atomium à Bruxelles, chaque sphère de 18 m de diamètre représente un atome de fer agrandi 64 milliards de fois.
	
	\includegraphics[scale=0.5]{img/atomium}
\end{multicols}


\begin{questions}
	\question[2] Calculer le diamètre d'un atome de fer.
	%\fillwithdottedlines{2cm}
	\begin{solution}
		64 milliards  = $64 \times 10^9$.
		\begin{eqnarray*}
			\dfrac{18}{64 \times 10^9}& = & \num{2.8125} \times 10^{-10} m \\
			& = & \num{281.25} \times 10^{-12} m \\
		\end{eqnarray*}
	Un atome de fer a un diamètre de $\num{281.25} pm$.
	\end{solution}
	  
	
	\question[1] Combien d'électrons contient-il ?
	%\fillwithdottedlines{2cm}
	\begin{solution}
		Le numéro atomique de l'atome de fer est 26, il contient 26 protons, et donc 26 électrons.
	\end{solution}
	
	\question[1] Quel est le diamètre du noyau d'un atome de fer ?
	%\fillwithdottedlines{3cm}
	\begin{solution}
		Le diamètre d'un atome est \num{100000} fois plus grand que celui de son noyau.
		
		\begin{eqnarray*}
			\dfrac{\num{2.8125} \times 10^{-10}}{10^5}& = & \num{2.8125} \times 10^{-15} m
		\end{eqnarray*}
	
	Donc le noyau d'un atome de fer a un diamètre de \num{2.8125} fm.
	\end{solution}
\end{questions}



%\newpage 

\section{Représentation de la vitesse (4 points)}

\begin{questions}
	\question Représenter la vitesse d'un objet à un instant précis, dans les conditions suivantes  :
	
	\begin{parts}
		\part[2] \begin{itemize}
			\item Mouvement : horizontal de gauche à droite;
			\item Valeur de la vitesse : 25 m/s;
			\item \'Echelle choisie: 1 cm pour 10 m/s.
			
		\end{itemize}
	
%		\makeemptybox{4cm}
		
		
		\part[2] \begin{itemize}
			\item Mouvement : chute verticale d'un objet;
			\item Valeur de la vitesse : 10 m/s;
			\item \'Echelle choisie: 1 cm pour 5 m/s.
		\end{itemize}
			
		
%		\makeemptybox{4cm}
	\end{parts}

%	\question[2] \'A l'aide d'un logiciel de traitement de vidéos, on peut repérer les positions successives prises par un point d'une grande roue lors de son mouvement.
%	
%	Sans tenir compte sa valeur, représenter la vitesse aux positions 3, 12, 22 et 33.
%	
%	\begin{center}
%		\includegraphics[scale=0.35]{roue}
%	\end{center}
\end{questions}

\newpage

\section{Représentation d'un atome (3 points)}\label{ex:avions}

Les avionneurs utilisent un métal léger pour construire certains avions. Un atome de ce métal, a pour numéro atomique Z = 13.

\begin{questions}
	\question[1] Combien cet atome possède-t-il de charges positives et négatives ?
	
	\question[1] Dessiner cette particule en représentant un électron par un point rouge et le noyau par un point bleu.
	
	\question[1] Quel est le nom de ce métal ?
	
	\question[1] Combien cet atome possède-t-il de protons ?
\end{questions}


 
%\newpage
%
%\includegraphics [scale=0.5, angle= 90 ]{img/tableau} 
\ \label{LastPage}

\end{document}