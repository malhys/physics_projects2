\section{Classement (2 points)}\label{ex:classement}

Soit huit échantillons de 10g de matériaux différents.

{\small \begin{center}
	\begin{tabular}{|@{\ }c@{\ }|@{\ }c@{\ }|}
		\hline
		\textbf{Matériau}  & \textbf{Masse volumique ($kg/m^3$)} \\ \hline
		diamant   & \num{3517}                 \\ \hline
		coton     & \num{40}                   \\ \hline
		acier     & \num{7800}                 \\ \hline
		bronze    & \num{8400}                 \\ \hline
		fer       & \num{7680}                 \\ \hline
		or        & \num{19300}                \\ \hline
		uranium   & \num{18700}                \\ \hline
		aluminium & \num{2700}                 \\ \hline
	\end{tabular}
\end{center}}

\begin{questions}
	\question[2] Classer les échantillons par ordre de volume croissant.
	
	\begin{solution}
		{\small \begin{center}
				\begin{tabular}{|@{\ }c@{\ }|@{\ }c@{\ }|@{\ }c@{\ }|@{\ }c@{\ }|@{\ }c@{\ }|}
					\hline
					\textbf{Matériau}  & \textbf{$\rho$ ($kg/m^3$)} & \textbf{$\rho$ ($g/m^3$)} & \textbf{$\rho$ ($g/cm^3$)} & Volume ($\frac{10}{\rho}$ ($cm^3$))   \\ \hline
					diamant   & \num{3517}  & \num{3517000} &  \num{3.517} & \num{2.84} \\ \hline
					coton     & \num{40}    & \num{40000}  & \num{0.04} & \num{250} \\ \hline
					acier     & \num{7800}  & \num{7800000}   & \num{7.8} & \num{1.28} \\ \hline
					bronze    & \num{8400}  & \num{8400000}   & \num{8.4} & \num{1.19} \\ \hline
					fer       & \num{7680}  & \num{7680000}  & \num{7.68} & \num{1.30} \\ \hline
					or        & \num{19300} & \num{19300000}  & \num{19.3} & \num{0.52} \\ \hline
					uranium   & \num{18700} & \num{18700000}  & \num{18.7} & \num{0.53} \\ \hline
					aluminium & \num{2700} & \num{2700000}    & \num{2.7} & \num{3.7} \\ \hline
				\end{tabular}
		\end{center}}
	
	D'où l'ordre suivant : or, uranium, bronze, acier, fer, diamant, aluminium, coton.
	\end{solution}
	
\end{questions}