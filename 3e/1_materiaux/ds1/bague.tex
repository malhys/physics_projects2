\section{Une bague en argent (4 points)}\label{ex:bague}

Florent observe la bague de Suzanne. Suzanne lui affirme que c'est une bague en argent mais Florent pense qu'elle est en fer-blanc. Pour en avoir le c\oe ur net, il pèse la bague et trouve $m = \num{14.4} \ g$. Il plonge la bague dans une éprouvette contenant$ \num{5.0} \ mL$ d'eau : le niveau monte jusqu'à $\num{6.4} \ mL$.  

\begin{questions}
	\question[1] De combien le volume d'eau dans l'éprouvette a-t-il augmenté ? En déduire la volume de la bague de Suzanne.
	
	\question[1] A l'aide des données du tableau, calculer la masse que ferait la bague si elle était en fer-blanc.
	
	\question[1] A l'aide du tableau, calculer la masse que ferait la bague si elle était en argent.
	
	\question[1] Déterminer à l'aide des réponses précédentes, si la bague de Suzanne est en argent ou en fer-blanc.

\end{questions}
