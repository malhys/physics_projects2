\section{Ordre de grandeur}\label{ex:grandeur}

Le fer a longtemps été utilisé dans a fabrication d'objets quotidiens et a servi à la réalisation de grands projets urbains de l'aire industrielle. Sachant que la masse volumique du fer est de l'ordre de 8 $g/cm^3$, donner une estimation du volume de fer nécessaire à la fabrication des objets suivants :

\begin{enumerate}
	\item Un clou d'une masse approximative de 12 g.
	\item Un fer à cheval d'une masse approximative de 500 g.
	\item Un fer à repasser d'une masse approximative de 1 kg.
	\item Un portail en fer forgé d'une masse approximative de 250 kg.
	\item La charpente métallique du pont Dom-Luis à Porto, dont la masse approximative est \num{3045} tonnes.
\end{enumerate}