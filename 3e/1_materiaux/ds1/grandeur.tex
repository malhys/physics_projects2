\section{Ordre de grandeur (5 points)}\label{ex:grandeur}

Le fer a longtemps été utilisé dans la fabrication d'objets quotidiens et a servi à la réalisation de grands projets urbains de l'aire industrielle. Sachant que la masse volumique du fer est de l'ordre de 8 $g/cm^3$, donner une estimation du volume de fer nécessaire à la fabrication des objets suivants :

\begin{enumerate}
	\item[1] Un clou d'une masse approximative de 12 g.
	
	\begin{solution}
		\begin{equation}
			12 \div 8 = 1.5
		\end{equation}
		
		Il faut \num{1.5} $cm^3$ de fer pour fabriquer un clou.
	\end{solution}
	
	\item[1] Un fer à cheval d'une masse approximative de 500 g.
	\begin{solution}
		\begin{equation}
			500 \div 8 = \num{62.5}
		\end{equation}
		
		Il faut \num{62.5} $cm^3$ de fer pour fabriquer un fer à cheval.
	\end{solution}

	\item[1] Un fer à repasser d'une masse approximative de 1 kg.
	\begin{solution}
		\begin{equation}
			1000 \div 8 = \num{125}
		\end{equation}
		
		Il faut \num{125} $cm^3$ de fer pour fabriquer un fer à repasser.
	\end{solution}

	\item[1] Un portail en fer forgé d'une masse approximative de 250 kg.
	\begin{solution}
		\begin{equation}
			125 \times 250 = \num{31250}
		\end{equation}
		
		Il faut \num{31250} $cm^3$ de fer pour fabriquer un portail, soit \num{31,250} $dm^3$.
	\end{solution}

	\item[1] La charpente métallique du pont Dom-Luis à Porto, dont la masse approximative est \num{3045} tonnes.
	\begin{solution}
		\begin{equation}
			\num{31250} \times 4 \times \num{3045} = \num{380625000}
		\end{equation}
		
		Il faut \num{380625000} $dm^3$ de fer pour fabriquer ce pont, soit \num{380625} $m^3$.
	\end{solution}
	
	
	
\end{enumerate}