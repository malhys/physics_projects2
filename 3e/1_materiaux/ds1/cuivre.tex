
\section{Monnaie en cuivre }

Au cours d'une opération de nettoyage de la plage, Romain a trouvé dix pièces de monnaie en cuivre. Il les plonge dans une éprouvette à moitié remplie d'eau. La différence de volume qu'il constate est $V = 5 cm^3$, il a trouvé sur internet que la masse volumique du cuivre est $\rho _{cuivre}$ est de $\num{8.96} g/mL$.  

\begin{questions}
	\question[] Convertir le volume $V$ des pièces de cuivre en mL.
	
	\question[] Calculer la masse $m$ de ces dix pièces de cuivre.
	
\end{questions}
