\begin{center}
	%\centering
	
	{\scshape\LARGE \textbf{DIPLÔME NATIONAL DU BREVET} \par}
	\vspace{1cm}
	{\scshape\Large \textbf{SESSION 2019}\par}
	\vspace{1.5cm}
	

	\begin{large}
		\begin{tabular}{|@{\ }c@{\ }|@{\ }c@{\ }|}
		\hline
		%\ & \ \\
		\'Epreuve : \textbf{PHYSIQUE-CHIMIE} & Série : \textbf{Générale} \\ \hline
		\ & \ \\
		Durée de l'épreuve : \textbf{30 mnutes} & 50 points \\ 
		\ & \ \\
		\hline
	\end{tabular}
	\end{large}
		
	\vspace{1cm}
	{\large\bfseries \'EPREUVE DU VENDREDI 18 JANVIER 2019}
	
	\vspace{1cm}
	{\itshape L'usage d'une calculatrice est autorisé\par}
	%\vfill
	\vspace{1.5cm}
	{\bfseries Ce sujet comporte \pageref{LastPage} pages numérotées de 1/\pageref{LastPage} à \pageref{LastPage}/\pageref{LastPage} }
	
%	\vspace{0.5cm}
%	{\bfseries Ce sujet comporte x annexes situés pages \pageref{annexe} /\pageref{LastPage} à \pageref{LastPage}/\pageref{LastPage} à remettre avec la copie.}
	
	\vspace{0.5cm}
	{\bfseries\itshape Le candidat doit s'assurer que le sujet distribué est complet. }
	
	%\vfill	
	
	%\fbox{
%		\begin{minipage}{0.9\textwidth}
%			\large
%			Il  est  rappelé  que  la  qualité  de  la  rédaction,  la  clarté  et  la  précision  des  raisonnements entreront pour une part importante dans l'appréciation des copies. 
%			
%			Cependant, le candidat est invité à faire figurer sur la copie toute trace de recherche, même incomplète ou infructueuse, qu'il aura développée. 
%		\end{minipage}
%	}
	
	%\vfill

% Bottom of the page
	%{\large \today\par}
\end{center}
\newpage