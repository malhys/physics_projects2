\section{Définitions}

\begin{questions}
	\question Quelle est l'unité de la tension électrique ?
	
	\fillwithdottedlines{1cm}
	
	\question Quel appareil utilise-t-on pour mesurer l'intensité électrique aux bornes d'un dipôle ?
	\fillwithdottedlines{1cm}
	
%	\question Comment brancher cet appareil dans le circuit ?
%	\fillwithdottedlines{1.5cm}
	
	\question Comment se comporte la tension dans un circuit qui comporte des dérivations ?
	\fillwithdottedlines{3cm}
	
	\question Comment se comporte l'intensité dans un circuit comporte des dérivations ?
	\fillwithdottedlines{3cm}
	
\end{questions}

\section{Comparaisons}

Dans chaque cas, indique si la première valeur est égale supérieure ou inférieure à la seconde :

\begin{multicols}{2}
	\begin{questions}
		
		\question \num{1500} V et \num{1.5} kV \\
		\fillwithdottedlines{1cm}
		
		\question 250 mA et \num{0.025} A\\
		\fillwithdottedlines{1cm}
		
		\question \num{0.02} kV et \num{20} V\\
		\fillwithdottedlines{1cm}
		
		\question 500 mA et \num{0.55} A\\
		\fillwithdottedlines{1cm}
		
		\question \num{4400} mA et \num{4.5} A\\
		\fillwithdottedlines{1cm}
		
		\question \num{23} mA et \num{0.23} A\\
		\fillwithdottedlines{1cm}
		
		\question \num{0.23} kV et \num{23000} mV\\		
		\fillwithdottedlines{1cm}
		
		\question \num{12.0} V et \num{1200} mV\\		
		\fillwithdottedlines{1cm}
		
	\end{questions}
	
\end{multicols}

\newpage


\section{Faire un schéma}
 
 \begin{questions}
 	\question Faire le schéma d'un circuit électrique série comprenant une pile, un interrupteur et deux lampes. Ajouter un ampèremètre entre l'interrupteur et une lampe et deux voltmètres pour mesurer la tension aux bornes de la pile et d'une des lampes.
 	
 	\makeemptybox{10cm}

 \end{questions}