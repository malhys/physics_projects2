\section{Classement de résistances (3 points)}

Voici les valeurs des résistances de cinq conducteurs ohmiques différents :
\begin{multicols}{2}
	\begin{itemize}
	\item R1 = \num{0.22} $k\Omega$
	\item R2 = \num{47} $\Omega$
	\item R3 = \num{68} $k\Omega$
	\item R4 = \num{0.1} $k\Omega$
	\item R5 = \num{200} $\Omega$
\end{itemize}
\end{multicols}

\begin{questions}
	\question Classer ces valeurs dans l’ordre croissant.
		\begin{solution}
		Je convertis ce valeurs en ohm :
		\begin{multicols}{2}
			\begin{itemize}
				\item R1 = \num{0.22} $k\Omega$ = 220 $\Omega$
				\item R2 = \num{47} $\Omega$ 
				\item R3 = \num{68} $k\Omega$ = \num{68000} $\Omega$
				\item R4 = \num{0.1} $k\Omega$ = 100 $\Omega$
				\item R5 = \num{200} $\Omega$ = 200 $\Omega$
			\end{itemize}
		\end{multicols}
		On a donc R2 < R4 < R5 < R1 < R3.
	\end{solution}
	\question Quelle résistance est la plus conductrice ?
	\begin{solution}
		Le conducteur ohmique le plus conducteur est celui qui a la résistance la plus faible. C'est donc R2.
	\end{solution}
	 
\end{questions}