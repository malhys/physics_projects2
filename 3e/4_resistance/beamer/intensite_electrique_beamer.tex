\documentclass[xcolor={dvipsnames}]{beamer}
%\usepackage[utf8]{inputenc}
%\usetheme{Madrid}
\usetheme{CambridgeUS}
\usecolortheme{}

%-------------------------------------------------------------------------------
%          -Packages nécessaires pour écrire en Français et en UTF8-
%-------------------------------------------------------------------------------
\usepackage[utf8]{inputenc}
\usepackage[french]{babel}
\usepackage[T1]{fontenc}
\usepackage{lmodern}
\usepackage{textcomp}

%-------------------------------------------------------------------------------

%-------------------------------------------------------------------------------
%                          -Outils de mise en forme-
%-------------------------------------------------------------------------------
\usepackage{hyperref}
\hypersetup{pdfstartview=XYZ}
\usepackage{enumerate}
\usepackage{graphicx}
%\usepackage{multicol}
%\usepackage{tabularx}

%\usepackage{anysize} %%pour pouvoir mettre les marges qu'on veut
%\marginsize{2.5cm}{2.5cm}{2.5cm}{2.5cm}

\usepackage{indentfirst} %%pour que les premier paragraphes soient aussi indentés
\usepackage{verbatim}
%\usepackage[table]{xcolor}  
%\usepackage{multirow}
\usepackage{ulem}
%-------------------------------------------------------------------------------


%-------------------------------------------------------------------------------
%                  -Nécessaires pour écrire des mathématiques-
%-------------------------------------------------------------------------------
\usepackage{amsfonts}
\usepackage{amssymb}
\usepackage{amsmath}
\usepackage{amsthm}
\usepackage{tikz}
\usepackage{xlop}
\usepackage[output-decimal-marker={,}]{siunitx}
%-------------------------------------------------------------------------------

%-------------------------------------------------------------------------------
%                  -Nécessaires pour écrire des formules chimiquess-
%-------------------------------------------------------------------------------

\usepackage[version=4]{mhchem}

%-------------------------------------------------------------------------------
%                    - Mise en forme 
%-------------------------------------------------------------------------------

\newcommand{\bu}[1]{\underline{\textbf{#1}}}


\usepackage{ifthen}


\newcommand{\ifTrue}[2]{\ifthenelse{\equal{#1}{true}}{#2}{$\qquad \qquad$}}

\newcommand{\kword}[1]{\textcolor{red}{\underline{#1}}}


%-------------------------------------------------------------------------------



%-------------------------------------------------------------------------------
%                    - Racourcis d'écriture -
%-------------------------------------------------------------------------------

% Angles orientés (couples de vecteurs)
\newcommand{\aopp}[2]{(\vec{#1}, \vec{#2})} %Les deuc vecteurs sont positifs
\newcommand{\aopn}[2]{(\vec{#1}, -\vec{#2})} %Le second vecteur est négatif
\newcommand{\aonp}[2]{(-\vec{#1}, \vec{#2})} %Le premier vecteur est négatif
\newcommand{\aonn}[2]{(-\vec{#1}, -\vec{#2})} %Les deux vecteurs sont négatifs

%Ensembles mathématiques
\newcommand{\naturels}{\mathbb{N}} %Nombres naturels
\newcommand{\relatifs}{\mathbb{Z}} %Nombres relatifs
\newcommand{\rationnels}{\mathbb{Q}} %Nombres rationnels
\newcommand{\reels}{\mathbb{R}} %Nombres réels
\newcommand{\complexes}{\mathbb{C}} %Nombres complexes


%Intégration des parenthèses aux cosinus
\newcommand{\cosP}[1]{\cos\left(#1\right)}
\newcommand{\sinP}[1]{\sin\left(#1\right)}

%Fractions
\newcommand{\myfrac}[2]{{\LARGE $\frac{#1}{#2}$}}

%Vocabulaire courrant
\newcommand{\cad}{c'est-à-dire}

%Droites
\newcommand{\dte}[1]{$(#1)$}
\newcommand{\fig}[1]{figure $#1$}
\newcommand{\sym}{symétrique}
\newcommand{\syms}{symétriques}
\newcommand{\asym}{axe de symétrie}
\newcommand{\asyms}{axes de symétrie}
\newcommand{\seg}[1]{$[#1]$}
\newcommand{\monAngle}[1]{$\widehat{#1}$}
\newcommand{\bissec}{bissectrice}
\newcommand{\mediat}{médiatrice}
\newcommand{\ddte}[1]{$[#1)$}

%Figures
\newcommand{\para}{parallélogramme}
\newcommand{\paras}{parallélogrammes}
\newcommand{\myquad}{quadrilatère}
\newcommand{\myquads}{quadrilatères}
\newcommand{\co}{côtés opposés}
\newcommand{\diag}{diagonale}
\newcommand{\diags}{diagonales}
\newcommand{\supp}{supplémentaires}
\newcommand{\car}{carré}
\newcommand{\cars}{carrés}
\newcommand{\rect}{rectangle}
\newcommand{\rects}{rectangles}
\newcommand{\los}{losange}
\newcommand{\loss}{losanges}


\newcommand{\homo}{homothétie}
\newcommand{\homos}{homothéties}




%----------------------------------------------------
% Environnements de cours
%------------------------------------------------------



%\usepackage{../../../../pas-math}
\usepackage{../../../moncours_beamer}





\graphicspath{{../img/}}
%Quelles sont les deux sortes de sources de lumière
\title{CH4 : La résistance électrique}
%\author{O. FINOT}\institute{Collège S$^t$ Bernard}


\AtBeginSection[]
{
	\begin{frame}
		\frametitle{}
		\tableofcontents[currentsection, hideallsubsections]
	\end{frame} 

}


%\AtBeginSubsection[]
%{
%	\begin{frame}
%		\frametitle{Sommaire}
%		\tableofcontents[currentsection, currentsubsection]
%	\end{frame} 
%}

\begin{document}

\begin{frame}
  \titlepage 
\end{frame}

\section{Intensité et tension}


\subsection{Intensité électrique}


%
%\begin{frame}
%	\begin{alertblock}
%		\begin{itemize}
%			\item L'unité de mesure de l'\kw{intensité} électrique est l' \kw{ampère} (symbole $A$). 
%			
%			\item Un \kw{ampèremètre} permet de mesurer l'intensité du courant, il se branche \kw{en série} dans le circuit.
%		\end{itemize}
%			
%			
%		\begin{center}
%			\includegraphics[scale=0.6]{img/schema1}
%		\end{center}
%	\end{alertblock}
%\end{frame}



\begin{frame}

\begin{mybilan}
	\begin{itemize}
		\item Dans un \kw{circuit série}, la valeur de l'intensité du courant est la même en tout point du circuit, quel que soit l'ordre des dipôles : c'est la \kw{loi d'unicité de l'intensité}.\pause
		
		
		\item Dans un circuit comportant des \kw{dérivations}, l'intensité du courant dans la branche principale est égale à la \kw{somme des intensités} des courants dans les \kw{branches dérivées}.		
		
	\end{itemize}
\end{mybilan}

\end{frame}


\subsection{Tension électrique}

\begin{frame}


\begin{mybilan}
	\begin{itemize}
		\item Dans un \kw{circuit série} :\pause
		\begin{itemize}
			\item la valeur de la tension entre les bornes d'un dipôle ne \kw{dépend pas de sa position} dans le circuit.\pause
			\item la valeur de la \kw{tension $U$} aux bornes du générateur est égale à la \kw{somme des valeurs des tension $U_1$ et $U_2$} entre les bornes des dipôles : c'est la \kw{loi d'additivité des tensions}.\pause
		\end{itemize}
		
		
		\item Dans un circuit comportant des \kw{dérivations}, la valeur de la \kw{tension est la même} entre les bornes des dipôles branchés en \kw{dérivation}.
		
		
	\end{itemize}
\end{mybilan}
\end{frame}


\begin{frame}
	\begin{center}
		\includegraphics[scale=0.5]{lois_elec}
	\end{center}
\end{frame}

\section{Résistance}
%
%\begin{frame}
%	\begin{myact}{4 page 127}
	\begin{enumerate}
		\item Le gaz prélevé dans la seringue a été extrait d'eau pétillante par déplacement d'eau.\pause
		\item Au début de l'expérience, la solution d'eau de chaux est incolore et transparente.\pause
		\item Après y avoir fait barboter le gaz l'eau de chaux s'est troublée.\pause
		\item Un précipité blanc s'est formé lors de cette expérience, donc le gaz dissous dans l'eau pétillante est du dioxyde de carbone.
	\end{enumerate}
\end{myact}
%\end{frame}
%
%
\begin{frame}
	\begin{mybilan}
		\begin{itemize}
			\item Un conducteur ohmique (ou résistance) est caractérisé par sa résistance électrique ($R$) (en \kw{ohm}, de symbole $\Omega$);\pause
			\item On utilise un  \kw{ohmmètre} pour mesurer la résistance d'un composant, hors du circuit;\pause
			\item Plus la résistance d'un composant est élevée, \kw{moins il est conducteur};\pause
			\item L'intensité du courant électrique \kw{diminue}, lorsqu'un conducteur ohmique est ajouté dans le circuit.
		\end{itemize}
	\end{mybilan}
\end{frame}


\section{Loi d'Ohm}

\begin{frame}
	\begin{mybilan}
		\begin{itemize}
			\item La \kw{tension ($U$)} aux bornes d'une résistance est \kw{proportionnelle à l'intensité ($I$)} qui la traverse;\pause
			
			\item C'est la \kw{loi d'Ohm} :
			
				\begin{center}
					\includegraphics[scale=0.4]{../img/loi}
				\end{center}
		\end{itemize}
	\end{mybilan}
\end{frame}

\end{document}