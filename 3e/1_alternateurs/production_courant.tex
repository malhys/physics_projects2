\documentclass[12pt,a4paper]{article}

%\usepackage[left=1.5cm,right=1.5cm,top=1cm,bottom=2cm]{geometry}
\usepackage[in, plain]{fullpage}
\usepackage{array}
%\usepackage{../../pas-math}
\usepackage{../../moncours}



%-------------------------------------------------------------------------------
%          -Packages nécessaires pour écrire en Français et en UTF8-
%-------------------------------------------------------------------------------
\usepackage[utf8]{inputenc}
\usepackage[frenchb]{babel}
%\usepackage{numprint}
\usepackage[T1]{fontenc}
%\usepackage{lmodern}
\usepackage{textcomp}
\usepackage[french, boxed]{algorithm2e}
\usepackage{hyperref}


%-------------------------------------------------------------------------------

%-------------------------------------------------------------------------------
%                          -Outils de mise en forme-
%-------------------------------------------------------------------------------
\usepackage{hyperref}
\hypersetup{pdfstartview=XYZ}
%\usepackage{enumerate}
\usepackage{graphicx}
\usepackage{multicol}
\usepackage{tabularx}
\usepackage{multirow}
\usepackage{color}
\usepackage{eurosym}


\usepackage{anysize} %%pour pouvoir mettre les marges qu'on veut
%\marginsize{2.5cm}{2.5cm}{2.5cm}{2.5cm}

\usepackage{indentfirst} %%pour que les premier paragraphes soient aussi indentés
\usepackage{verbatim}
\usepackage{enumitem}
\usepackage{booktabs}
\usepackage[usenames,dvipsnames,svgnames,table]{xcolor}

\usepackage{variations}

%-------------------------------------------------------------------------------


%-------------------------------------------------------------------------------
%                  -Nécessaires pour écrire des mathématiques-
%-------------------------------------------------------------------------------
\usepackage{amsfonts}
\usepackage{amssymb}
\usepackage{amsmath}
\usepackage{amsthm}
\usepackage{tikz}
\usepackage{xlop}
\usepackage[output-decimal-marker={,}]{siunitx}
%-------------------------------------------------------------------------------

%-------------------------------------------------------------------------------
%                  -Nécessaires pour écrire des formules chimiquess-
%-------------------------------------------------------------------------------

\usepackage[version=4]{mhchem}

%-------------------------------------------------------------------------------
% Pour pouvoir exploiter les fichiers directement dans beamer
\newcommand{\pause}{\ }
%-------------------------------------------------------------------------------
%                    - Mise en forme avancée
%-------------------------------------------------------------------------------

\usepackage{ifthen}
\usepackage{ifmtarg}


\newcommand{\ifTrue}[2]{\ifthenelse{\equal{#1}{true}}{#2}{$\qquad \qquad$}}

%\newcommand{\kword}[1]{\textcolor{red}{\underline{#1}}}
%-------------------------------------------------------------------------------

%-------------------------------------------------------------------------------
%                     -Mise en forme d'exercices-
%-------------------------------------------------------------------------------
%\newtheoremstyle{exostyle}
%{\topsep}% espace avant
%{\topsep}% espace apres
%{}% Police utilisee par le style de thm
%{}% Indentation (vide = aucune, \parindent = indentation paragraphe)
%{\bfseries}% Police du titre de thm
%{.}% Signe de ponctuation apres le titre du thm
%{ }% Espace apres le titre du thm (\newline = linebreak)
%{\thmname{#1}\thmnumber{ #2}\thmnote{. \normalfont{\textit{#3}}}}% composants du titre du thm : \thmname = nom du thm, \thmnumber = numéro du thm, \thmnote = sous-titre du thm

%\theoremstyle{exostyle}
%\newtheorem{exercice}{Exercice}
%
%\newenvironment{questions}{
%\begin{enumerate}[\hspace{12pt}\bfseries\itshape a.]}{\end{enumerate}
%} %mettre un 1 à la place du a si on veut des numéros au lieu de lettres pour les questions 
%-------------------------------------------------------------------------------

%-------------------------------------------------------------------------------
%                    - Mise en forme de tableaux -
%-------------------------------------------------------------------------------

\renewcommand{\arraystretch}{1.7}

\setlength{\tabcolsep}{1.2cm}

%-------------------------------------------------------------------------------



%-------------------------------------------------------------------------------
%                    - Racourcis d'écriture -
%-------------------------------------------------------------------------------
%Droites
\newcommand{\dte}[1]{$(#1)$}
\newcommand{\fig}[1]{figure $#1$}
\newcommand{\sym}{symétrique}
\newcommand{\syms}{symétriques}
\newcommand{\asym}{axe de symétrie}
\newcommand{\asyms}{axes de symétrie}
\newcommand{\seg}[1]{$[#1]$}
\newcommand{\monAngle}[1]{$\widehat{#1}$}
\newcommand{\bissec}{bissectrice}
\newcommand{\mediat}{médiatrice}
\newcommand{\ddte}[1]{$[#1)$}


% Angles orientés (couples de vecteurs)
\newcommand{\aopp}[2]{(\vec{#1}, \vec{#2})} %Les deuc vecteurs sont positifs
\newcommand{\aopn}[2]{(\vec{#1}, -\vec{#2})} %Le second vecteur est négatif
\newcommand{\aonp}[2]{(-\vec{#1}, \vec{#2})} %Le premier vecteur est négatif
\newcommand{\aonn}[2]{(-\vec{#1}, -\vec{#2})} %Les deux vecteurs sont négatifs

%Ensembles mathématiques
\newcommand{\naturels}{\mathbb{N}} %Nombres naturels
\newcommand{\relatifs}{\mathbb{Z}} %Nombres relatifs
\newcommand{\rationnels}{\mathbb{Q}} %Nombres rationnels
\newcommand{\reels}{\mathbb{R}} %Nombres réels
\newcommand{\complexes}{\mathbb{C}} %Nombres complexes


%Intégration des parenthèses aux cosinus
\newcommand{\cosP}[1]{\cos\left(#1\right)}
\newcommand{\sinP}[1]{\sin\left(#1\right)}


%Probas stats
\newcommand{\stat}{statistique}
\newcommand{\stats}{statistiques}


\newcommand{\homo}{homothétie}
\newcommand{\homos}{homothéties}


\newcommand{\mycoord}[3]{(\textcolor{red}{\num{#1}} ; \textcolor{Green}{\num{#2}} ; \textcolor{blue}{\num{#3}})}
%-------------------------------------------------------------------------------

%-------------------------------------------------------------------------------
%                    - Mise en page -
%-------------------------------------------------------------------------------

\newcommand{\twoCol}[1]{\begin{multicols}{2}#1\end{multicols}}


\setenumerate[1]{font=\bfseries,label=\textit{\alph*})}
\setenumerate[2]{font=\bfseries,label=\arabic*)}


%-------------------------------------------------------------------------------
%                    - Elements cours -
%-------------------------------------------------------------------------------

%Correction d'exercice
\newcommand{\exoSec}[2]{\subsection*{Exercice #1 page #2}}
%-------------------------------------------------------------------------------
%                    - raccourcis d'écriture -
%-------------------------------------------------------------------------------

%Mise en évidence de termes clés
\newcommand{\mykw}[1]{\textcolor{red}{\underline{\textbf{#1}}}}

%Exercices
\newcommand{\exo}[2]{exercice #1 page #2}
\newcommand{\Exo}[2]{Exercice #1 page #2}

\renewcommand{\pause}{\ }

\graphicspath{{./img/}}


\date{}
\title{}


\begin{document}
	
	
\chap[num=1, color=blue]{Production de courant électrique}{Olivier FINOT, \today }	

\section{Production d'une tension variable}

\section{Comment caractériser un mouvement ?}

\begin{questions}
	\question Le mouvement du tunnelier est \underline{rectiligne} et \underline{uniforme}.
	
	\question Lors du fonctionnement du tunnelier, la roue coupante a une trajectoire \underline{circulaire}.
	
	\question Lors d'un cycle de fonctionnement du tunnelier la roue :
	\begin{enumerate}
		\item commence par démarrer, donc sa vitesse augmente ;
		\item puis elle se stabilise à vitesse constante;
		\item enfin elle ralenti pour s'arrêter.
	\end{enumerate} 

	\question La roue coupante du tunnelier a donc un mouvement :
	\begin{enumerate}
		\item d'abord circulaire accéléré;
		\item ensuite circulaire uniforme;
		\item enfin circulaire ralenti;
	\end{enumerate} 
\end{questions}

\begin{mybilan}
	\begin{itemize}
		\item Dans un environnement sec, un courant électrique est dangereux à partir d'une tension de 50 V.\pause
		
		\item En France, une prise électrique fournit une tension de \kw{230 V}. Il ne faut pas toucher toucher les bornes d'une prise car cela pourrait provoquer une \kw{électrisation} voire une \kw{électrocution}.\pause
				 
	\end{itemize}

\end{mybilan}

\begin{mydefs}
	\begin{itemize}
		\item \kw{\'Electrisation} : passage du courant électrique à travers le corps humain. 
		
		\item \kw{\'Electrocution} : \'electrisation qui entraine la mort.
	\end{itemize}
\end{mydefs}




\begin{myexos}
	\begin{itemize}
		\item \exo{6}{176}
		\item \exo{16}{178}
	\end{itemize}
\end{myexos}

\section{L'alternateur de bicyclette}

\begin{myact}{}

		Activité 16 page 51 cahier d'activités

\end{myact}

\begin{mybilan}
	\begin{itemize}
		\item L'énergie peut être \kw{transférée} d'un objet vers un autre objet.
		
		\item Une forme d'énergie peut être \kw{convertie} en une autre forme d'énergie.
		
		
		\begin{center}
			\includegraphics[scale=0.8]{conversion}
		\end{center}
		
		\item On représente un ensemble de transferts et conversions d'énergie par une \kw{chaine énergétique}.

		\begin{center}
			\includegraphics[scale=0.5]{chaine}
		\end{center}
	\end{itemize}
\end{mybilan}



\begin{myexos}
	\begin{multicols}{2}
	
		\begin{itemize}
			\item \exo{7}{176}
			\item \exo{9}{176}
			\item \exo{12}{177}			
		\end{itemize}
	
	\end{multicols}
\end{myexos}


\section{Centrale hydraulique et centrale éolienne}

\begin{myact}{3 page 186}
	\begin{enumerate}
		\item Durant l'enregistrement, la tension est variable.\pause
		\item La valeur de la tension maximale est \num{4} $V$.
		\item La valeur de la tension minimale est \num{-2.3} $V$.
		\item La valeur de la période est de environ \num{130} $s$ ($170 - 40$).
		\item LA tension a une valeur nulle à $t_1$ $\approx$ 40 $s$ et $t_2$ $\approx$ 170 $s$.
		
	\end{enumerate}
\end{myact}

\begin{mybilan}
	\begin{itemize}
		\item L'unité de masse du système international est \kw{le kilogramme} ($kg$). En chimie, on utilise souvent un sous-multiple, le \kw{gramme} ($g$).\pause
		\item Si l'on pose un récipient vide sur le plateau d'une \kw{balance}, le bouton TARE permet de remettre l'affichage à 0 ; ainsi on ne tient pas compte de la masse de ce récipient.\pause
		\item Mesure d'une masse : voir fiche méthode 3 page 104 (partie 2)
	\end{itemize}
\end{mybilan}

\begin{myexos}
	\twoCol{
		\begin{itemize}
			\item \exo{3}{1752}
			\item \exo{8}{176}
		\end{itemize}
	}
\end{myexos}


\section{Centrale thermique}



\begin{myact}{4 page 187}
	\begin{enumerate}
		\item La tension observée est variable et périodique.\pause
		\item La durée entre deux valeurs successives de la tension maximale est de \num{2.5} $ms$, c'est la période.\pause
		\item $U_{max}$ = \num{7.5} $V$.\pause
		\item  $U_{min}$ = \num{-7.5} $V$.\pause
		\item 4 motifs sont représentés sur le document B.\pause
		\item Les parties où la tension est positive sont comparables à celles où elle est négative : elles se compensent.		
	\end{enumerate}
\end{myact}


\begin{mybilan}
	\begin{itemize}
		\item La masse d'un corps est \kw{proportionnelle} à son volume; \pause
		\item Le coefficient de proportionnalité est la \kw{masse volumique} (notée $\rho$);\pause
		\item \kw{Un litre d'eau} a une masse de \kw{1 kilogramme};\pause
		\item Une substance est \kw{plus dense} qu'une autre si, pour un même volume, sa masse est supérieure.		
	\end{itemize}
\end{mybilan}

\begin{myexos}
	\twoCol{\begin{itemize}
		\item \exo{4}{175}
		\item \exo{11}{177}
		\item \exo{15}{178}
	\end{itemize}}
\end{myexos}
\appendix

\newpage

\section*{Correction des exercices}

\subsection*{\Exo{4}{175}}

\begin{enumerate}[label=\arabic*.]
	\item Une tension variable peut être obtenue par \textbf{déplacement} d'un aimant au voisinage d'une \textbf{bobine}.
	\item Un alternateur transforme l'énergie \textbf{mécanique} en énergie \textbf{électrique}.
	\item L'\kw{alternateur} est la partie commune à toutes les \kw{centrales} électriques.
	\item Une \textbf{partie} de l'énergie reçue ar une centrale électrique n'est pas transformée, on dit qu'elle est <<\textbf{perdue}>>.
	\item Les combustibles qui alimentent les centrales \textbf{thermiques} sont des sources d'énergie non renouvelables.
	\item L'eau en mouvement qui alimente les centrales \textbf{hydrauliques} est une source d'énergie renouvelable.
	\item Le \textbf{vent} qui alimente les centrales éoliennes est une source d'énergie \textbf{renouvelable}.
\end{enumerate}

\subsection*{\Exo{6}{176}}

\begin{tabular}{|@{ } c @{ }|@{ } l @{ }|@{ } l @{ }|}
	\hline
	\begin{tabular}[c]{@{}c@{}}Ordre des \\ opérations\end{tabular} & \multicolumn{1}{c|}{Opérations}              & Tension           \\ \hline
	1                                                               & approcher l'aimant de la bobine              & positive          \\ \hline
	2                                                               & éloigner l'aimant de la bobine               & \textbf{négative} \\ \hline
	3                                                               & inverser les bornes du voltmètre             & \textbf{aucune}   \\ \hline
	4                                                               & éloigner la bobine de l'aimant               & \textbf{négative} \\ \hline
	5                                                               & approcher la bobine de l'aimant              & \textbf{positive} \\ \hline
	6                                                               & déplacer solidairement l'aimant et la bobine & \textbf{aucune}   \\ \hline
\end{tabular}

\subsection*{\Exo{7}{176}}

lampe allumée : énergie électrique.\\
turbine en rotation, jet de vapeur, jet d'eau, vent, pales d'une éolienne en rotation : énergie mécanique.

\subsection*{\exo{8}{176}}

\begin{enumerate}[label=\arabic*.]
	\item Dans une centrale hydraulique, 20 \%  ($ 100 - 80 = 20 $) de l'énergie n'est pas convertie, elle est <<perdue>> sous forme d'énergie mécanique.
	\item Dans une centrale éolienne, 40 \%  ($ 100 - 60 = 40 $) de l'énergie n'est pas convertie, elle est <<perdue>> sous forme d'énergie mécanique.
	\item  \ \\ \begin{center}
		\includegraphics[scale=0.5]{exo8}
	\end{center}
\end{enumerate}

\subsection*{\exo{9}{176}}

\begin{center}
	\includegraphics[scale=0.5]{exo9}
\end{center}

\subsection*{\exo{11}{177}}

\begin{center}
	\includegraphics[scale=0.6]{exo11}
\end{center}

\subsection*{\exo{12}{177}}

\begin{enumerate}[label=\arabic*.]
	\item Quand les feux sont allumés, une partie de l'énergie mécanique qu'il fournit est convertie en énergie électrique donc il doit fournir plus d'énergie musculaire.
	\item En modifiant la position du levier de vitesse, il doit fournir une quantité d'énergie différente pour faire tourner les roues à la même vitesse.
	\item Si une lampe s'éteint il y aura besoin de moins d'énergie électrique donc il aura besoin de faire moins d'efforts pour conserver une vitesse constante.
\end{enumerate}

\subsection*{\exo{15}{178}}

\begin{enumerate}[label=\arabic*.]
	\item L'essence fourni de l'énergie thermique au moteur.
	\item Non toute l'énergie thermique fournie par l'essence n'est pas convertie, elle est perdue dans le circuit de refroidissement.
	\item \ \\ \begin{center}
		\includegraphics[scale=0.4]{exo15}
	\end{center}
\end{enumerate}


\subsection*{\exo{16}{178}}

%\twoCol{
\begin{enumerate}[label=\arabic*.]
	\item La tension change de sens chaque fois qu'une dent passe devant la bobine car quand une dent arrive devant la bobine l'aimant s'approche, puis lorsqu'elle continue son parcours, il s'éloigne.
	\item Il y a 12 dents sur le disque, donc quand la roue fait un tour la tension prend 12 fois le même signe. 
	\item $240 \div 12 = 20$, donc si la tension prend 240 fois le même signe pendant une seconde, la roue fait 20 tours.
	\item $20 \times $ \num{0.8} $=$ \num{16}, donc en 1 seconde, le véhicule parcourt \num{16} $m$ sa vitesse est donc \num{16} $m/s$ soit \num{57600} $km/h$ (16 $\times $ \num{3600} $=$ \num{57600}).
\end{enumerate}
%}

\end{document}