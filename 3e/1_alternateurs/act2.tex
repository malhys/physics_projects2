\begin{myact}{2 page 169}
	\begin{enumerate}
		\item Les deux éléments principaux de l'alternateur de bicyclette sont un aimant et une bobine. La bobine est fixe, et l'aimant mobile.\pause
		\item Pour faire fonctionner l'alternateur, l'élève actionne une manivelle.\pause
		\item Lorsque l'on ferme l'interrupteur on observe que l'ampoule est allumée en faisant tourner le galet. Elle est éteinte si il ne tourne pas. \pause
		\item Le galet fait tourner l'aimant de l'alternateur devant la bobine.\pause
		\item En observant l'ampoule on peut conclure qu'actionner le galet produit une tension électrique.\pause
		\item L'énergie mécanique fait tourner l'alternateur qui produit de l'énergie électrique.
	\end{enumerate}
\end{myact}