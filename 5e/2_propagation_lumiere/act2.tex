\begin{myact}{2 page 77}
	\begin{enumerate}
		\item Sur les parois du vase, on observe des zone éclairées.\pause
		\item La fumée fait apparaître le faisceau lumineux produit par la source de lumière.\pause
		\item Les bords du faisceau de lumière rendu visible par la fumée sont rectilignes (droits).\pause
		\item Entre les points A et B, la lumière se déplace en ligne droite.\pause
		\item Les parois ne sont pas parfaitement transparentes et diffusent donc la lumière reçue. Sur la photo B, la lumière est diffusée par les particules de fumée.\pause
		\item Un rayon de lumière est représente par une droite. Une flèche sur cette droite indique le sens de propagation de la lumière de la source à l'objet éclairé.\pause
		\item La représentation d'un faisceau de lumière ne traduit pas la réalité, on représente uniquement les deux rayons extrêmes.
	\end{enumerate}
\end{myact}