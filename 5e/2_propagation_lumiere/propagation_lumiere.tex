\documentclass[12pt,a4paper]{article}

%\usepackage[left=1.5cm,right=1.5cm,top=1cm,bottom=2cm]{geometry}
\usepackage[in, plain]{fullpage}
\usepackage{array}
%\usepackage{../../pas-math}
\usepackage{../../moncours}



%-------------------------------------------------------------------------------
%          -Packages nécessaires pour écrire en Français et en UTF8-
%-------------------------------------------------------------------------------
\usepackage[utf8]{inputenc}
\usepackage[frenchb]{babel}
%\usepackage{numprint}
\usepackage[T1]{fontenc}
%\usepackage{lmodern}
\usepackage{textcomp}
\usepackage[french, boxed]{algorithm2e}
\usepackage{hyperref}


%-------------------------------------------------------------------------------

%-------------------------------------------------------------------------------
%                          -Outils de mise en forme-
%-------------------------------------------------------------------------------
\usepackage{hyperref}
\hypersetup{pdfstartview=XYZ}
%\usepackage{enumerate}
\usepackage{graphicx}
\usepackage{multicol}
\usepackage{tabularx}
\usepackage{multirow}
\usepackage{color}
\usepackage{eurosym}


\usepackage{anysize} %%pour pouvoir mettre les marges qu'on veut
%\marginsize{2.5cm}{2.5cm}{2.5cm}{2.5cm}

\usepackage{indentfirst} %%pour que les premier paragraphes soient aussi indentés
\usepackage{verbatim}
\usepackage{enumitem}
\usepackage{booktabs}
\usepackage[usenames,dvipsnames,svgnames,table]{xcolor}

\usepackage{variations}

%-------------------------------------------------------------------------------


%-------------------------------------------------------------------------------
%                  -Nécessaires pour écrire des mathématiques-
%-------------------------------------------------------------------------------
\usepackage{amsfonts}
\usepackage{amssymb}
\usepackage{amsmath}
\usepackage{amsthm}
\usepackage{tikz}
\usepackage{xlop}
\usepackage[output-decimal-marker={,}]{siunitx}
%-------------------------------------------------------------------------------

%-------------------------------------------------------------------------------
%                  -Nécessaires pour écrire des formules chimiquess-
%-------------------------------------------------------------------------------

\usepackage[version=4]{mhchem}

%-------------------------------------------------------------------------------
% Pour pouvoir exploiter les fichiers directement dans beamer
\newcommand{\pause}{\ }
%-------------------------------------------------------------------------------
%                    - Mise en forme avancée
%-------------------------------------------------------------------------------

\usepackage{ifthen}
\usepackage{ifmtarg}


\newcommand{\ifTrue}[2]{\ifthenelse{\equal{#1}{true}}{#2}{$\qquad \qquad$}}

%\newcommand{\kword}[1]{\textcolor{red}{\underline{#1}}}
%-------------------------------------------------------------------------------

%-------------------------------------------------------------------------------
%                     -Mise en forme d'exercices-
%-------------------------------------------------------------------------------
%\newtheoremstyle{exostyle}
%{\topsep}% espace avant
%{\topsep}% espace apres
%{}% Police utilisee par le style de thm
%{}% Indentation (vide = aucune, \parindent = indentation paragraphe)
%{\bfseries}% Police du titre de thm
%{.}% Signe de ponctuation apres le titre du thm
%{ }% Espace apres le titre du thm (\newline = linebreak)
%{\thmname{#1}\thmnumber{ #2}\thmnote{. \normalfont{\textit{#3}}}}% composants du titre du thm : \thmname = nom du thm, \thmnumber = numéro du thm, \thmnote = sous-titre du thm

%\theoremstyle{exostyle}
%\newtheorem{exercice}{Exercice}
%
%\newenvironment{questions}{
%\begin{enumerate}[\hspace{12pt}\bfseries\itshape a.]}{\end{enumerate}
%} %mettre un 1 à la place du a si on veut des numéros au lieu de lettres pour les questions 
%-------------------------------------------------------------------------------

%-------------------------------------------------------------------------------
%                    - Mise en forme de tableaux -
%-------------------------------------------------------------------------------

\renewcommand{\arraystretch}{1.7}

\setlength{\tabcolsep}{1.2cm}

%-------------------------------------------------------------------------------



%-------------------------------------------------------------------------------
%                    - Racourcis d'écriture -
%-------------------------------------------------------------------------------
%Droites
\newcommand{\dte}[1]{$(#1)$}
\newcommand{\fig}[1]{figure $#1$}
\newcommand{\sym}{symétrique}
\newcommand{\syms}{symétriques}
\newcommand{\asym}{axe de symétrie}
\newcommand{\asyms}{axes de symétrie}
\newcommand{\seg}[1]{$[#1]$}
\newcommand{\monAngle}[1]{$\widehat{#1}$}
\newcommand{\bissec}{bissectrice}
\newcommand{\mediat}{médiatrice}
\newcommand{\ddte}[1]{$[#1)$}


% Angles orientés (couples de vecteurs)
\newcommand{\aopp}[2]{(\vec{#1}, \vec{#2})} %Les deuc vecteurs sont positifs
\newcommand{\aopn}[2]{(\vec{#1}, -\vec{#2})} %Le second vecteur est négatif
\newcommand{\aonp}[2]{(-\vec{#1}, \vec{#2})} %Le premier vecteur est négatif
\newcommand{\aonn}[2]{(-\vec{#1}, -\vec{#2})} %Les deux vecteurs sont négatifs

%Ensembles mathématiques
\newcommand{\naturels}{\mathbb{N}} %Nombres naturels
\newcommand{\relatifs}{\mathbb{Z}} %Nombres relatifs
\newcommand{\rationnels}{\mathbb{Q}} %Nombres rationnels
\newcommand{\reels}{\mathbb{R}} %Nombres réels
\newcommand{\complexes}{\mathbb{C}} %Nombres complexes


%Intégration des parenthèses aux cosinus
\newcommand{\cosP}[1]{\cos\left(#1\right)}
\newcommand{\sinP}[1]{\sin\left(#1\right)}


%Probas stats
\newcommand{\stat}{statistique}
\newcommand{\stats}{statistiques}


\newcommand{\homo}{homothétie}
\newcommand{\homos}{homothéties}


\newcommand{\mycoord}[3]{(\textcolor{red}{\num{#1}} ; \textcolor{Green}{\num{#2}} ; \textcolor{blue}{\num{#3}})}
%-------------------------------------------------------------------------------

%-------------------------------------------------------------------------------
%                    - Mise en page -
%-------------------------------------------------------------------------------

\newcommand{\twoCol}[1]{\begin{multicols}{2}#1\end{multicols}}


\setenumerate[1]{font=\bfseries,label=\textit{\alph*})}
\setenumerate[2]{font=\bfseries,label=\arabic*)}


%-------------------------------------------------------------------------------
%                    - Elements cours -
%-------------------------------------------------------------------------------

%Correction d'exercice
\newcommand{\exoSec}[2]{\subsection*{Exercice #1 page #2}}
%-------------------------------------------------------------------------------
%                    - raccourcis d'écriture -
%-------------------------------------------------------------------------------

%Mise en évidence de termes clés
\newcommand{\mykw}[1]{\textcolor{red}{\underline{\textbf{#1}}}}

%Exercices
\newcommand{\exo}[2]{exercice #1 page #2}
\newcommand{\Exo}[2]{Exercice #1 page #2}

\renewcommand{\pause}{\ }




\date{}
\title{}

\graphicspath{{./img/}}


\begin{document}
	
	
\chap[num=2, color=blue]{\small Comment se propage la lumière et se forment les ombres ?}{Olivier FINOT, \today }	

\section{Propagation de la lumière}

\section{Comment caractériser un mouvement ?}

\begin{questions}
	\question Le mouvement du tunnelier est \underline{rectiligne} et \underline{uniforme}.
	
	\question Lors du fonctionnement du tunnelier, la roue coupante a une trajectoire \underline{circulaire}.
	
	\question Lors d'un cycle de fonctionnement du tunnelier la roue :
	\begin{enumerate}
		\item commence par démarrer, donc sa vitesse augmente ;
		\item puis elle se stabilise à vitesse constante;
		\item enfin elle ralenti pour s'arrêter.
	\end{enumerate} 

	\question La roue coupante du tunnelier a donc un mouvement :
	\begin{enumerate}
		\item d'abord circulaire accéléré;
		\item ensuite circulaire uniforme;
		\item enfin circulaire ralenti;
	\end{enumerate} 
\end{questions}

\begin{mybilan}
	\begin{itemize}
		\item Dans un environnement sec, un courant électrique est dangereux à partir d'une tension de 50 V.\pause
		
		\item En France, une prise électrique fournit une tension de \kw{230 V}. Il ne faut pas toucher toucher les bornes d'une prise car cela pourrait provoquer une \kw{électrisation} voire une \kw{électrocution}.\pause
				 
	\end{itemize}

\end{mybilan}

\begin{mydefs}
	\begin{itemize}
		\item \kw{\'Electrisation} : passage du courant électrique à travers le corps humain. 
		
		\item \kw{\'Electrocution} : \'electrisation qui entraine la mort.
	\end{itemize}
\end{mydefs}




\begin{myexos}
	\begin{itemize}
		\item \exo{1}{84}
		\item \exo{6}{85}
	\end{itemize}
\end{myexos}

\section{Modélisation du trajet de la lumière}

\begin{myact}{}

		Activité 16 page 51 cahier d'activités

\end{myact}

\begin{mybilan}
	\begin{itemize}
		\item L'énergie peut être \kw{transférée} d'un objet vers un autre objet.
		
		\item Une forme d'énergie peut être \kw{convertie} en une autre forme d'énergie.
		
		
		\begin{center}
			\includegraphics[scale=0.8]{conversion}
		\end{center}
		
		\item On représente un ensemble de transferts et conversions d'énergie par une \kw{chaine énergétique}.

		\begin{center}
			\includegraphics[scale=0.5]{chaine}
		\end{center}
	\end{itemize}
\end{mybilan}



\begin{myexos}
	\begin{multicols}{2}
	
		\begin{itemize}
			\item \exo{2}{84}
			\item \exo{7}{85}
			
		\end{itemize}
	
	\end{multicols}
\end{myexos}


\section{Zone éclairée et zone d'ombre}

\begin{myact}{3 page 186}
	\begin{enumerate}
		\item Durant l'enregistrement, la tension est variable.\pause
		\item La valeur de la tension maximale est \num{4} $V$.
		\item La valeur de la tension minimale est \num{-2.3} $V$.
		\item La valeur de la période est de environ \num{130} $s$ ($170 - 40$).
		\item LA tension a une valeur nulle à $t_1$ $\approx$ 40 $s$ et $t_2$ $\approx$ 170 $s$.
		
	\end{enumerate}
\end{myact}

\begin{mybilan}
	\begin{itemize}
		\item L'unité de masse du système international est \kw{le kilogramme} ($kg$). En chimie, on utilise souvent un sous-multiple, le \kw{gramme} ($g$).\pause
		\item Si l'on pose un récipient vide sur le plateau d'une \kw{balance}, le bouton TARE permet de remettre l'affichage à 0 ; ainsi on ne tient pas compte de la masse de ce récipient.\pause
		\item Mesure d'une masse : voir fiche méthode 3 page 104 (partie 2)
	\end{itemize}
\end{mybilan}

\begin{myexos}
	\begin{itemize}
		\item \exo{3}{84}
		\item \exo{8}{85}
	\end{itemize}
\end{myexos}


\section{Ombre propre et ombre portée}



\begin{myact}{4 page 187}
	\begin{enumerate}
		\item La tension observée est variable et périodique.\pause
		\item La durée entre deux valeurs successives de la tension maximale est de \num{2.5} $ms$, c'est la période.\pause
		\item $U_{max}$ = \num{7.5} $V$.\pause
		\item  $U_{min}$ = \num{-7.5} $V$.\pause
		\item 4 motifs sont représentés sur le document B.\pause
		\item Les parties où la tension est positive sont comparables à celles où elle est négative : elles se compensent.		
	\end{enumerate}
\end{myact}


\begin{mybilan}
	\begin{itemize}
		\item La masse d'un corps est \kw{proportionnelle} à son volume; \pause
		\item Le coefficient de proportionnalité est la \kw{masse volumique} (notée $\rho$);\pause
		\item \kw{Un litre d'eau} a une masse de \kw{1 kilogramme};\pause
		\item Une substance est \kw{plus dense} qu'une autre si, pour un même volume, sa masse est supérieure.		
	\end{itemize}
\end{mybilan}

\begin{myexos}
	\twoCol{\begin{itemize}
		\item \exo{4}{84}
		\item \exo{10}{86}
		\item \exo{11}{86}
		\item \exo{13}{86}
	\end{itemize}}
\end{myexos}
\appendix

\newpage

\section*{Correction des exercices}

\subsection*{\Exo{1}{84}}

\begin{enumerate}[label=\alph*)]
	\item Dans l'air d'une salle de classe, la lumière se propage en \textbf{ligne droite}.
	\item L'air qui nous entoure est \textbf{transparent} car il \textbf{se laisse traverser} par la lumière.
	\item Un milieu est homogène lorsqu'il est \textbf{identique} en chacun de ses points.
	\item Dans un milieu \textbf{homogène et transparent}, la lumière se propage en ligne droite.
\end{enumerate}

\subsection*{\Exo{2}{84}}
	\begin{center}
		\includegraphics[scale=0.7]{exo2}
	\end{center}

\subsection*{\Exo{3}{84}}
	\twoCol{\begin{enumerate}
		\item source ponctuelle de lumière
		\item faisceau de lumière
		\item rayon de lumière
		\item zone éclairée
		\item cône d'ombre
	\end{enumerate}}
	
\subsection*{\Exo{4}{84}}

\begin{enumerate}[label=\alph*)]
	\item Si l'on éclaire une petite éolienne, alors on observe sur celle-ci une ombre \textbf{propre}, et sur l'écran derrière une ombre \textbf{portée}.
	\item Pour modifier la forme de l'ombre portée de l'éolienne on peut modifier \textbf{l'orientation de l'éolienne}.
	\item Si l'on regarde l'éolienne depuis son cône d'ombre, on \textbf{ne voit pas} la source de lumière.
\end{enumerate}

\subsection*{\Exo{6}{85}}

	\begin{enumerate}[label=\alph*)]
		\item La lumière se propage en ligne droite dans un milieu homogène et transparent.
		\item La lumière se déplace en ligne droite dans l'eau de l'aquarium donc c'est un milieu homogène et transparent.
		\item Si l'on ajoutait du sucre dans l'eau, le milieu ne serait plus homogène, et donc la lumière ne s'y propagerait plus ne ligne droite.
	\end{enumerate}	
	
\subsection*{\Exo{7}{85}}
	\begin{enumerate}[label=\alph*)] 
		\item Les phares de la voiture sont la source de lumière.
		\item La lumière est visible car elle est diffusée par le brouillard.
	\end{enumerate}

\subsection*{\Exo{8}{85}}
	\begin{enumerate}[label=\alph*)]
		\item % L'ouverture C semble être dans le cône d'ombre du meuble et pas les A et B. Donc elle devrait pouvoir identifier la source de lumière depuis les ouvertures A ou B.
		
		\item \begin{center}
			\includegraphics[scale=0.5]{exo8}
		\end{center}
		
	Les rayons de lumière émis par la bougie qui traversent la porte passent par les trous A et B, le trou C se trouve dans le cône d'ombre du meuble. Le trou A est trop haut pour l'inspectrice, elle devra donc se placer devant le trou B.
	\end{enumerate}

\subsection*{\Exo{10}{86}}
	\begin{enumerate}[label=\alph*)]
		\item \begin{center}
			\includegraphics[scale=0.4]{exo10}
		\end{center}
	
		\item Les trous M et O se trouvent dans les cônes d'ombre des sphères, donc la source de lumière est visible uniquement depuis le trou N.
	\end{enumerate}

\subsection*{\Exo{11}{86}}
	\begin{enumerate}
		\item La source de lumière qui éclaire la Terre est le Soleil.
		\item \begin{center}
			\includegraphics[scale=0.4]{exo11}
		\end{center}
	\end{enumerate}

\subsection*{\Exo{13}{86}}
	\begin{enumerate}
		\item Une source de lumière (la lampe), éclaire un obstacle en papier blanc (objet éclairé) qui est placé devant un écran blanc. L'ombre porté de l'objet éclairé se forme sur l'écran.
		\item L'ombre propre se trouve sur l'arrière de l'obstacle en papier blanc, l'ombre portée sur l'écran et le cône d'ombre entre les deux.
	\end{enumerate}


\end{document}]