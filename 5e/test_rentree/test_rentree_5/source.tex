\documentclass[a4paper]{article}

\usepackage[utf8x]{inputenc}    
\usepackage[T1]{fontenc}
\usepackage{multicol}
\usepackage[output-decimal-marker={,}]{siunitx}

\usepackage[francais,bloc,completemulti]{automultiplechoice}    
\begin{document}
	
	\exemplaire{10}{    
		
		%%% debut de l'en-tête des copies :    
		
		\noindent{\bf Sciences Physiques  \hfill Cinquième}
		
		\vspace*{.5cm}
		\begin{minipage}{.4\linewidth}
			\centering\large\bf Test de rentrée\\ 04/09/2017\end{minipage}
		\champnom{\fbox{    
				\begin{minipage}{.5\linewidth}
					Nom et prénom :
					
					\vspace*{.5cm}\dotfill
					\vspace*{1mm}
				\end{minipage}
			}}
			
			\begin{center}\em
				Durée : 10 minutes.
				
				%  Aucun document n'est autorisé.
				%  L'usage de la calculatrice est interdit.
				
				%  Les questions faisant apparaître le symbole \multiSymbole{} peuvent
				%  présenter zéro, une ou plusieurs bonnes réponses. Les autres ont
				%  une unique bonne réponse.
				%
				%  Des points négatifs pourront être affectés à de \emph{très
				%    mauvaises} réponses.
			\end{center}
			\vspace{1ex}
			
			%%% fin de l'en-tête
			\begin{multicols}{2}
				
				\begin{question}{rang}    
					En s'éloignant du Soleil, la Terre est la~?
					\begin{reponses}
						\bonne{Troisième planète}
						\mauvaise{Première planète}
						\mauvaise{Cinquième planète}
					\end{reponses}
				\end{question}
				
				\begin{question}{nature}    
					La Terre est une planète~?
					\begin{reponses}
						\bonne{Tellurique}
						\mauvaise{Gazeuse}
						\mauvaise{Nébuleuse}
					\end{reponses}
				\end{question}
				
				
				\begin{question}{neptune}    
					Neptune est la planète~?
					\begin{reponses}
						\bonne{La plus proche du Soleil}
						\mauvaise{La plus proche de la Terre}
						\mauvaise{La plus éloignée du Soleil}
					\end{reponses}
				\end{question}
				
				\begin{question}{grande}    
					Quelle est la plus grande planète du système solaire~?
					\begin{reponses}		
						\mauvaise{Mercure}
						\bonne{Jupiter}
						\mauvaise{Neptune}		
					\end{reponses}
				\end{question}
				
				\begin{question}{etoile}    
					Combien d'étoiles comporte le système solaire~?
					\begin{reponses}		
						\mauvaise{Plus de 100}
						\mauvaise{10}
						\bonne{1}
					\end{reponses}
				\end{question}
				
				\begin{question}{eloignement}    
					Plus on s'éloigne du Soleil et plus ...
					\begin{reponses}		
						\mauvaise{La température est élevée}
						\bonne{La température est faible}
						\mauvaise{Il y a d'êtres vivants}
						
					\end{reponses}
				\end{question}
				
				\begin{question}{equinoxe}    
					Le 21 mars c'est :
					\begin{reponses}		
						\mauvaise{Le solstice de printemps}
						\bonne{L'équinoxe de printemps}
						\mauvaise{Pâques}
						
					\end{reponses}
				\end{question}
				
				\begin{question}{phrase}    
					Quelle phrase est vraie ?
					\begin{reponses}		
						\mauvaise{Le Soleil tourne autour de la Terre.}
						\bonne{La Terre tourne autour du Soleil.}
						\mauvaise{La Lune tourne autour du Soleil.}
						
					\end{reponses}
				\end{question}
				
				
				\begin{question}{etats}
					Quels sont les différents états physiques de la matière ?
					\begin{reponses}
						\mauvaise{Rouge, Vert et Bleu}
						\bonne{Solide, Liquide et Gazeux}
						\mauvaise{Résistant, Souple et Dur} 
					\end{reponses}
				\end{question}
				
				
				\begin{question}{metaux}
					Les métaux :
					\begin{reponses}
						\mauvaise{sont tous attirés par un aimant.}
						\bonne{n'ont pas tous les mêmes propriétés.}
						\mauvaise{ont tous la même couleur.} 
					\end{reponses}
				\end{question}
				
				\begin{question}{air}
					L'air qui nous entoure :
					\begin{reponses}
						\mauvaise{est composé d'un seul gaz}
						\bonne{est une mélange de plusieurs gaz}
						\mauvaise{ne contient pas de gaz} 
					\end{reponses}
				\end{question}
				
				\begin{question}{repartitionAir}
					Quel gaz est le plus présent dans l'air ?
					\begin{reponses}
						\mauvaise{l'oxygène}
						\bonne{l'azote}
						\mauvaise{l'argon} 
					\end{reponses}
				\end{question}
				
				\begin{question}{separation}
					Quel méthode ne permet pas de séparer les composants d'un mélange ?
					\begin{reponses}
						\mauvaise{La décantation}
						\mauvaise{La filtration}
						\bonne{L'agitation} 
					\end{reponses}
				\end{question}
				
				\begin{question}{marais}
					Quelle méthode est utilisée dans les marais salants pour extraire le sel présent dans l'eau de mer ?\begin{reponses}
						\bonne{L'évaporation}
						\mauvaise{La filtration}
						\mauvaise{La décantation}
					\end{reponses}
				\end{question}
				
				\begin{question}{trajectoire}
					Si la trajectoire d'un objet suit une ligne droite, son mouvement est :
					\begin{reponses}
						\bonne{rectiligne}
						\mauvaise{curviligne}
						\mauvaise{circulaire}
					\end{reponses}
				\end{question}
				
				\begin{question}{vitesse}
					Si la vitesse d'un objet diminue, on dit qu'il :
					\begin{reponses}
						\mauvaise{accélère}
						\bonne{décélère}
						\mauvaise{s'arrête}
					\end{reponses}
				\end{question}
				
				\begin{question}{uniteVitesse}
					La vitesse d'un objet peut se mesurer en :
					\begin{reponses}
						\mauvaise{s/m}
						\bonne{m/s}
						\mauvaise{h/km}
						\mauvaise{kg/h}
					\end{reponses}
				\end{question}
				
				\begin{question}{mesureVitesse}
					Pour vérifier si un objet accélère ou décélère :
					\begin{reponses}
						\mauvaise{je regarde si l'objet avance ou recule.}
						\mauvaise{je vérifie qu'il n'est pas immobile.}
						\bonne{je compare sa vitesse à différents instants.}
					\end{reponses}
				\end{question}
				
				
				\begin{question}{conversionDist}
					\num{0.25} $m$ équivaut à :
					\begin{reponses}
						\mauvaise{\num{250} $cm$}
						\mauvaise{\num{2500} $mm$}
						\bonne{\num{25} $cm$}
					\end{reponses}
				\end{question}
				
				
				
				\begin{question}{conversionSurf}
					\num{1000000} $m²$ équivaut à :
					\begin{reponses}
						\bonne{\num{1} $km²$}
						\mauvaise{\num{10} $km²$}
						\mauvaise{\num{1000} $km²$}
						
					\end{reponses}
				\end{question}
				
			\end{multicols} 
			
			\clearpage    
			
			
		}  
		
	\end{document}
