\begin{myact}{1 page 124}
	\begin{enumerate}
		\item Non les glaçons n'ont pas la forme du récipient qui les contient.\pause
		\item Le liquide obtenu lorsque les glaçons ont fondu a la forme du récipient.\pause
		\item La surface libre du liquide est plane.\pause
		\item On peut saisir un glaçon avec ses doigts, mais pas de l'eau liquide.\pause
		\item Lorsqu'il est placé dans des récipients de forme différentes, un solide conserve sa forme.\pause
		\item Un solide a une forme propre parce qu'elle ne change pas.\pause
		\item Un liquide placé dans dans des récipients de formes différentes prend la forme de ces récipients.\pause
		\item Le fil à plomb indique la direction verticale, donc le petit côté de l'équerre indique la direction horizontale. On en déduit que la surface libre d'un liquide au repos est horizontale.
	\end{enumerate}
\end{myact}