\begin{myact}{2 page 125}
	\begin{enumerate}
		\item Lorsque l'eau bout, il se forme de la vapeur dans l'erlenmeyer.\pause
		\item La vapeur d'eau emprisonnée dans l'erlenmeyer occupe tout l'espace disponible.\pause
		\item Quand les deux erlenmeyers sont en communication, on voit apparaître de la buée sur la paroi, car la vapeur est montée dans le deuxième erlenmeyer.\pause
		\item Après la mise en communication, la vapeur occupe l'espace des deux erlenmeyers.\pause
		\item Lorsque l'on appuie sur le piston, le volume d'air contenu dans la seringue fermée diminue.\pause
		\item Non, la vapeur d'eau n'a pas de forme propre.\pause
		\item La vapeur d'eau est expansible car lorsque l'on ajoute le second erlenmeyer, elle l'occupe en plus du premier.\pause
		\item Lorsque l'on appuie sur le piston de la seringue fermée, le volume d'air diminue, l'air est donc compressible.
	\end{enumerate}
\end{myact}