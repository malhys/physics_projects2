\section{Qu'est ce que la température (5 points) }\label{ex:temperture}

En physique, la température d'un corps représente l'agitation des molécules qui composent ce corps : plus les molécules bougent et plus le corps est chaud.

\begin{questions}
	\question[1] Au zéro absolu [\num{-273.15} \degree C], les molécules peuvent-elles bouger ?
	
	\question[2] Expliquer pourquoi, pour une matière donnée, l'état solide est plus froid que l'état liquide, lui même plus froids que l'état gazeux.
	
	\question[2] À l'état solide, les molécules sont très proches les unes des autres et fortement attachées les unes aux autres. C'est la raison pour laquelle les solides ont une forme et un volume propre : les molécules ne se déplacent presque pas les unes par rapport aux autres : elles se déplacent en blocs. \\ Expliquer comment l'augmentation de température permet de passer à l'état liquide.
\end{questions}