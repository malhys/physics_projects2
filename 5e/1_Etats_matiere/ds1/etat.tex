
\section{Quel est cet état ? (3 points)}\label{ex:etat}



\begin{questions}
	\question[3] Pour chaque phrase, indiquer quel(s) état(s) est (sont) décrit(s).
	
	
	Les molécules :\\
	\begin{parts}
		\part sont proches les unes des autres et peuvent bouger les unes par rapport aux autres.
		\begin{solution}
			L'état décrit est l'état liquide.
		\end{solution}
		
		\part sont très éloignées les unes des autres.
		\begin{solution}
			L'état décrit est l'état gazeux.
		\end{solution}
		
%		\part sont ordonnées [sont liées].
%		\begin{solution}
%			L'état décrit est l'état solide.
%		\end{solution}
%		
		\part ne peuvent pas se déplacer les unes par rapport aux autres.
		\begin{solution}
			L'état décrit est l'état solide.
		\end{solution}
		
		\part se déplacent et occupent le maximum d'espace.
		\begin{solution}
			L'état décrit est l'état gazeux.
		\end{solution}
		
		\part ont un volume propre et pas de forme propre.
		\begin{solution}
			L'état décrit est l'état solide.
		\end{solution}
		
		\part sont désordonnées [sont agitées].
		\begin{solution}
			Les états décrits sont les états liquide et gazeux.
		\end{solution}
	\end{parts}
	
\end{questions}
