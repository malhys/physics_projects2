\begin{mybilan}
	\begin{itemize}
		\item Le \kw{volume} représente l'espace occupé par une substance.\pause
		\item L'unité de volume du système international est le \kw{mètre cube} ($m^3$). L'unité de volume la plus utilisée pour un liquide est le \kw{litre} ($L$).\pause
		\item Un litre et un décimètre cube représentent le même volume : \kw{1 $L =$ 1 $dm^3$}. Si on divise ces volumes par \num{1000}, l'égalité est toujours vérifiée : 1$mL$ = 1 $cm^3$\pause
		\item Mesure d'un volume : voir fiche méthode 3 page 104 (partie 1)
	\end{itemize}
\end{mybilan}