\section{L'aquarium (3 points)}

Mina veut remplir d'eau l'aquarium qu'elle vient d'acheter. Ses dimensions sont les suivantes :
\begin{itemize}
	\item longueur = 40 cm;
	\item largeur = 25 cm ;
	\item hauteur = 30 cm.
\end{itemize} 
\begin{questions}
	\question[2] Elle utilise un seau de 5 $L$. Combien doit-elle en verser pour emplir l'aquarium ?
	\begin{solution}
		Calcul du volume de l'aquarium :
		
		\begin{eqnarray*}
			V &= & longueur \times largeur \times hauteur \\
			V &=& 40 \times 25 \times 30 \\
			V &=& \num{30000} 
		\end{eqnarray*}
	
	L'aquarium contient \num{30000} $cm^3$ d'eau, soit 30 L. Mina devra donc verser 6 seaux d'eau dans l'aquarium pour le remplir (30 $\div$ 5).
	\end{solution}
	
	\question[1] Quelle masse d'eau contient l'aquarium ?
	\begin{solution}
		1 L d'eau a une masse de 1 kg, donc l'aquarium de 30 L contient 30 kg d'eau.
	\end{solution}
\end{questions}