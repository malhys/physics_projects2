\section{L'or de Max (5 points)}

Max dispose d'un lot de 12 pièces de collection et souhaite vérifier qu'elles sont en or pur. Il a lu dans son livre de Physique que 1 $dm^3$ d'or avait une masse de \num{19.3} $kg$. Il possède une éprouvette graduée de 100 $mL$ et une balance. Il sait que :

\begin{itemize}
	\item 1 $dm^3$ de plomb a une masse de \num{11.34} $kg$;
	\item 1 $dm^3$ de nickel a une masse de \num{8.9} $kg$;
\end{itemize} 
\begin{questions}
	\question[1] Quelles grandeurs Max doit-il mesurer afin de vérifier le métal dont les pièces sont faites.
	\begin{solution}
		Max doit mesurer la masse et le volume de ses pièces.
	\end{solution}
	
	\question[1] Expliquer pourquoi c'est mieux de mesurer le volume d'au moins 10 pièces en même temps dans l'éprouvette graduée.
	\begin{solution}
		Le volume d'une pièce est proche de la précision de l'éprouvette, en mesurant le volume de 10 pièce on aura une mesure plus précise.
	\end{solution}
	
	\question[1] 10 pièces ont un volume $V = 14\; mL$ et la masse d'une pièce est $m  = \num{12,46}\; g$. Calcule la masse de 1 $dm^3$ de pièces.
	\begin{solution}
		1 pièce a une masse de $\num{12,46}\; g$, donc 10 pièces auront une masse de $\num{124,6}\; g$. soit $\num{0,1246}\; kg$. Ces 1à pièces occupent $14 mL$ soit $\num{0.014} L$ donc $\num{0.014} dm^3$ .
		
		\begin{equation*}
			\num{0,1246} \div \num{0.014} = \num{8.9}
		\end{equation*}
		
		1 $dm^3$ de pièce a donc une masse de \num{8.9} kg.
	\end{solution}


	\question[1] Les pièces de Max sont-elles en or ?
	\begin{solution}
		$\num{8.9} \neq \num{19.3}$, donc les pièces ne sont pas en or.
	\end{solution}
	
	\question[1] Dans quel métal sont elles faites ?
	\begin{solution}
		Ces pièces ont une masse volumique de $\num{8.9} kg/dm^3$, elles sont donc en nickel.
	\end{solution}
\end{questions}