\documentclass[12pt,a4paper]{article}

%\usepackage[left=1.5cm,right=1.5cm,top=1cm,bottom=2cm]{geometry}
\usepackage[in, plain]{fullpage}
\usepackage{array}
%\usepackage{../../pas-math}
\usepackage{../../moncours}



%-------------------------------------------------------------------------------
%          -Packages nécessaires pour écrire en Français et en UTF8-
%-------------------------------------------------------------------------------
\usepackage[utf8]{inputenc}
\usepackage[frenchb]{babel}
%\usepackage{numprint}
\usepackage[T1]{fontenc}
%\usepackage{lmodern}
\usepackage{textcomp}
\usepackage[french, boxed]{algorithm2e}
\usepackage{hyperref}


%-------------------------------------------------------------------------------

%-------------------------------------------------------------------------------
%                          -Outils de mise en forme-
%-------------------------------------------------------------------------------
\usepackage{hyperref}
\hypersetup{pdfstartview=XYZ}
%\usepackage{enumerate}
\usepackage{graphicx}
\usepackage{multicol}
\usepackage{tabularx}
\usepackage{multirow}
\usepackage{color}
\usepackage{eurosym}


\usepackage{anysize} %%pour pouvoir mettre les marges qu'on veut
%\marginsize{2.5cm}{2.5cm}{2.5cm}{2.5cm}

\usepackage{indentfirst} %%pour que les premier paragraphes soient aussi indentés
\usepackage{verbatim}
\usepackage{enumitem}
\usepackage{booktabs}
\usepackage[usenames,dvipsnames,svgnames,table]{xcolor}

\usepackage{variations}

%-------------------------------------------------------------------------------


%-------------------------------------------------------------------------------
%                  -Nécessaires pour écrire des mathématiques-
%-------------------------------------------------------------------------------
\usepackage{amsfonts}
\usepackage{amssymb}
\usepackage{amsmath}
\usepackage{amsthm}
\usepackage{tikz}
\usepackage{xlop}
\usepackage[output-decimal-marker={,}]{siunitx}
%-------------------------------------------------------------------------------

%-------------------------------------------------------------------------------
%                  -Nécessaires pour écrire des formules chimiquess-
%-------------------------------------------------------------------------------

\usepackage[version=4]{mhchem}

%-------------------------------------------------------------------------------
% Pour pouvoir exploiter les fichiers directement dans beamer
\newcommand{\pause}{\ }
%-------------------------------------------------------------------------------
%                    - Mise en forme avancée
%-------------------------------------------------------------------------------

\usepackage{ifthen}
\usepackage{ifmtarg}


\newcommand{\ifTrue}[2]{\ifthenelse{\equal{#1}{true}}{#2}{$\qquad \qquad$}}

%\newcommand{\kword}[1]{\textcolor{red}{\underline{#1}}}
%-------------------------------------------------------------------------------

%-------------------------------------------------------------------------------
%                     -Mise en forme d'exercices-
%-------------------------------------------------------------------------------
%\newtheoremstyle{exostyle}
%{\topsep}% espace avant
%{\topsep}% espace apres
%{}% Police utilisee par le style de thm
%{}% Indentation (vide = aucune, \parindent = indentation paragraphe)
%{\bfseries}% Police du titre de thm
%{.}% Signe de ponctuation apres le titre du thm
%{ }% Espace apres le titre du thm (\newline = linebreak)
%{\thmname{#1}\thmnumber{ #2}\thmnote{. \normalfont{\textit{#3}}}}% composants du titre du thm : \thmname = nom du thm, \thmnumber = numéro du thm, \thmnote = sous-titre du thm

%\theoremstyle{exostyle}
%\newtheorem{exercice}{Exercice}
%
%\newenvironment{questions}{
%\begin{enumerate}[\hspace{12pt}\bfseries\itshape a.]}{\end{enumerate}
%} %mettre un 1 à la place du a si on veut des numéros au lieu de lettres pour les questions 
%-------------------------------------------------------------------------------

%-------------------------------------------------------------------------------
%                    - Mise en forme de tableaux -
%-------------------------------------------------------------------------------

\renewcommand{\arraystretch}{1.7}

\setlength{\tabcolsep}{1.2cm}

%-------------------------------------------------------------------------------



%-------------------------------------------------------------------------------
%                    - Racourcis d'écriture -
%-------------------------------------------------------------------------------
%Droites
\newcommand{\dte}[1]{$(#1)$}
\newcommand{\fig}[1]{figure $#1$}
\newcommand{\sym}{symétrique}
\newcommand{\syms}{symétriques}
\newcommand{\asym}{axe de symétrie}
\newcommand{\asyms}{axes de symétrie}
\newcommand{\seg}[1]{$[#1]$}
\newcommand{\monAngle}[1]{$\widehat{#1}$}
\newcommand{\bissec}{bissectrice}
\newcommand{\mediat}{médiatrice}
\newcommand{\ddte}[1]{$[#1)$}


% Angles orientés (couples de vecteurs)
\newcommand{\aopp}[2]{(\vec{#1}, \vec{#2})} %Les deuc vecteurs sont positifs
\newcommand{\aopn}[2]{(\vec{#1}, -\vec{#2})} %Le second vecteur est négatif
\newcommand{\aonp}[2]{(-\vec{#1}, \vec{#2})} %Le premier vecteur est négatif
\newcommand{\aonn}[2]{(-\vec{#1}, -\vec{#2})} %Les deux vecteurs sont négatifs

%Ensembles mathématiques
\newcommand{\naturels}{\mathbb{N}} %Nombres naturels
\newcommand{\relatifs}{\mathbb{Z}} %Nombres relatifs
\newcommand{\rationnels}{\mathbb{Q}} %Nombres rationnels
\newcommand{\reels}{\mathbb{R}} %Nombres réels
\newcommand{\complexes}{\mathbb{C}} %Nombres complexes


%Intégration des parenthèses aux cosinus
\newcommand{\cosP}[1]{\cos\left(#1\right)}
\newcommand{\sinP}[1]{\sin\left(#1\right)}


%Probas stats
\newcommand{\stat}{statistique}
\newcommand{\stats}{statistiques}


\newcommand{\homo}{homothétie}
\newcommand{\homos}{homothéties}


\newcommand{\mycoord}[3]{(\textcolor{red}{\num{#1}} ; \textcolor{Green}{\num{#2}} ; \textcolor{blue}{\num{#3}})}
%-------------------------------------------------------------------------------

%-------------------------------------------------------------------------------
%                    - Mise en page -
%-------------------------------------------------------------------------------

\newcommand{\twoCol}[1]{\begin{multicols}{2}#1\end{multicols}}


\setenumerate[1]{font=\bfseries,label=\textit{\alph*})}
\setenumerate[2]{font=\bfseries,label=\arabic*)}


%-------------------------------------------------------------------------------
%                    - Elements cours -
%-------------------------------------------------------------------------------

%Correction d'exercice
\newcommand{\exoSec}[2]{\subsection*{Exercice #1 page #2}}
%-------------------------------------------------------------------------------
%                    - raccourcis d'écriture -
%-------------------------------------------------------------------------------

%Mise en évidence de termes clés
\newcommand{\mykw}[1]{\textcolor{red}{\underline{\textbf{#1}}}}

%Exercices
\newcommand{\exo}[2]{exercice #1 page #2}
\newcommand{\Exo}[2]{Exercice #1 page #2}

\renewcommand{\pause}{\ }

%Intervalles
\newcommand{\interOO}[2]{$]$#1 , #2$[$}
\newcommand{\interOF}[2]{$]$#1 , #2$]$}
\newcommand{\interFO}[2]{$[$#1 , #2$[$}
\newcommand{\interFF}[2]{$[$#1 , #2$]$}





\date{}
\title{}


\begin{document}
	
	
\chap[num=3, color=blue]{Où trouve-t-on de l'eau sur Terre ?}{Olivier FINOT, \today }	

\section{L'eau dans notre environnement}

%\begin{myact}{1 page 110-111}
%	\begin{enumerate}
%		\item Les grands réservoirs d'eau visibles sur ces documents sont \kw{les océans} et \kw{la banquise}.
%		\item Les palmiers arrivent à pousser en plein désert car \kw{ils ont de faibles besoins en eau}.
%		\item Les boissons nous sont nécessaires pour \kw{renouveler l'eau dans notre corps}.
%		\item \kw{Le c\oe ur} est l'organe du corps humain qui contient le plus d'eau.
%		\item \kw{La peau} est l'organe du corps humain qui contient le moins d'eau.
%	\end{enumerate}
%\end{myact}

\begin{mybilan}
	Les océans, les fleuves et les glaces polaires sont les grands réservoirs d'eau de la Terre. Tous les êtres vivants contiennent de l'eau, ils en ont besoin pour vivre. \kw{L'eau est présente partout autour de nous, sans elle la vie ne pourrait pas exister}.
\end{mybilan}

\begin{myexos}
	\begin{itemize}
		\item 1 page 118
		\item 9 page 119
		\item 11 page 120
	\end{itemize}
\end{myexos}

\section{Détecter la présence d'eau}

%\begin{myact}{2 page 112}
%	\begin{enumerate}
%		\item Le sulfate de cuivre anhydre est blanc.
%		\item Lorsque l'eau rentre en contact avec le sulfate de cuivre anhydre il devient bleu.
%		\item Lorsque l'on chauffe le sulfate de cuivre hydraté, il redevient blanc.
%		\item L'eau fait changer la couleur du sulfate de cuivre.
%		\item Le sulfate de cuivre hydraté change de couleur quand on le chauffe parce l'eau s'évapore.
%		\item Le sulfate de cuivre anhydre vire au bleu en présence d'eau.
%	\end{enumerate}
%\end{myact}

\begin{mybilan}
	En présence d'eau, le \kw{sulfate de cuivre anhydre} blanc, \kw{devient bleu}.
	Lorsque l'on chauffe le sulfate de cuivre hydraté, \kw{l'eau s'évapore, il redevient blanc}.
	Le sulfate de cuivre anhydre est utilisé pour effectuer le \kw{test de reconnaissance de l'eau.}
\end{mybilan}

\begin{myexos}
	\begin{multicols}{2}
	
		\begin{itemize}
			\item 2 page 118
			\item 3 page 118
			\item 4 page 118
			\item 8 page 119
			\item 16 page 121
		\end{itemize}
	
	\end{multicols}
\end{myexos}


\section{L'eau dans les produits alimentaires}

\begin{mybilan}
	\begin{itemize}
		\item De nombreux produits alimentaires contiennent de l'eau, comme par exemple les \oe ufs, les fruits ou les boissons.
		\item Certains produits alimentaires, même liquides, ne contiennent pas d'eau, comme l'huile par exemple.
	\end{itemize}
	
\end{mybilan}

\begin{myexos}
	\begin{multicols}{2}
		\begin{itemize}
			\item 7 page 119
			\item 9 page 119
			\item 13 page 120
			\item 17 page 121
		\end{itemize}
	\end{multicols}
\end{myexos}
\appendix

\newpage

\section*{Correction des exercices}

\subsection*{Exercice 1 p 118}

\begin{enumerate}[label=\alph*)]
	\item Dans ce document, il y a de l'eau dans : la mer, les pêcheurs, les poissons.
	\item Le grand réservoir d'eau qui figure sur cette photo est une mer.
\end{enumerate}

\subsection*{Exercice 2 p 118}

\begin{multicols}{2}
	\begin{itemize}
		\item qui ne contient pas d'eau : anhydre
		\item sulfate de cuivre hydraté : bleu
		\item sulfate de cuivre anhydre : blanc
		\item qui a perdu de l'eau : déshydraté
		\item qui a gagné de l'eau : hydraté
	\end{itemize}
\end{multicols}

\subsection*{Exercice 3 p 118}
\begin{multicols}{2}
	\begin{enumerate}[label=\arabic*)]
		\item sulfate de cuivre anhydre
		\item sulfate de cuivre hydraté
		\item eau
		\item pipette
	\end{enumerate}
\end{multicols}

\subsection*{Exercice 4 p 118}


	\begin{enumerate}[label=\alph*)]
		\item Pour détecter la présence d'eau de l'eau dans un liquide, on utilise du sulfate de cuivre \mykw{anhydre}.
		
		\item Le sulfate de cuivre anhydre est de couleur \mykw{bleue}.
		\item Si on verse un peu de liquide sur du sulfate de cuivre anhydre et s'il reste blanc, alors le liquide \kw{ne contient pas} d'eau.
		\item L'eau est \mykw{indispensable} à la vie. Le corps humain a un besoin vital \mykw{d'eau}.
		\item Parmi les grands réservoirs d'eau, on trouve \mykw{les océans}.
		\item Le constituant principal des boissons est \mykw{l'eau}.
	\end{enumerate}



\exoSec{8}{119}

\begin{enumerate}[label=\alph*)]
	\item Au début de l'expérience, le sulfate de cuivre est blanc.
	\item Après quelques jours d'expérience, le sulfate de cuivre est bleu.
	\item Cette expérience prouve qu'il ya de l'eau dans l'air.
\end{enumerate}

\exoSec{9}{119}

\begin{enumerate}[label=\alph*)]
	\item Aliments de l'exercice classés par ordre croissant de teneur en eau :
	1) pain, 2) viande, 3) \oe uf / banane 5) pomme de terre, 6) poissons, 7) tomate, 8) laitue.
	\item La variété de légume qui contient le plus d'eau est la laitue
\end{enumerate}

\newpage

\subsection*{\Exo{11}{120}}

\begin{enumerate}
	\item  Le lave linge consomme au minimum 70 L pour chaque lavage. La famille effectue 5 lavages par semaine,un mois comporte 4 semaines. On a donc $4 \times 5 = 20$ lavages par mois. $20 \times 70 =\num{ 1400}$ La machine à laver le linge consomme donc au minimum $\num{1400}$ litres d'eau par mois.
	
	\item Un bain nécessite au minimum 150 L d'eau, et une douche 60 L. $150 \div 60 = \num{2.5}$, avec le volume d'eau nécessaire à un bain on peut prendre 2 douches courtes.
	
	\item Si une personne prend une douche au lieu d'un bain, elle économise 90 L d'eau. Pour un mois de 30 jours avec une douche par jour, cela représente une économie de \num{1800} litres d'eau ($30 \times 60 = \num{1800}$). Cela correspond à 60 douches supplémentaires. (Pour 28 jours : 1680 L d'eau économisés, soit 56 douches).
\end{enumerate}

\subsection*{\Exo{13}{120}}
\begin{enumerate}[label=\alph*)]
	\item Quantité d'eau absorbée par Yasmine et Sandrine pendant le repas :
		\begin{itemize}
			\item 50 g de pâtes : $70 \div 2 = 35$ , 35 ml;
			\item 100 g de filets de dinde : 70 ml;
			\item 20 cL (= 200 mL) de lait : $81 \times 2 = 162$, 162 mL;
			\item 80g de pomme : $84 \times \num{0.80} = \num{67.2}$, \num{67,2} mL.
		\end{itemize}
		$35 + 70 + 162 + \num{67.2} = \num{334,2}$, chacune a absorbé \num{343.2} mL d'eau pendant le repas.
	
	
	\item Quantité d'eau absorbée par Karim :
		
		\begin{itemize}
			\item 50 g de pâtes : $70 \div 2 = 35$ , 35 ml;
			\item 100 g de filets de dinde : 70 ml;
			\item 2 pommes de 80 g : $2 \times \num{67.2} = \num{134.4}$, \num{134.4} mL;
			\item 1 verre d'eau de 125 mL : 125 mL.
		\end{itemize}
		$35 + 70 + \num{134.4} + 125 = \num{364,4}$, il a donc absorbé plus d'eau que Yasmine et Sandrine.
\end{enumerate}
\end{document}]