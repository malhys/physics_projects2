\documentclass[xcolor={dvipsnames}]{beamer}
%\usepackage[utf8]{inputenc}
%\usetheme{Madrid}
\usetheme{CambridgeUS}
\usecolortheme{}

%-------------------------------------------------------------------------------
%          -Packages nécessaires pour écrire en Français et en UTF8-
%-------------------------------------------------------------------------------
\usepackage[utf8]{inputenc}
\usepackage[french]{babel}
\usepackage[T1]{fontenc}
\usepackage{lmodern}
\usepackage{textcomp}

%-------------------------------------------------------------------------------

%-------------------------------------------------------------------------------
%                          -Outils de mise en forme-
%-------------------------------------------------------------------------------
\usepackage{hyperref}
\hypersetup{pdfstartview=XYZ}
\usepackage{enumerate}
\usepackage{graphicx}
%\usepackage{multicol}
%\usepackage{tabularx}

%\usepackage{anysize} %%pour pouvoir mettre les marges qu'on veut
%\marginsize{2.5cm}{2.5cm}{2.5cm}{2.5cm}

\usepackage{indentfirst} %%pour que les premier paragraphes soient aussi indentés
\usepackage{verbatim}
%\usepackage[table]{xcolor}  
%\usepackage{multirow}
\usepackage{ulem}
%-------------------------------------------------------------------------------


%-------------------------------------------------------------------------------
%                  -Nécessaires pour écrire des mathématiques-
%-------------------------------------------------------------------------------
\usepackage{amsfonts}
\usepackage{amssymb}
\usepackage{amsmath}
\usepackage{amsthm}
\usepackage{tikz}
\usepackage{xlop}
\usepackage[output-decimal-marker={,}]{siunitx}
%-------------------------------------------------------------------------------

%-------------------------------------------------------------------------------
%                  -Nécessaires pour écrire des formules chimiquess-
%-------------------------------------------------------------------------------

\usepackage[version=4]{mhchem}

%-------------------------------------------------------------------------------
%                    - Mise en forme 
%-------------------------------------------------------------------------------

\newcommand{\bu}[1]{\underline{\textbf{#1}}}


\usepackage{ifthen}


\newcommand{\ifTrue}[2]{\ifthenelse{\equal{#1}{true}}{#2}{$\qquad \qquad$}}

\newcommand{\kword}[1]{\textcolor{red}{\underline{#1}}}


%-------------------------------------------------------------------------------



%-------------------------------------------------------------------------------
%                    - Racourcis d'écriture -
%-------------------------------------------------------------------------------

% Angles orientés (couples de vecteurs)
\newcommand{\aopp}[2]{(\vec{#1}, \vec{#2})} %Les deuc vecteurs sont positifs
\newcommand{\aopn}[2]{(\vec{#1}, -\vec{#2})} %Le second vecteur est négatif
\newcommand{\aonp}[2]{(-\vec{#1}, \vec{#2})} %Le premier vecteur est négatif
\newcommand{\aonn}[2]{(-\vec{#1}, -\vec{#2})} %Les deux vecteurs sont négatifs

%Ensembles mathématiques
\newcommand{\naturels}{\mathbb{N}} %Nombres naturels
\newcommand{\relatifs}{\mathbb{Z}} %Nombres relatifs
\newcommand{\rationnels}{\mathbb{Q}} %Nombres rationnels
\newcommand{\reels}{\mathbb{R}} %Nombres réels
\newcommand{\complexes}{\mathbb{C}} %Nombres complexes


%Intégration des parenthèses aux cosinus
\newcommand{\cosP}[1]{\cos\left(#1\right)}
\newcommand{\sinP}[1]{\sin\left(#1\right)}

%Fractions
\newcommand{\myfrac}[2]{{\LARGE $\frac{#1}{#2}$}}

%Vocabulaire courrant
\newcommand{\cad}{c'est-à-dire}

%Droites
\newcommand{\dte}[1]{$(#1)$}
\newcommand{\fig}[1]{figure $#1$}
\newcommand{\sym}{symétrique}
\newcommand{\syms}{symétriques}
\newcommand{\asym}{axe de symétrie}
\newcommand{\asyms}{axes de symétrie}
\newcommand{\seg}[1]{$[#1]$}
\newcommand{\monAngle}[1]{$\widehat{#1}$}
\newcommand{\bissec}{bissectrice}
\newcommand{\mediat}{médiatrice}
\newcommand{\ddte}[1]{$[#1)$}

%Figures
\newcommand{\para}{parallélogramme}
\newcommand{\paras}{parallélogrammes}
\newcommand{\myquad}{quadrilatère}
\newcommand{\myquads}{quadrilatères}
\newcommand{\co}{côtés opposés}
\newcommand{\diag}{diagonale}
\newcommand{\diags}{diagonales}
\newcommand{\supp}{supplémentaires}
\newcommand{\car}{carré}
\newcommand{\cars}{carrés}
\newcommand{\rect}{rectangle}
\newcommand{\rects}{rectangles}
\newcommand{\los}{losange}
\newcommand{\loss}{losanges}


\newcommand{\homo}{homothétie}
\newcommand{\homos}{homothéties}




%----------------------------------------------------
% Environnements de cours
%------------------------------------------------------



%\usepackage{../../../../pas-math}
\usepackage{../../../moncours_beamer}





\graphicspath{{../img/}}
%Quelles sont les deux sortes de sources de lumière
\title[Où trouve-t-on de l'eau sur Terre ?]{Chapitre 3 : Où trouve-t-on de l'eau sur Terre ?}
\author{O. FINOT}\institute{Collège S$^t$ Bernard}


\AtBeginSection[]
{
	\begin{frame}
		\frametitle{}
		\tableofcontents[currentsection, hideallsubsections]
	\end{frame} 

}


%\AtBeginSubsection[]
%{
%	\begin{frame}
%		\frametitle{Sommaire}
%		\tableofcontents[currentsection, currentsubsection]
%	\end{frame} 
%}

\begin{document}

\begin{frame}
  \titlepage 
\end{frame}

\section{L'eau dans notre environnement}

\begin{frame}
	\begin{alertblock}
		
		\begin{itemize}
			\item Les océans, les fleuves et les glaces polaires sont les grands réservoirs d'eau de la Terre.
			\item Tous les êtres vivants contiennent de l'eau, ils en ont besoin pour vivre.
			\item \kw{L'eau est présente partout autour de nous}, \kw{sans elle la vie ne pourrait pas exister}.
		\end{itemize}
		
		
		  
	\end{alertblock}
	
\end{frame}


\section{Détecter la présence d'eau}

\begin{frame}
\begin{alertblock}
	
	\begin{itemize}
		\item En présence d'eau, le \kw{sulfate de cuivre anhydre} blanc, \kw{devient bleu}.
		\item Lorsque l'on chauffe le sulfate de cuivre hydraté, \kw{l'eau s'évapore, il redevient blanc}.
		\item Le sulfate de cuivre anhydre est utilisé pour effectuer le \kw{test de reconnaissance de l'eau.}.
	\end{itemize}
	
	
	
\end{alertblock}

\end{frame}


\section{L'eau dans les produits alimentaires}

\begin{frame}
\begin{alertblock}
	
	\begin{itemize}
		\item De nombreux produits alimentaires contiennent de l'eau, comme par exemple les \oe ufs, les fruits ou les boissons.
		\item Certains produits alimentaires, même liquides, ne contiennent pas d'eau, comme l'huile par exemple.
	\end{itemize}
	
	
	
\end{alertblock}

\end{frame}
%\begin{frame}
%	\begin{myact}{1 page 14}
	\begin{enumerate}
		\item L'atmosphère est la couche d'air de faible épaisseur qui entoure la Terre. \pause
		\item L'atmosphère a une épaisseur moyenne de 600 km, soit environ $\frac{1}{10}$ du rayon de la Terre. Elle est formée de 5 couches.\pause
		\item La couche d'ozone nous protège des rayons UV, elle se situe dans la stratosphère.\pause
		\item Nous vivons dans la troposphère, elle contient l'air que l'on respire.\pause
		\item La troposphère mesure en moyenne 15 km d'épaisseur soit environ $\frac{1}{40}$ de l'atmosphère et $\frac{1}{400}$ du rayon de la Terre.\pause
		\item Les autres couches de l'atmosphère ne contiennent que très peu d'air, nous ne pourrions pas y vivre. 
	\end{enumerate}
\end{myact}
%\end{frame}
%
%
%\begin{frame}
%	\begin{mybilan}
	\begin{itemize}
		\item Dans un environnement sec, un courant électrique est dangereux à partir d'une tension de 50 V.\pause
		
		\item En France, une prise électrique fournit une tension de \kw{230 V}. Il ne faut pas toucher toucher les bornes d'une prise car cela pourrait provoquer une \kw{électrisation} voire une \kw{électrocution}.\pause
				 
	\end{itemize}

\end{mybilan}

\begin{mydefs}
	\begin{itemize}
		\item \kw{\'Electrisation} : passage du courant électrique à travers le corps humain. 
		
		\item \kw{\'Electrocution} : \'electrisation qui entraine la mort.
	\end{itemize}
\end{mydefs}
%\end{frame}
%


%\section{Les propriétés des gaz}
%
%\begin{frame}
%	\begin{myact}{2 page 125}
	\begin{enumerate}
		\item Lorsque l'eau bout, il se forme de la vapeur dans l'erlenmeyer.\pause
		\item La vapeur d'eau emprisonnée dans l'erlenmeyer occupe tout l'espace disponible.\pause
		\item Quand les deux erlenmeyers sont en communication, on voit apparaître de la buée sur la paroi, car la vapeur est montée dans le deuxième erlenmeyer.\pause
		\item Après la mise en communication, la vapeur occupe l'espace des deux erlenmeyers.\pause
		\item Lorsque l'on appuie sur le piston, le volume d'air contenu dans la seringue fermée diminue.\pause
		\item Non, la vapeur d'eau n'a pas de forme propre.\pause
		\item La vapeur d'eau est expansible car lorsque l'on ajoute le second erlenmeyer, elle l'occupe en plus du premier.\pause
		\item Lorsque l'on appuie sur le piston de la seringue fermée, le volume d'air diminue, l'air est donc compressible.
	\end{enumerate}
\end{myact}
%\end{frame}
%
%
%\begin{frame}
%	\begin{mybilan}
	\begin{itemize}
		\item L'énergie peut être \kw{transférée} d'un objet vers un autre objet.
		
		\item Une forme d'énergie peut être \kw{convertie} en une autre forme d'énergie.
		
		
		\begin{center}
			\includegraphics[scale=0.8]{conversion}
		\end{center}
		
		\item On représente un ensemble de transferts et conversions d'énergie par une \kw{chaine énergétique}.

		\begin{center}
			\includegraphics[scale=0.5]{chaine}
		\end{center}
	\end{itemize}
\end{mybilan}


%\end{frame}
%
%\section{Des gaz dans l'eau}
%
%\begin{frame}
%	\begin{myact}{3 page 126}
	\begin{enumerate}
		\item Au début de l'expérience le tube à essais contient de l'eau.\pause
		\item Au cours de l'expérience, dans le tube à essais des bulles apparaissent et le niveau de l'eau diminue.\pause
		\item Le bain-marie est à \num{57.6} °C.\pause
		\item Non il n'est pas nécessaire de faire beaucoup chauffer l'eau pétillante pour en récupérer le gaz.\pause
		\item Au cours de l'expérience, l'eau des tubes à essais est remplacée par du gaz.\pause
		\item Le gaz dégagé est récupéré par déplacement d'eau car il prend la place de l'eau contenue dans le tube à essais.\pause
		\item Pour récupérer le gaz contenu dans le l'eau pétillante on peut l'agiter  ou la chauffer.
	\end{enumerate}
\end{myact}
%\end{frame}
%
%
%\begin{frame}
%	\begin{mybilan}
	\begin{columns}[c]
		
		\begin{column}{0.5\textwidth}
			\begin{itemize}
				\item L'eau peut contenir des \kw{gaz dissous}.
				\item On peut extraire ce gaz de l'eau qui le contient par \kw{agitation} ou par \kw{chauffage}.
				\item Le gaz est extrait par \kw{déplacement d'eau}, il prend la place de l'eau contenue dans le tube à essais.
			\end{itemize}				
		\end{column}
		\begin{column}{0.5\textwidth}
			\includegraphics[scale=0.4]{recueilgaz}
		\end{column}

	
\end{columns}.
\end{mybilan}
%\end{frame}
%
%\section{Reconnaître le dioxyde de carbone}
%
%\begin{frame}
%	\begin{myact}{4 page 127}
	\begin{enumerate}
		\item Le gaz prélevé dans la seringue a été extrait d'eau pétillante par déplacement d'eau.\pause
		\item Au début de l'expérience, la solution d'eau de chaux est incolore et transparente.\pause
		\item Après y avoir fait barboter le gaz l'eau de chaux s'est troublée.\pause
		\item Un précipité blanc s'est formé lors de cette expérience, donc le gaz dissous dans l'eau pétillante est du dioxyde de carbone.
	\end{enumerate}
\end{myact}
%\end{frame}
%
%
%\begin{frame}
%	\begin{mybilan}
	\begin{itemize}
		\item La masse d'un corps est \kw{proportionnelle} à son volume; \pause
		\item Le coefficient de proportionnalité est la \kw{masse volumique} (notée $\rho$);\pause
		\item \kw{Un litre d'eau} a une masse de \kw{1 kilogramme};\pause
		\item Une substance est \kw{plus dense} qu'une autre si, pour un même volume, sa masse est supérieure.		
	\end{itemize}
\end{mybilan}
%\end{frame}

\end{document}