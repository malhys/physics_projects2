\documentclass[12pt,a4paper]{article}

%\usepackage[left=1.5cm,right=1.5cm,top=1cm,bottom=2cm]{geometry}
\usepackage[in, plain]{fullpage}
\usepackage{array}
%\usepackage{../../pas-math}
\usepackage{../../moncours}



%-------------------------------------------------------------------------------
%          -Packages nécessaires pour écrire en Français et en UTF8-
%-------------------------------------------------------------------------------
\usepackage[utf8]{inputenc}
\usepackage[frenchb]{babel}
%\usepackage{numprint}
\usepackage[T1]{fontenc}
%\usepackage{lmodern}
\usepackage{textcomp}
\usepackage[french, boxed]{algorithm2e}
\usepackage{hyperref}


%-------------------------------------------------------------------------------

%-------------------------------------------------------------------------------
%                          -Outils de mise en forme-
%-------------------------------------------------------------------------------
\usepackage{hyperref}
\hypersetup{pdfstartview=XYZ}
%\usepackage{enumerate}
\usepackage{graphicx}
\usepackage{multicol}
\usepackage{tabularx}
\usepackage{multirow}
\usepackage{color}
\usepackage{eurosym}


\usepackage{anysize} %%pour pouvoir mettre les marges qu'on veut
%\marginsize{2.5cm}{2.5cm}{2.5cm}{2.5cm}

\usepackage{indentfirst} %%pour que les premier paragraphes soient aussi indentés
\usepackage{verbatim}
\usepackage{enumitem}
\usepackage{booktabs}
\usepackage[usenames,dvipsnames,svgnames,table]{xcolor}

\usepackage{variations}

%-------------------------------------------------------------------------------


%-------------------------------------------------------------------------------
%                  -Nécessaires pour écrire des mathématiques-
%-------------------------------------------------------------------------------
\usepackage{amsfonts}
\usepackage{amssymb}
\usepackage{amsmath}
\usepackage{amsthm}
\usepackage{tikz}
\usepackage{xlop}
\usepackage[output-decimal-marker={,}]{siunitx}
%-------------------------------------------------------------------------------

%-------------------------------------------------------------------------------
%                  -Nécessaires pour écrire des formules chimiquess-
%-------------------------------------------------------------------------------

\usepackage[version=4]{mhchem}

%-------------------------------------------------------------------------------
% Pour pouvoir exploiter les fichiers directement dans beamer
\newcommand{\pause}{\ }
%-------------------------------------------------------------------------------
%                    - Mise en forme avancée
%-------------------------------------------------------------------------------

\usepackage{ifthen}
\usepackage{ifmtarg}


\newcommand{\ifTrue}[2]{\ifthenelse{\equal{#1}{true}}{#2}{$\qquad \qquad$}}

%\newcommand{\kword}[1]{\textcolor{red}{\underline{#1}}}
%-------------------------------------------------------------------------------

%-------------------------------------------------------------------------------
%                     -Mise en forme d'exercices-
%-------------------------------------------------------------------------------
%\newtheoremstyle{exostyle}
%{\topsep}% espace avant
%{\topsep}% espace apres
%{}% Police utilisee par le style de thm
%{}% Indentation (vide = aucune, \parindent = indentation paragraphe)
%{\bfseries}% Police du titre de thm
%{.}% Signe de ponctuation apres le titre du thm
%{ }% Espace apres le titre du thm (\newline = linebreak)
%{\thmname{#1}\thmnumber{ #2}\thmnote{. \normalfont{\textit{#3}}}}% composants du titre du thm : \thmname = nom du thm, \thmnumber = numéro du thm, \thmnote = sous-titre du thm

%\theoremstyle{exostyle}
%\newtheorem{exercice}{Exercice}
%
%\newenvironment{questions}{
%\begin{enumerate}[\hspace{12pt}\bfseries\itshape a.]}{\end{enumerate}
%} %mettre un 1 à la place du a si on veut des numéros au lieu de lettres pour les questions 
%-------------------------------------------------------------------------------

%-------------------------------------------------------------------------------
%                    - Mise en forme de tableaux -
%-------------------------------------------------------------------------------

\renewcommand{\arraystretch}{1.7}

\setlength{\tabcolsep}{1.2cm}

%-------------------------------------------------------------------------------



%-------------------------------------------------------------------------------
%                    - Racourcis d'écriture -
%-------------------------------------------------------------------------------
%Droites
\newcommand{\dte}[1]{$(#1)$}
\newcommand{\fig}[1]{figure $#1$}
\newcommand{\sym}{symétrique}
\newcommand{\syms}{symétriques}
\newcommand{\asym}{axe de symétrie}
\newcommand{\asyms}{axes de symétrie}
\newcommand{\seg}[1]{$[#1]$}
\newcommand{\monAngle}[1]{$\widehat{#1}$}
\newcommand{\bissec}{bissectrice}
\newcommand{\mediat}{médiatrice}
\newcommand{\ddte}[1]{$[#1)$}


% Angles orientés (couples de vecteurs)
\newcommand{\aopp}[2]{(\vec{#1}, \vec{#2})} %Les deuc vecteurs sont positifs
\newcommand{\aopn}[2]{(\vec{#1}, -\vec{#2})} %Le second vecteur est négatif
\newcommand{\aonp}[2]{(-\vec{#1}, \vec{#2})} %Le premier vecteur est négatif
\newcommand{\aonn}[2]{(-\vec{#1}, -\vec{#2})} %Les deux vecteurs sont négatifs

%Ensembles mathématiques
\newcommand{\naturels}{\mathbb{N}} %Nombres naturels
\newcommand{\relatifs}{\mathbb{Z}} %Nombres relatifs
\newcommand{\rationnels}{\mathbb{Q}} %Nombres rationnels
\newcommand{\reels}{\mathbb{R}} %Nombres réels
\newcommand{\complexes}{\mathbb{C}} %Nombres complexes


%Intégration des parenthèses aux cosinus
\newcommand{\cosP}[1]{\cos\left(#1\right)}
\newcommand{\sinP}[1]{\sin\left(#1\right)}


%Probas stats
\newcommand{\stat}{statistique}
\newcommand{\stats}{statistiques}


\newcommand{\homo}{homothétie}
\newcommand{\homos}{homothéties}


\newcommand{\mycoord}[3]{(\textcolor{red}{\num{#1}} ; \textcolor{Green}{\num{#2}} ; \textcolor{blue}{\num{#3}})}
%-------------------------------------------------------------------------------

%-------------------------------------------------------------------------------
%                    - Mise en page -
%-------------------------------------------------------------------------------

\newcommand{\twoCol}[1]{\begin{multicols}{2}#1\end{multicols}}


\setenumerate[1]{font=\bfseries,label=\textit{\alph*})}
\setenumerate[2]{font=\bfseries,label=\arabic*)}


%-------------------------------------------------------------------------------
%                    - Elements cours -
%-------------------------------------------------------------------------------

%Correction d'exercice
\newcommand{\exoSec}[2]{\subsection*{Exercice #1 page #2}}
%-------------------------------------------------------------------------------
%                    - raccourcis d'écriture -
%-------------------------------------------------------------------------------

%Mise en évidence de termes clés
\newcommand{\mykw}[1]{\textcolor{red}{\underline{\textbf{#1}}}}

%Exercices
\newcommand{\exo}[2]{exercice #1 page #2}
\newcommand{\Exo}[2]{Exercice #1 page #2}

\renewcommand{\pause}{\ }




\date{}
\title{}

\graphicspath{{./img/}}


\begin{document}
	
	
\chap[num=1, color=blue]{\small Quelles sont les deux sortes de sources de lumière ?}{Olivier FINOT, \today }	

\section{Sources de lumière}

\section{Comment caractériser un mouvement ?}

\begin{questions}
	\question Le mouvement du tunnelier est \underline{rectiligne} et \underline{uniforme}.
	
	\question Lors du fonctionnement du tunnelier, la roue coupante a une trajectoire \underline{circulaire}.
	
	\question Lors d'un cycle de fonctionnement du tunnelier la roue :
	\begin{enumerate}
		\item commence par démarrer, donc sa vitesse augmente ;
		\item puis elle se stabilise à vitesse constante;
		\item enfin elle ralenti pour s'arrêter.
	\end{enumerate} 

	\question La roue coupante du tunnelier a donc un mouvement :
	\begin{enumerate}
		\item d'abord circulaire accéléré;
		\item ensuite circulaire uniforme;
		\item enfin circulaire ralenti;
	\end{enumerate} 
\end{questions}

\begin{mybilan}
	\begin{itemize}
		\item Dans un environnement sec, un courant électrique est dangereux à partir d'une tension de 50 V.\pause
		
		\item En France, une prise électrique fournit une tension de \kw{230 V}. Il ne faut pas toucher toucher les bornes d'une prise car cela pourrait provoquer une \kw{électrisation} voire une \kw{électrocution}.\pause
				 
	\end{itemize}

\end{mybilan}

\begin{mydefs}
	\begin{itemize}
		\item \kw{\'Electrisation} : passage du courant électrique à travers le corps humain. 
		
		\item \kw{\'Electrocution} : \'electrisation qui entraine la mort.
	\end{itemize}
\end{mydefs}




\begin{myexos}
	\begin{itemize}
		\item \exo{1}{70}
		\item \exo{13}{72}
		\item \exo{17}{73}
	\end{itemize}
\end{myexos}

\section{Voir un objet}

\begin{myact}{}

		Activité 16 page 51 cahier d'activités

\end{myact}

\begin{mybilan}
	\begin{itemize}
		\item L'énergie peut être \kw{transférée} d'un objet vers un autre objet.
		
		\item Une forme d'énergie peut être \kw{convertie} en une autre forme d'énergie.
		
		
		\begin{center}
			\includegraphics[scale=0.8]{conversion}
		\end{center}
		
		\item On représente un ensemble de transferts et conversions d'énergie par une \kw{chaine énergétique}.

		\begin{center}
			\includegraphics[scale=0.5]{chaine}
		\end{center}
	\end{itemize}
\end{mybilan}



\begin{myexos}
	\begin{multicols}{2}
	
		\begin{itemize}
			\item \exo{2}{70}
			\item \exo{6}{71}
			\item \exo{7}{71}
			\item \exo{10}{72}
			\item \exo{11}{72}

		\end{itemize}
	
	\end{multicols}
\end{myexos}


\section{Voir la lumière}

\begin{myact}{3 page 186}
	\begin{enumerate}
		\item Durant l'enregistrement, la tension est variable.\pause
		\item La valeur de la tension maximale est \num{4} $V$.
		\item La valeur de la tension minimale est \num{-2.3} $V$.
		\item La valeur de la période est de environ \num{130} $s$ ($170 - 40$).
		\item LA tension a une valeur nulle à $t_1$ $\approx$ 40 $s$ et $t_2$ $\approx$ 170 $s$.
		
	\end{enumerate}
\end{myact}

\begin{mybilan}
	\begin{itemize}
		\item L'unité de masse du système international est \kw{le kilogramme} ($kg$). En chimie, on utilise souvent un sous-multiple, le \kw{gramme} ($g$).\pause
		\item Si l'on pose un récipient vide sur le plateau d'une \kw{balance}, le bouton TARE permet de remettre l'affichage à 0 ; ainsi on ne tient pas compte de la masse de ce récipient.\pause
		\item Mesure d'une masse : voir fiche méthode 3 page 104 (partie 2)
	\end{itemize}
\end{mybilan}

\begin{myexos}
	\begin{itemize}
		\item \exo{3}{70}
		\item \exo{8}{71}
	\end{itemize}
\end{myexos}


\section{Les dangers du rayon laser}



\begin{myact}{4 page 187}
	\begin{enumerate}
		\item La tension observée est variable et périodique.\pause
		\item La durée entre deux valeurs successives de la tension maximale est de \num{2.5} $ms$, c'est la période.\pause
		\item $U_{max}$ = \num{7.5} $V$.\pause
		\item  $U_{min}$ = \num{-7.5} $V$.\pause
		\item 4 motifs sont représentés sur le document B.\pause
		\item Les parties où la tension est positive sont comparables à celles où elle est négative : elles se compensent.		
	\end{enumerate}
\end{myact}


\begin{mybilan}
	\begin{itemize}
		\item La masse d'un corps est \kw{proportionnelle} à son volume; \pause
		\item Le coefficient de proportionnalité est la \kw{masse volumique} (notée $\rho$);\pause
		\item \kw{Un litre d'eau} a une masse de \kw{1 kilogramme};\pause
		\item Une substance est \kw{plus dense} qu'une autre si, pour un même volume, sa masse est supérieure.		
	\end{itemize}
\end{mybilan}

\begin{myexos}
	\twoCol{\begin{itemize}
		\item \exo{4}{70}
		\item \exo{9}{71}
		\item \exo{14}{72}
	\end{itemize}}
\end{myexos}
\appendix

\newpage

\section*{Correction des exercices}

\subsection*{\Exo{1}{70}}

\begin{enumerate}[label=\alph*)]
	\item Une lampe allumée est \textbf{une source primaire}.
	\item Les planètes sont des \textbf{objets diffusants}.
	\item Un objet qui renvoie la lumière qu'il reçoit dans toutes les directions est \textbf{un objet diffusant}.
	\item Un objet qui produit la lumière qu'il émet est \textbf{une source primaire}.
\end{enumerate}

\subsection*{\Exo{2}{70}}
	La {source de lumière} (le Soleil) éclaire le livre qui est un \textbf{objet diffusant} et qui renvoie la lumière dans les \textbf{yeux} de la fille.

\subsection*{\Exo{3}{70}}
	La lumière émise par la lampe ne doit pas être visible là où il n'y a pas de poussière.
	
\subsection*{\Exo{4}{70}}

\begin{enumerate}[label=\alph*)]
	\item Un laser est est un faisceau de lumière de \textbf{grande} énergie.
	\item Lorsque le faisceau laser touche un obstacle, on observe un \textbf{minuscule} point d'impact.
	\item Le faisceau laser est  \textbf{dangereux} pour les yeux d'une personne.
	\item On \textbf{ne doit jamais} diriger la lumière d'un pointeur laser vers les yeux d'une personne. 
\end{enumerate}

\subsection*{\Exo{6}{71}}

	Tom doit diriger sa lampe vers l'écran blanc pour que la lumière soit diffusée et renvoyée vers \'Elodie. Ainsi elle sera éclairée.
	
\subsection*{\Exo{7}{71}}
	\begin{enumerate}[label=\alph*)]
		\item La deuxième image (en haut à droite).
		\item Un objet est visible si il est éclairé et renvoie la lumière qu'il reçoit vers les yeux de l'observateur.
	\end{enumerate}

\subsection*{\Exo{8}{71}}
	\begin{enumerate}[label=\alph*)]
		\item On ne voit pas la lumière du projecteur entre celui-ci et le bécher.
		\item Quand on fait bouillir de l'eau il se forme de la vapeur au dessus.
		\item On peut voir le trajet de la lumière car elle est diffusée par les gouttelettes d'eau qui forment la vapeur.
	\end{enumerate}

\subsection*{\Exo{9}{72}}
	\begin{enumerate}[label=\alph*)]
		\item Le panneau indique l'utilisation d'un laser qui peuvent être dangereux.
		\item Les lunettes mises à disposition sont utilisées pour protéger les yeux des observateurs.
	\end{enumerate}

\subsection*{\Exo{10}{72}}
	\begin{enumerate}
		\item La lampe éclaire Tom qui renvoie la lumière vers les yeux d'\'Eric qui peut donc le voir.
		\item Il y a un obstacle opaque, le foulard, entre les yeux de Tom et \'Eric, donc Tom ne peut pas le voir.
		\item Si la lampe est éteinte, il n'y plus de source de lumière principale donc aucun d'eux ne peut diffuser de lumière vers les yeux de l'autre. Ils ne peuvent pas se voir.
	\end{enumerate}

\subsection*{\Exo{11}{72}}
	\begin{enumerate}
		\item On peut voir la figurine éclairée donc l'écran est transparent.
		\item La vue de la figurine est bloquée par l'écran, il est opaque.
	\end{enumerate}

\subsection*{\Exo{13}{72}}
\begin{enumerate}[label=\alph*)]
	\item Dans une salle de cinéma, le projecteur se trouve derrière les spectateurs.
	\item L'écran de cinéma est un objet diffusant.
	\item La lumière qui éclaire le visage des spectateurs est produite par le projecteur et renvoyée par l'écran.
	\item Un écran de télévision produit lui-même la lumière qu'il renvoie, c'est une source primaire.
\end{enumerate}

\subsection*{\Exo{14}{72}}

$\num{300000} \times \num{2.56} = \num{768000}$\\
La Lune se trouve à \num{768000} km de la Terre.

\subsection*{\Exo{17}{73}}
\begin{enumerate}[label=\alph*)]
	\item Le gilet porté par le cycliste est jaune fluorescent. Cette couleur lui permet d'être bien repéré la nuit.
	\item Les bandes fluorescentes sont des objets diffusants.
	\item Un cycliste doit porter cette sorte de gilet lorsqu'il se déplace hors agglomération de nuit ou lorsque la visibilité est réduite.
\end{enumerate}

\end{document}]