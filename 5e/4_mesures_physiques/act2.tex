\begin{myact}{2 page 153}
	\begin{enumerate}
		\item Sur la fiole jaugée, le volume est exprimé en millilitres ($mL$). \pause
		\item La fiole jaugée contient \num{1000} $mL$ de liquide, soit 1 litre ($L$). \pause
		\item Le volume du récipient cubique est de 1 décimètre cube ($dm^3$). \pause
		\item On nous indique que le contenu de la fiole jaugée a été transféré sans perdre de liquide, donc le volume n'a pas changé.\pause
		\item Le récipient cubique contient 1 $dm^3$ de liquide, soit \num{1000} centimètres cubes ($cm^3$).\pause
		\item On a 1 $L =$ 1 $dm^3$ et 1 $mL = $ 1 $cm^3$.
	\end{enumerate}
\end{myact}