\begin{myact}{1 page 152}
	\begin{enumerate}
		\item Les éprouvettes sont graduées en millilitre ($mL$).\pause
		\item Dans cette éprouvette, un intervalle entre deux graduations correspond à 1 $mL$.\pause
		\item Pour lire le volume l'élève place son regard au niveau du bas du ménisque.\pause
		\item Sur l'éprouvette on lit la valeur 74.\pause
		\item La deuxième et la troisième éprouvette peuvent être utilisées pour mesurer le volume de liquide du document B.\pause
		\item La lecture du volume avec une éprouvette graduée se fait au bas du ménisque, c'est à cet endroit qu'il faut placer le regard.\pause
		\item L'éprouvette contient 74 $mL$ de liquide.
	\end{enumerate}
\end{myact}