\documentclass[12pt,a4paper]{article}

%\usepackage[left=1.5cm,right=1.5cm,top=1cm,bottom=2cm]{geometry}
\usepackage[in, plain]{fullpage}
\usepackage{array}
%\usepackage{../../pas-math}
\usepackage{../../moncours}



%-------------------------------------------------------------------------------
%          -Packages nécessaires pour écrire en Français et en UTF8-
%-------------------------------------------------------------------------------
\usepackage[utf8]{inputenc}
\usepackage[frenchb]{babel}
%\usepackage{numprint}
\usepackage[T1]{fontenc}
%\usepackage{lmodern}
\usepackage{textcomp}
\usepackage[french, boxed]{algorithm2e}
\usepackage{hyperref}


%-------------------------------------------------------------------------------

%-------------------------------------------------------------------------------
%                          -Outils de mise en forme-
%-------------------------------------------------------------------------------
\usepackage{hyperref}
\hypersetup{pdfstartview=XYZ}
%\usepackage{enumerate}
\usepackage{graphicx}
\usepackage{multicol}
\usepackage{tabularx}
\usepackage{multirow}
\usepackage{color}
\usepackage{eurosym}


\usepackage{anysize} %%pour pouvoir mettre les marges qu'on veut
%\marginsize{2.5cm}{2.5cm}{2.5cm}{2.5cm}

\usepackage{indentfirst} %%pour que les premier paragraphes soient aussi indentés
\usepackage{verbatim}
\usepackage{enumitem}
\usepackage{booktabs}
\usepackage[usenames,dvipsnames,svgnames,table]{xcolor}

\usepackage{variations}

%-------------------------------------------------------------------------------


%-------------------------------------------------------------------------------
%                  -Nécessaires pour écrire des mathématiques-
%-------------------------------------------------------------------------------
\usepackage{amsfonts}
\usepackage{amssymb}
\usepackage{amsmath}
\usepackage{amsthm}
\usepackage{tikz}
\usepackage{xlop}
\usepackage[output-decimal-marker={,}]{siunitx}
%-------------------------------------------------------------------------------

%-------------------------------------------------------------------------------
%                  -Nécessaires pour écrire des formules chimiquess-
%-------------------------------------------------------------------------------

\usepackage[version=4]{mhchem}

%-------------------------------------------------------------------------------
% Pour pouvoir exploiter les fichiers directement dans beamer
\newcommand{\pause}{\ }
%-------------------------------------------------------------------------------
%                    - Mise en forme avancée
%-------------------------------------------------------------------------------

\usepackage{ifthen}
\usepackage{ifmtarg}


\newcommand{\ifTrue}[2]{\ifthenelse{\equal{#1}{true}}{#2}{$\qquad \qquad$}}

%\newcommand{\kword}[1]{\textcolor{red}{\underline{#1}}}
%-------------------------------------------------------------------------------

%-------------------------------------------------------------------------------
%                     -Mise en forme d'exercices-
%-------------------------------------------------------------------------------
%\newtheoremstyle{exostyle}
%{\topsep}% espace avant
%{\topsep}% espace apres
%{}% Police utilisee par le style de thm
%{}% Indentation (vide = aucune, \parindent = indentation paragraphe)
%{\bfseries}% Police du titre de thm
%{.}% Signe de ponctuation apres le titre du thm
%{ }% Espace apres le titre du thm (\newline = linebreak)
%{\thmname{#1}\thmnumber{ #2}\thmnote{. \normalfont{\textit{#3}}}}% composants du titre du thm : \thmname = nom du thm, \thmnumber = numéro du thm, \thmnote = sous-titre du thm

%\theoremstyle{exostyle}
%\newtheorem{exercice}{Exercice}
%
%\newenvironment{questions}{
%\begin{enumerate}[\hspace{12pt}\bfseries\itshape a.]}{\end{enumerate}
%} %mettre un 1 à la place du a si on veut des numéros au lieu de lettres pour les questions 
%-------------------------------------------------------------------------------

%-------------------------------------------------------------------------------
%                    - Mise en forme de tableaux -
%-------------------------------------------------------------------------------

\renewcommand{\arraystretch}{1.7}

\setlength{\tabcolsep}{1.2cm}

%-------------------------------------------------------------------------------



%-------------------------------------------------------------------------------
%                    - Racourcis d'écriture -
%-------------------------------------------------------------------------------
%Droites
\newcommand{\dte}[1]{$(#1)$}
\newcommand{\fig}[1]{figure $#1$}
\newcommand{\sym}{symétrique}
\newcommand{\syms}{symétriques}
\newcommand{\asym}{axe de symétrie}
\newcommand{\asyms}{axes de symétrie}
\newcommand{\seg}[1]{$[#1]$}
\newcommand{\monAngle}[1]{$\widehat{#1}$}
\newcommand{\bissec}{bissectrice}
\newcommand{\mediat}{médiatrice}
\newcommand{\ddte}[1]{$[#1)$}


% Angles orientés (couples de vecteurs)
\newcommand{\aopp}[2]{(\vec{#1}, \vec{#2})} %Les deuc vecteurs sont positifs
\newcommand{\aopn}[2]{(\vec{#1}, -\vec{#2})} %Le second vecteur est négatif
\newcommand{\aonp}[2]{(-\vec{#1}, \vec{#2})} %Le premier vecteur est négatif
\newcommand{\aonn}[2]{(-\vec{#1}, -\vec{#2})} %Les deux vecteurs sont négatifs

%Ensembles mathématiques
\newcommand{\naturels}{\mathbb{N}} %Nombres naturels
\newcommand{\relatifs}{\mathbb{Z}} %Nombres relatifs
\newcommand{\rationnels}{\mathbb{Q}} %Nombres rationnels
\newcommand{\reels}{\mathbb{R}} %Nombres réels
\newcommand{\complexes}{\mathbb{C}} %Nombres complexes


%Intégration des parenthèses aux cosinus
\newcommand{\cosP}[1]{\cos\left(#1\right)}
\newcommand{\sinP}[1]{\sin\left(#1\right)}


%Probas stats
\newcommand{\stat}{statistique}
\newcommand{\stats}{statistiques}


\newcommand{\homo}{homothétie}
\newcommand{\homos}{homothéties}


\newcommand{\mycoord}[3]{(\textcolor{red}{\num{#1}} ; \textcolor{Green}{\num{#2}} ; \textcolor{blue}{\num{#3}})}
%-------------------------------------------------------------------------------

%-------------------------------------------------------------------------------
%                    - Mise en page -
%-------------------------------------------------------------------------------

\newcommand{\twoCol}[1]{\begin{multicols}{2}#1\end{multicols}}


\setenumerate[1]{font=\bfseries,label=\textit{\alph*})}
\setenumerate[2]{font=\bfseries,label=\arabic*)}


%-------------------------------------------------------------------------------
%                    - Elements cours -
%-------------------------------------------------------------------------------

%Correction d'exercice
\newcommand{\exoSec}[2]{\subsection*{Exercice #1 page #2}}
%-------------------------------------------------------------------------------
%                    - raccourcis d'écriture -
%-------------------------------------------------------------------------------

%Mise en évidence de termes clés
\newcommand{\mykw}[1]{\textcolor{red}{\underline{\textbf{#1}}}}

%Exercices
\newcommand{\exo}[2]{exercice #1 page #2}
\newcommand{\Exo}[2]{Exercice #1 page #2}

\renewcommand{\pause}{\ }




\date{}
\title{}


\begin{document}
	
	
\chap[num=4, color=blue]{Mesures physiques}{Olivier FINOT, \today }	

\section{La mesure des volumes}

\section{Comment caractériser un mouvement ?}

\begin{questions}
	\question Le mouvement du tunnelier est \underline{rectiligne} et \underline{uniforme}.
	
	\question Lors du fonctionnement du tunnelier, la roue coupante a une trajectoire \underline{circulaire}.
	
	\question Lors d'un cycle de fonctionnement du tunnelier la roue :
	\begin{enumerate}
		\item commence par démarrer, donc sa vitesse augmente ;
		\item puis elle se stabilise à vitesse constante;
		\item enfin elle ralenti pour s'arrêter.
	\end{enumerate} 

	\question La roue coupante du tunnelier a donc un mouvement :
	\begin{enumerate}
		\item d'abord circulaire accéléré;
		\item ensuite circulaire uniforme;
		\item enfin circulaire ralenti;
	\end{enumerate} 
\end{questions}

\begin{mybilan}
	\begin{itemize}
		\item Dans un environnement sec, un courant électrique est dangereux à partir d'une tension de 50 V.\pause
		
		\item En France, une prise électrique fournit une tension de \kw{230 V}. Il ne faut pas toucher toucher les bornes d'une prise car cela pourrait provoquer une \kw{électrisation} voire une \kw{électrocution}.\pause
				 
	\end{itemize}

\end{mybilan}

\begin{mydefs}
	\begin{itemize}
		\item \kw{\'Electrisation} : passage du courant électrique à travers le corps humain. 
		
		\item \kw{\'Electrocution} : \'electrisation qui entraine la mort.
	\end{itemize}
\end{mydefs}




\begin{myexos}
	\begin{itemize}
		\item \exo{1}{160}
		\item \exo{5}{161}
		\item \exo{7}{161}
	\end{itemize}
\end{myexos}

\section{Volume et unités}

\begin{myact}{}

		Activité 16 page 51 cahier d'activités

\end{myact}

\begin{mybilan}
	\begin{itemize}
		\item L'énergie peut être \kw{transférée} d'un objet vers un autre objet.
		
		\item Une forme d'énergie peut être \kw{convertie} en une autre forme d'énergie.
		
		
		\begin{center}
			\includegraphics[scale=0.8]{conversion}
		\end{center}
		
		\item On représente un ensemble de transferts et conversions d'énergie par une \kw{chaine énergétique}.

		\begin{center}
			\includegraphics[scale=0.5]{chaine}
		\end{center}
	\end{itemize}
\end{mybilan}



\begin{myexos}
	\begin{multicols}{2}
	
		\begin{itemize}
			\item \exo{2}{160}
			\item \exo{6}{161}
			\item \exo{13}{162}
		\end{itemize}
	
	\end{multicols}
\end{myexos}


\section{La mesure des masses}

\begin{myact}{3 page 186}
	\begin{enumerate}
		\item Durant l'enregistrement, la tension est variable.\pause
		\item La valeur de la tension maximale est \num{4} $V$.
		\item La valeur de la tension minimale est \num{-2.3} $V$.
		\item La valeur de la période est de environ \num{130} $s$ ($170 - 40$).
		\item LA tension a une valeur nulle à $t_1$ $\approx$ 40 $s$ et $t_2$ $\approx$ 170 $s$.
		
	\end{enumerate}
\end{myact}

\begin{mybilan}
	\begin{itemize}
		\item L'unité de masse du système international est \kw{le kilogramme} ($kg$). En chimie, on utilise souvent un sous-multiple, le \kw{gramme} ($g$).\pause
		\item Si l'on pose un récipient vide sur le plateau d'une \kw{balance}, le bouton TARE permet de remettre l'affichage à 0 ; ainsi on ne tient pas compte de la masse de ce récipient.\pause
		\item Mesure d'une masse : voir fiche méthode 3 page 104 (partie 2)
	\end{itemize}
\end{mybilan}

\begin{myexos}
	\twoCol{
		\begin{itemize}
			\item \exo{3}{160}
			\item \exo{9}{161}
			\item \exo{11}{161}
			\item \exo{16}{163}
		\end{itemize}
	}
\end{myexos}


\section{La mesure des températures}



\begin{myact}{4 page 187}
	\begin{enumerate}
		\item La tension observée est variable et périodique.\pause
		\item La durée entre deux valeurs successives de la tension maximale est de \num{2.5} $ms$, c'est la période.\pause
		\item $U_{max}$ = \num{7.5} $V$.\pause
		\item  $U_{min}$ = \num{-7.5} $V$.\pause
		\item 4 motifs sont représentés sur le document B.\pause
		\item Les parties où la tension est positive sont comparables à celles où elle est négative : elles se compensent.		
	\end{enumerate}
\end{myact}


\begin{mybilan}
	\begin{itemize}
		\item La masse d'un corps est \kw{proportionnelle} à son volume; \pause
		\item Le coefficient de proportionnalité est la \kw{masse volumique} (notée $\rho$);\pause
		\item \kw{Un litre d'eau} a une masse de \kw{1 kilogramme};\pause
		\item Une substance est \kw{plus dense} qu'une autre si, pour un même volume, sa masse est supérieure.		
	\end{itemize}
\end{mybilan}

\begin{myexos}
	\twoCol{\begin{itemize}
		\item \exo{4}{160}
		\item \exo{10}{161}
	\end{itemize}}
\end{myexos}
\appendix

\newpage

\section*{Correction des exercices}

\subsection*{\Exo{1}{160}}

Seul le schéma 1 présente une lecture du volume à la base du ménisque avec le regard à ce niveau. C'est le seul qui correspond à une lecture correcte du volume.

\subsection*{\Exo{2}{160}}

\begin{enumerate}[label=\alph*)]
	\item L'unité de volume du système international est le \textbf{mètre cube} ($m^3$).
	\item L'unité usuelle de volume est le \textbf{litre} ($L$).
	\item Un décimètre cube et un \textbf{litre} représentent le même volume.
	\item Un millilitre et un \textbf{centimètre cube} représentent le même volume.
	\item Un fiole jaugée de 100 $mL$ contient \textbf{100 $cm^3$} de liquide jusqu'au trait de jauge.
\end{enumerate}

\subsection*{\exo{3}{160}}

\begin{enumerate}[label=\alph*)]
	\item La balance indique \num{46.7} $g$ et \num{78.5} $g$ en 2.
	\item L'erlenmeyer 2 contient \num{31.8} $g$ d'eau (\num{78.5} $-$ \num{46.7} $=$ \num{31.8}).
	\item La masse aurait pu être trouvée plus rapidement en utilisant le bouton TARE de la balance.
\end{enumerate}

\subsection*{\exo{4}{160}}

\begin{enumerate}[label=\alph*)]
	\item L'appareil de mesure utilisé est un thermomètre électronique.
	\item L'unité de mesure affichée sur l'écran est le degré Celsuis (°$C$).
	\item La valeur de la température est de \num{18}°$C$.
\end{enumerate}

\subsection*{\exo{5}{161}}

\begin{enumerate}[label=\alph*)]
	\item Avec l'éprouvette 1 on peut mesurer un volume maximal de 50 $mL$, 100 $mL$ avec l'éprouvette 2 et 250 $mL$ avec l'éprouvette 3.
	\item L'intervalle entre deux graduations correspond à 1 $mL$ pour les éprouvettes 1 et 2, pour la 3 il correspond à 2 $mL$.
	\item La première éprouvette contient 40 $mL$ de liquide, la deuxième 90$mL$ et la troisième 222 $mL$.
\end{enumerate}

\subsection*{\exo{6}{161}}

\twoCol{
\begin{enumerate}[label=\alph*)]
	\item 1 $dm^3 = $ \textbf{1} $L$.
	\item 1 $mL = $ \textbf{1}$cm^3$.
	\item 18 $cm^3 = $ \textbf{18}$mL$.
	\item \num{4.5} $L = $ \textbf{\num{4.5}} $dm^3$.
	\item 350 $ml = $ \textbf{\num{0.35}} $L$.
	\item \num{0.15} $dm^3 = $ \textbf{150} $cm^3$.
\end{enumerate}
}

\newpage
\subsection*{\exo{7}{161}}

	\begin{enumerate}[label=\alph*)]
		\item L'intervalle entre deux graduations correspond à 10 $mL$.
		\item Dans ce biberon il ya un peu moins de 180 $mL$.
		\item Pour lire le bon volume il faut placer son regard au niveau de la base du ménisque.
		\item La graduation est trop imprécise pour mesurer 152 $mL$ d'eau, il vaudrait mieux utiliser une éprouvette graduée.
	\end{enumerate}


\subsection*{\exo{8}{161}}


\begin{enumerate}[label=\alph*)]
	\item Il doit arrêter de mettre du sucre lorsque la balnce indiquera 255 $g$ ($230 + 125 = 355$).
	\item Pour mesurer directement la masse de sucre il aurait pu utiliser la fonction TARE de la balance. Dans ce cas il aurait lu la valeur 0 sur l'écran de la balance avant de commencer à verser le sucre.
\end{enumerate}

\subsection*{\exo{9}{161}}

Pour vérifier l'affirmation on utilise une fiole jaugée de 1 $L$ et une balance.
\begin{enumerate}
	\item Placer la fiole vide sur la balance.
	\item Faire la tare.
	\item Remplir la fiole d'éthanol jusqu'à la jauge.
	\item Poser à nouveau la fiole remplie sur la balance.
	\item Lire la masse d'un litre d'éthanol.
\end{enumerate}


\subsection*{\exo{10}{161}}


\begin{enumerate}[label=\alph*)]
	\item L'appareil de mesure est un thermomètre électronique. L'unité usuelle de la température est le degré Celsius (°$C$).
	\item La première photographie correspond au début de la mesure car la température est la plus basse et qu'il est indiqué qu'on chauffe l'eau après la première mesure.
	\item La température de l'eau chaude est de \num{40.6} °$C$.
\end{enumerate}

\subsection*{\exo{11}{161}}


\begin{enumerate}[label=\alph*)]
	\item La masse d'un litre d'eau est de 1 $kg$.
	\item La masse d'eau contenue dans une bouteille de \num{1.5} $L$ est de \num{1.5} $kg$.
	\item Si marine avait choisi une bouteille de \num{0.5} $L$, la masse d'eau à transporter aurait été de \num{0.5} $kg$, soit 500 $g$.
\end{enumerate}
\end{document}