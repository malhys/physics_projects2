\section{Le train à vapeur}\label{ex:train}

Les premiers train fonctionnaient grâce à des moteurs à vapeur. 
L'énergie stockée dans l'air et 
dans le charbon était transférée au moteur à vapeur sous forme d'énergie thermique.
Le moteur convertissait ensuite l'énergie reçue en énergie de mouvement qu'il transférait à l'ensemble du train. On considère que le charbon et l'air font partie d'un seul et même  réservoir d'énergie.

\begin{questions}
	\question Quels étaient les deux réservoirs d'énergie et le convertisseur d'énergie ?
	
	\question Le moteur convertissait l'énergie qu'il recevait en une autre forme d'énergie. Laquelle ?
	
	\question Réaliser la chaine énergétique du fonctionnement de ce train. 
\end{questions}