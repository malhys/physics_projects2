\begin{mybilan}
	\begin{itemize}
		\item On étudie le mouvement d'un objet par rapport à un lieu ou un objet, c'est le \kw{référentiel}. C'est la \kw{relativité du mouvement}.
				
		\item Dans un référentiel donné, la \kw{trajectoire} d'un objet en mouvement est formée par l'ensemble des positions prises par l'objet au cours du mouvement.		
		\begin{itemize}
			\item Si la trajectoire décrit une \kw{ligne droite} le mouvement est \kw{rectiligne}.
			\item Si elle décrit \kw{un cercle ou un arc de cercle}, le mouvement est \kw{circulaire}.
			\item Sinon il est \kw{quelconque}.
		\end{itemize}	
		
	\end{itemize}
\end{mybilan}