\begin{mybilan}
	\begin{itemize}
		\item Pour calculer la vitesse $v$ (en m/s ou km/h) d'un objet en mouvement on a besoin de la \kw{distance parcourue} $d$ (en m ou en km) et de la \kw{durée du parcours} $t$ (en s ou en h).
		
		On a :
		\begin{equation*}
		v = \frac{d}{t}
		\end{equation*}
		
		\item Si la valeur de la vitesse est constante, un mouvement est est \kw{uniforme}; si elle augmente il est \kw{accéléré} ; si elle diminue, il est \kw{ralenti}. 	
	\end{itemize}
	
	

	
	
\end{mybilan}

