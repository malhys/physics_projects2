\documentclass[a4paper,11pt]{exam}
\printanswers % pour imprimer les réponses (corrigé)
%\noprintanswers % Pour ne pas imprimer les réponses (énoncé)
\addpoints % Pour compter les points
% \noaddpoints % pour ne pas compter les points
%\qformat{\textbf{\thequestion ) } }
%\qformat{\textbf{\thequestion )}} % Pour définir le style des questions (facultatif)
\usepackage{color} % définit une nouvelle couleur
\shadedsolutions % définit le style des réponses
% \framedsolutions % définit le style des réponses
\definecolor{SolutionColor}{rgb}{0.8,0.9,1} % bleu ciel
\renewcommand{\solutiontitle}{\noindent\textbf{Solution:}\par\noindent} % Définit le titre des solutions
\usepackage{gensymb}



\makeatletter

\def\maketitle{{\centering%
	\par{\huge\textbf{\@title}}%
	\par{\@date}%
	\par}}


\renewcommand{\thesubsection}{\Alph{subsection}.}   

\makeatother

\lhead{NOM Pr\'enom :}
\rhead{\textbf{Les r\'eponses doivent \^etre justifi\'ees et r\'edig\'ees}}
\cfoot{\thepage / \pageref{LastPage}}


%\usepackage{../../pas-math}
%\usepackage{../../moncours}


%\usepackage{pas-cours}
%-------------------------------------------------------------------------------
%          -Packages nécessaires pour écrire en Français et en UTF8-
%-------------------------------------------------------------------------------
\usepackage[utf8]{inputenc}
\usepackage[frenchb]{babel}
%\usepackage{numprint}
\usepackage[T1]{fontenc}
%\usepackage{lmodern}
\usepackage{textcomp}
\usepackage[french, boxed]{algorithm2e}
\usepackage{hyperref}


%-------------------------------------------------------------------------------

%-------------------------------------------------------------------------------
%                          -Outils de mise en forme-
%-------------------------------------------------------------------------------
\usepackage{hyperref}
\hypersetup{pdfstartview=XYZ}
%\usepackage{enumerate}
\usepackage{graphicx}
\usepackage{multicol}
\usepackage{tabularx}
\usepackage{multirow}
\usepackage{color}
\usepackage{eurosym}


\usepackage{anysize} %%pour pouvoir mettre les marges qu'on veut
%\marginsize{2.5cm}{2.5cm}{2.5cm}{2.5cm}

\usepackage{indentfirst} %%pour que les premier paragraphes soient aussi indentés
\usepackage{verbatim}
\usepackage{enumitem}
\usepackage{booktabs}
\usepackage[usenames,dvipsnames,svgnames,table]{xcolor}

\usepackage{variations}

%-------------------------------------------------------------------------------


%-------------------------------------------------------------------------------
%                  -Nécessaires pour écrire des mathématiques-
%-------------------------------------------------------------------------------
\usepackage{amsfonts}
\usepackage{amssymb}
\usepackage{amsmath}
\usepackage{amsthm}
\usepackage{tikz}
\usepackage{xlop}
\usepackage[output-decimal-marker={,}]{siunitx}
%-------------------------------------------------------------------------------

%-------------------------------------------------------------------------------
%                  -Nécessaires pour écrire des formules chimiquess-
%-------------------------------------------------------------------------------

\usepackage[version=4]{mhchem}

%-------------------------------------------------------------------------------
% Pour pouvoir exploiter les fichiers directement dans beamer
\newcommand{\pause}{\ }
%-------------------------------------------------------------------------------
%                    - Mise en forme avancée
%-------------------------------------------------------------------------------

\usepackage{ifthen}
\usepackage{ifmtarg}


\newcommand{\ifTrue}[2]{\ifthenelse{\equal{#1}{true}}{#2}{$\qquad \qquad$}}

%\newcommand{\kword}[1]{\textcolor{red}{\underline{#1}}}
%-------------------------------------------------------------------------------

%-------------------------------------------------------------------------------
%                     -Mise en forme d'exercices-
%-------------------------------------------------------------------------------
%\newtheoremstyle{exostyle}
%{\topsep}% espace avant
%{\topsep}% espace apres
%{}% Police utilisee par le style de thm
%{}% Indentation (vide = aucune, \parindent = indentation paragraphe)
%{\bfseries}% Police du titre de thm
%{.}% Signe de ponctuation apres le titre du thm
%{ }% Espace apres le titre du thm (\newline = linebreak)
%{\thmname{#1}\thmnumber{ #2}\thmnote{. \normalfont{\textit{#3}}}}% composants du titre du thm : \thmname = nom du thm, \thmnumber = numéro du thm, \thmnote = sous-titre du thm

%\theoremstyle{exostyle}
%\newtheorem{exercice}{Exercice}
%
%\newenvironment{questions}{
%\begin{enumerate}[\hspace{12pt}\bfseries\itshape a.]}{\end{enumerate}
%} %mettre un 1 à la place du a si on veut des numéros au lieu de lettres pour les questions 
%-------------------------------------------------------------------------------

%-------------------------------------------------------------------------------
%                    - Mise en forme de tableaux -
%-------------------------------------------------------------------------------

\renewcommand{\arraystretch}{1.7}

\setlength{\tabcolsep}{1.2cm}

%-------------------------------------------------------------------------------



%-------------------------------------------------------------------------------
%                    - Racourcis d'écriture -
%-------------------------------------------------------------------------------
%Droites
\newcommand{\dte}[1]{$(#1)$}
\newcommand{\fig}[1]{figure $#1$}
\newcommand{\sym}{symétrique}
\newcommand{\syms}{symétriques}
\newcommand{\asym}{axe de symétrie}
\newcommand{\asyms}{axes de symétrie}
\newcommand{\seg}[1]{$[#1]$}
\newcommand{\monAngle}[1]{$\widehat{#1}$}
\newcommand{\bissec}{bissectrice}
\newcommand{\mediat}{médiatrice}
\newcommand{\ddte}[1]{$[#1)$}


% Angles orientés (couples de vecteurs)
\newcommand{\aopp}[2]{(\vec{#1}, \vec{#2})} %Les deuc vecteurs sont positifs
\newcommand{\aopn}[2]{(\vec{#1}, -\vec{#2})} %Le second vecteur est négatif
\newcommand{\aonp}[2]{(-\vec{#1}, \vec{#2})} %Le premier vecteur est négatif
\newcommand{\aonn}[2]{(-\vec{#1}, -\vec{#2})} %Les deux vecteurs sont négatifs

%Ensembles mathématiques
\newcommand{\naturels}{\mathbb{N}} %Nombres naturels
\newcommand{\relatifs}{\mathbb{Z}} %Nombres relatifs
\newcommand{\rationnels}{\mathbb{Q}} %Nombres rationnels
\newcommand{\reels}{\mathbb{R}} %Nombres réels
\newcommand{\complexes}{\mathbb{C}} %Nombres complexes


%Intégration des parenthèses aux cosinus
\newcommand{\cosP}[1]{\cos\left(#1\right)}
\newcommand{\sinP}[1]{\sin\left(#1\right)}


%Probas stats
\newcommand{\stat}{statistique}
\newcommand{\stats}{statistiques}


\newcommand{\homo}{homothétie}
\newcommand{\homos}{homothéties}


\newcommand{\mycoord}[3]{(\textcolor{red}{\num{#1}} ; \textcolor{Green}{\num{#2}} ; \textcolor{blue}{\num{#3}})}
%-------------------------------------------------------------------------------

%-------------------------------------------------------------------------------
%                    - Mise en page -
%-------------------------------------------------------------------------------

\newcommand{\twoCol}[1]{\begin{multicols}{2}#1\end{multicols}}


\setenumerate[1]{font=\bfseries,label=\textit{\alph*})}
\setenumerate[2]{font=\bfseries,label=\arabic*)}


%-------------------------------------------------------------------------------
%                    - Elements cours -
%-------------------------------------------------------------------------------

%Correction d'exercice
\newcommand{\exoSec}[2]{\subsection*{Exercice #1 page #2}}
%-------------------------------------------------------------------------------
%                    - raccourcis d'écriture -
%-------------------------------------------------------------------------------

%Mise en évidence de termes clés
\newcommand{\mykw}[1]{\textcolor{red}{\underline{\textbf{#1}}}}

%Exercices
\newcommand{\exo}[2]{exercice #1 page #2}
\newcommand{\Exo}[2]{Exercice #1 page #2}

\renewcommand{\pause}{\ }


%\usepackage{fullpage}
\author{\ }
\date{5 Octobre 2018}
\title{Sciences Physiques : DS n° 1}


\begin{document}
%	\usepackage{fancyhdr}
%	
%	\pagestyle{fancy}
%	\fancyhf{}
	%\rhead{Share\LaTeX}

	\maketitle
	
\begin{small}
	\begin{center}
		\begin{tabular}{|@{\ }l@{}|@{\ }c@{\ }|}
			\hline
			\textbf{Compétence} & \textbf{Maitrise} \\
			\hline
			Caractériser les différents états de la matière (solide, liquide et gaz)\ &  \ \ \ \\
			\hline	
			Changements d’états de la matière \ &  \ \ \ \\
			\hline
			Conservation de la masse, variation du volume, température de changement d’état \ &  \ \ \ \\
			\hline
		\end{tabular}
	\end{center}
\end{small}	
	
	
%\vspace*{-0.5cm}	

%\section{\'Equations de réaction}

Ajuster les équations de réactions suivantes :
\begin{questions}
	\question $CH_4 + ....O_2 \rightarrow ....CO_2 + ....H_2O$
	
	\question $C_7H_{16} + ....O_2 \rightarrow ....CO_2 + ....H_2O$	
	
	\question $C_6H_{2}O + ....O_2 \rightarrow ....CO_2 + ....H_2O$
\end{questions}


%\section{À chaque modèle sa formule}
\begin{questions}
	\question \'A partir de ces dessins de modèles, donner la formule des molécules suivantes.

	\begin{center}
		\includegraphics[scale=0.6]{img/exemples}
	\end{center}
	\fillwithdottedlines{2cm}
	
\end{questions}

%Seul l'\ref{ex:surface} est à faire sur le sujet.
Le soin et la qualité de rédaction sont pris en compte dans la notation.


\section{Quels atomes dans cette particule ?}\label{ex:particule}



\begin{questions}
	\question[] Pour chaque espèce chimique, indiquer le type d'atome, le nombre d'atomes de chaque type et le nombre total d'atomes qu'elle contient.
	 
	\begin{multicols}{4}
		\begin{itemize}
			\item $CO_2$
			\item $H_2$
			\item $CH_4$
			\item $O_2$
			\item $C_4H_{10} $
			\item $C_6H_{12} O_6$
			\item $C$
			\item $H_2O$
		\end{itemize}
	\end{multicols}
	
\end{questions}

%\newpage

\section{Une bouteille d'eau au congélateur (2 points)}\label{ex:congel}

Palmyre verse 1 L d'eau, de masse 1 kg, dans une bouteille qu'elle place ensuite au congélateur. Après quelques heures, la bouteille est déformée.

\begin{questions}
	\question[1] Que vaut alors la masse de l'eau contenue dans la bouteille ?
	
	\question[1] Que peut-on dire du volume d'eau contenu dans la bouteille ?
\end{questions}



%\section{Reconnaître des mélanges }\label{ex:melanges}

\begin{center}
	\includegraphics[scale=0.4]{img/melanges}
\end{center}

Ces schémas représentent deux mélanges différents.

\begin{questions}
	\question[] Dans quel mélange est-il possible de distinguer les espèces chimiques mélangées ? Donner un exemple d'un tel mélange.
	
	\question[] Dans quel mélange n'est-il pas possible de distinguer les espèces chimiques mélangées ? Donner un exemple d'un tel mélange.
\end{questions}



%\newpage

\section{Et la température dans tout ça ?}

Le graphique ci-dessous représente la solubilité du  sel (ou chlorure de sodium) dans l'eau en fonction de la température.

\begin{questions}
	\question Complète le graphique en indiquant les grandeurs représentées en abscisse et en ordonnée.
	
	\begin{center}
		\includegraphics[scale=0.5]{img/courbe}
	\end{center}

	\question \`A $20 °C$, quelle masse de chlorure de sodium peut-on dissoudre au maximum dans 1L de solution ?
	
	\question \`A $90 °C$, quelle masse de chlorure de sodium peut-on dissoudre au maximum dans 1L de solution ?
	
	\question De quoi dépend la solubilité du sel dans l'eau ?
\end{questions}


\newpage


\section{Quel est cet état ? (3 points)}\label{ex:etat}



\begin{questions}
	\question[3] Pour chaque phrase, indiquer quel(s) état(s) est (sont) décrit(s).
	
	
	Les molécules :\\
	\begin{parts}
		\part sont proches les unes des autres et peuvent bouger les unes par rapport aux autres.
		\begin{solution}
			L'état décrit est l'état liquide.
		\end{solution}
		
		\part sont très éloignées les unes des autres.
		\begin{solution}
			L'état décrit est l'état gazeux.
		\end{solution}
		
%		\part sont ordonnées [sont liées].
%		\begin{solution}
%			L'état décrit est l'état solide.
%		\end{solution}
%		
		\part ne peuvent pas se déplacer les unes par rapport aux autres.
		\begin{solution}
			L'état décrit est l'état solide.
		\end{solution}
		
		\part se déplacent et occupent le maximum d'espace.
		\begin{solution}
			L'état décrit est l'état gazeux.
		\end{solution}
		
		\part ont un volume propre et pas de forme propre.
		\begin{solution}
			L'état décrit est l'état solide.
		\end{solution}
		
		\part sont désordonnées [sont agitées].
		\begin{solution}
			Les états décrits sont les états liquide et gazeux.
		\end{solution}
	\end{parts}
	
\end{questions}


%\newpage 

%
\section{Monnaie en cuivre }

Au cours d'une opération de nettoyage de la plage, Romain a trouvé dix pièces de monnaie en cuivre. Il les plonge dans une éprouvette à moitié remplie d'eau. La différence de volume qu'il constate est $V = 5 cm^3$, il a trouvé sur internet que la masse volumique du cuivre est $\rho _{cuivre}$ est de $\num{8.96} g/mL$.  

\begin{questions}
	\question[] Convertir le volume $V$ des pièces de cuivre en mL.
	
	\question[] Calculer la masse $m$ de ces dix pièces de cuivre.
	
\end{questions}


%\newpage 

%\section{Dans quel état est cette matière}\label{ex:etat2}

\begin{center}
	\includegraphics[scale=0.4]{img/etats}
\end{center}

\begin{questions}
\question[] Indiquer pour chacune de ces matières si elles sont dans l'état solide; liquide ou gazeux. Justifier.

\end{questions}

%\section{Surface libre (3 points)}\label{ex:surface}

Plusieurs récipients sont remplis à ras bord avec de l'eau liquide.


\begin{questions}
	
			
		\question[3] Tracer la surface libre du liquide au repos dans chacun des récipients.
		
		\begin{center}
			\includegraphics[scale=0.4]{img/surface}
		\end{center}	
\end{questions}


%\section{Je reconnais les trois états physiques (3 poinst)}

Voici des récipients contenant des substances à l'état solide, à l'état liquide et à l'état gazeux.

\begin{questions}
	\question[3] En justifiant la réponse, indiquer l'état représenté dans chaque cas.
	%\begin{multicols}{2}
		
		\begin{center}
			\includegraphics[scale=0.4]{img/reco1}
		\end{center}
	
	\begin{solution}
		\begin{itemize}
			\item Le contenu du bécher A n'a pas de surface libre plane donc c'est un solide.
			
			\item Le contenu du bécher B occupe tout l'espace disponible c'est un gaz.
			
			\item Le contenu du bécher C a une surface libre plane et horizontale donc c'est un liquide.
		\end{itemize}
	\end{solution}


		%\begin{center}
		%	\includegraphics[scale=0.22]{img/reco2}
		%\end{center}
%		\begin{solution}
%			\begin{itemize}
%				\item Le contenu du bécher D n'a pas de surface libre horizontale donc c'est un solide.
%				
%				\item Le contenu du bécher E a une forme propre, c'est un solide.
%				
%				\item Le contenu du bécher F occupe tout l'espace disponible c'est un gaz.
%			\end{itemize}
%		\end{solution}

	%\end{multicols}
\end{questions}

\section{La température qui monte (4 points)}\label{ex:fusion}

Dans un récipient qui contient de l'eau, on a placé un thermomètre. On a relevé la température de l'eau toutes les 10 min.

\begin{center}
	\includegraphics[scale=0.4]{./img/courbe}
\end{center}

\begin{questions}
	\question[2] Quel est l'état de l'eau après 10 minutes ? Après 60 minutes ?
	\begin{solution}
		Après 10 min, la température est inférieure à 0°C donc l'eau est solide. Après 60 min, le température est supérieure à 0°C, donc l'eau est liquide.
	\end{solution}
	
	\question[1] Combien de temps a duré le changement d'état ?
	\begin{solution}
		Sur la courbe, il y a un palier de température à 0°C entre 20 et 40 min, donc le changement d'état a duré 20 min.
	\end{solution}
	
	\question[1] A quel instant n'y a-t-il plus d'eau solide dans le récipient. 
	\begin{solution}
		Il n'y a aura plus d'eau solide dans le récipient à la fin du changement d'état, donc à 40 min.
	\end{solution}
\end{questions}
 
\section{Réaliser des soudures sur les circuits (3 points)}

En électronique, pour fixer un composant sur un circuit imprimé, on fait fondre un fil d'étain (métal dont la température de fusion est 232 °C) avec un fer à souder. La goutte d'étain déposée sur le circuit refroidit, fixant ainsi le composant sur le circuit.

\begin{questions}
	\question[1] Donner le nom du changement d'état subit par l'étain lorsqu'on le chauffe au fer à souder.
	\begin{solution}
		Le changement d'état qui à lieu lorsque l'étain chauffe au fer à souder est la fusion.
	\end{solution}
	
	\question[1] Nommer le changement d'état subit par la goutte d'étain sur le circuit en refroidissant.
	\begin{solution}
		Le changement d'état qui à lieu lorsque l'étain refroidit est la solidification.
	\end{solution}
	
	\question[1] Justifier l'utilisation de l'étain pour effectuer les soudures plutôt que le fer dont la température de fusion est de \num{1535} °C.
	\begin{solution}
		Les soudures sont réalisées en étain plutôt qu'en fer, car la température du fusion du fer est beaucoup plus importante et donc qu'il faudrait fournir beaucoup plus d'énergie pour le faire fondre. 
	\end{solution}
\end{questions}
%\newpage
%
%\includegraphics [scale=0.5, angle= 90 ]{img/tableau} 
\ \label{LastPage}

\end{document}