\section{Chauffer de l'eau pour faire cuire du riz (4 points)}

Pour la cuisson du riz, on peut lire sur le paquet : << verser un volume de riz dans 5 volumes d'eau bouillante >>.

\begin{questions}
	\question[1\half] Indiquer ce que fournit le dispositif de chauffage pour augmenter la température de l'eau.
	\begin{solution}
		Le dispositif de chauffage apporte de l'énergie thermique.
	\end{solution}
	
	\question[1\half] Pourquoi il faut fournir plus d'énergie lorsque l'on met initialement dans la casserole de l'eau froide, plutôt que de l'eau chaude, pour la porter à ébullition ?
	\begin{solution}
		Dans un premier temps, l'énergie fournie permet de faire monter la température de l'eau jusqu'à la température d'ébullition. Donc plus l'eau mise dans la casserole est froide et plus il faudra fournir d'énergie pour arriver à la température d'ébullition.
	\end{solution}
	
	\question[1] Indiquer à quoi sert l'énergie fournie par la plaque électrique, une fois l'eau à ébullition.
	\begin{solution}
		Une fois l'eau à ébullition, l'énergie fournie sert à effectuer la changement d'état.
	\end{solution}
\end{questions}