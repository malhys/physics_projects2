\section{Réaliser des soudures sur les circuits (3 points)}

En électronique, pour fixer un composant sur un circuit imprimé, on fait fondre un fil d'étain (métal dont la température de fusion est 232 °C) avec un fer à souder. La goutte d'étain déposée sur le circuit refroidit, fixant ainsi le composant sur le circuit.

\begin{questions}
	\question[1] Donner le nom du changement d'état subit par l'étain lorsqu'on le chauffe au fer à souder.
	\begin{solution}
		Le changement d'état qui à lieu lorsque l'étain chauffe au fer à souder est la fusion.
	\end{solution}
	
	\question[1] Nommer le changement d'état subit par la goutte d'étain sur le circuit en refroidissant.
	\begin{solution}
		Le changement d'état qui à lieu lorsque l'étain refroidit est la solidification.
	\end{solution}
	
	\question[1] Justifier l'utilisation de l'étain pour effectuer les soudures plutôt que le fer dont la température de fusion est de \num{1535} °C.
	\begin{solution}
		Les soudures sont réalisées en étain plutôt qu'en fer, car la température du fusion du fer est beaucoup plus importante et donc qu'il faudrait fournir beaucoup plus d'énergie pour le faire fondre. 
	\end{solution}
\end{questions}