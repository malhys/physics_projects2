\documentclass[a4paper]{article}

\usepackage[utf8x]{inputenc}    
\usepackage[T1]{fontenc}
\usepackage{multicol}
\usepackage[output-decimal-marker={,}]{siunitx}

\usepackage[francais,bloc,completemulti]{automultiplechoice}    
\begin{document}
	
	\exemplaire{10}{    
		
		%%% debut de l'en-tête des copies :    
		
		\noindent{\bf Sciences Physiques  \hfill Quatrième}
		
		\vspace*{.5cm}
		\begin{minipage}{.4\linewidth}
			\centering\large\bf Test sur la matière\\ 28/09/2018\end{minipage}
		\champnom{\fbox{    
				\begin{minipage}{.5\linewidth}
					Nom et prénom :
					
					\vspace*{.5cm}\dotfill
					\vspace*{1mm}
				\end{minipage}
			}}
			
			\begin{center}\em
				Durée : 15 minutes.
				
				%  Aucun document n'est autorisé.
				%  L'usage de la calculatrice est interdit.
				
				  Les questions faisant apparaître le symbole \multiSymbole{} peuvent
				  présenter zéro, une ou plusieurs bonnes réponses. Les autres ont
				  une unique bonne réponse.
				%
				Une bonne réponse rapporte 2 points et une mauvaise en retire 0,5.
			\end{center}
			\vspace{1ex}
			
			%%% fin de l'en-tête
			\begin{multicols}{2}
				
				\begin{question}{neige}    
					La neige est de l'eau à l'état~:
					\begin{reponses}
						\bonne{Solide}\bareme{2}
						\mauvaise{Liquide}\bareme{-0.5}
						\mauvaise{Gazeux}\bareme{-0.5}
					\end{reponses}
				\end{question}
				
				\begin{questionmult}{gaz}    
					Un gaz~:
					\begin{reponses}
						\bonne{n'a pas de forme propre.}\bareme{1}
						\bonne{est compressible.}\bareme{1}
						\mauvaise{a un volume propre.}\bareme{-0.5}
					\end{reponses}
				\end{questionmult}
				
				
				\begin{question}{glacon}
					L'eau d'un glaçon exposé au Soleil un jour d'été est~:
					\begin{reponses}
						\bonne{d'abord solide puis liquide.}\bareme{2}
						\mauvaise{d'abord liquide puis solide.}\bareme{-0.5}
						\mauvaise{d'abord gazeuse puis solide.}\bareme{-0.5}
					\end{reponses}
				\end{question}
				
				\begin{question}{chgmt}    
					Pendant un changement d'état d'un corps pur, sa température~:
					\begin{reponses}		
						\mauvaise{augmente.}\bareme{-0.5}
						\bonne{reste constante.}\bareme{2}
						\mauvaise{diminue.}\bareme{-0.5}						
					\end{reponses}
				\end{question}
				
				\begin{question}{CO2}    
					du dioxyde de carbone pur est constitué~:
					\begin{reponses}		
						\mauvaise{d'un seul type d'atomes.}\bareme{-0.5}
						\mauvaise{d'au moins trois types d'atomes.}\bareme{-0.5}
						\bonne{d'un seul type de molécules.}\bareme{2}
						\mauvaise{d'au moins trois types de molécules.}\bareme{-0.5}
					\end{reponses}
				\end{question}
				
				\begin{question}{melange}    
					Un mélange est constitué~:
					\begin{reponses}		
						\mauvaise{d'un seul type d'atomes}\bareme{-0.5}
						\bonne{de plusieurs espèces chimiques.}\bareme{2}
						\mauvaise{d'une seule espèce chimique.}\bareme{-0.5}
						\mauvaise{de plusieurs types d'atomes.}\bareme{-0.5}
						
					\end{reponses}
				\end{question}
				
				\begin{question}{liquide}    
					Au niveau microscopiques, dans un liquide, les molécules sont :
					\begin{reponses}		
						\mauvaise{proches les unes des autres et ordonnées}\bareme{-0.5}
						\mauvaise{éloignées les unes des autres et peuvent se déplacer.}\bareme{-0.5}
						\mauvaise{éloignées les unes des autres et ordonnées.}\bareme{-0.5}
						\bonne{proches les unes des autres et peuvent se déplacer.}\bareme{2}
						
					\end{reponses}
				\end{question}
				
				
				
				\begin{question}{solidification}
					Lorsque de l'eau passe de l'état liquide à l'état solide~:
					\begin{reponses}
						\mauvaise{sa masse et son volume changent.}\bareme{-0.5}
						\bonne{seul son volume change.}\bareme{2}
						\mauvaise{seule sa masse change.}\bareme{-0.5}
					\end{reponses}
				\end{question}
				
				\begin{question}{surface}
					La surface libre d'un liquide est~:
					\begin{reponses}
						\mauvaise{parallèle au mur.}\bareme{-0.5}
						\mauvaise{parallèle au sol.}\bareme{-0.5}
						\bonne{plane et horizontale.}\bareme{2}
						\mauvaise{perpendiculaire au sol.}\bareme{-0.5}
					\end{reponses}   
					
				\end{question}
				
				
				\begin{question}{energie}
					Lorsqu'on fournit de l'énergie a de l'eau à 25 °C~:
					\begin{reponses}
						\mauvaise{elle change d'état.}\bareme{-0.5}
						\mauvaise{sa température reste constante.}\bareme{-0.5}
						\bonne{sa température augmente.}\bareme{2}
					\end{reponses}   
					
				\end{question}
				
				%\begin{question}{air}
				%	L'air qui nous entoure :
				%	\begin{reponses}
				%		\mauvaise{est composé d'un seul gaz}
				%		\bonne{est une mélange de plusieurs gaz}
				%		\mauvaise{ne contient pas de gaz} 
				%	\end{reponses}
				%\end{question}
				
			\end{multicols} 
			
			\clearpage    
			
			
		}  
		
	\end{document}
	\grid
\grid
\grid
