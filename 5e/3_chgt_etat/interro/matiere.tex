\documentclass[a4paper]{article}

\usepackage[utf8x]{inputenc}    
\usepackage[T1]{fontenc}
\usepackage{multicol}
\usepackage[output-decimal-marker={,}]{siunitx}

\usepackage[francais,bloc,completemulti]{automultiplechoice}    
\begin{document}

\exemplaire{10}{    

%%% debut de l'en-tête des copies :    

\noindent{\bf Sciences Physiques  \hfill Cinquième}

\vspace*{.5cm}
\begin{minipage}{.4\linewidth}
\centering\large\bf Test de rentrée\\ 18/09/2018\end{minipage}
\champnom{\fbox{    
                \begin{minipage}{.5\linewidth}
                  Nom et prénom :

                  \vspace*{.5cm}\dotfill
                  \vspace*{1mm}
                \end{minipage}
         }}

\begin{center}\em
Durée : 15 minutes.

%  Aucun document n'est autorisé.
%  L'usage de la calculatrice est interdit.

%  Les questions faisant apparaître le symbole \multiSymbole{} peuvent
%  présenter zéro, une ou plusieurs bonnes réponses. Les autres ont
%  une unique bonne réponse.
%
Une bonne réponse rapporte 2 points et une mauvaise en retire 0,5.
\end{center}
\vspace{1ex}

%%% fin de l'en-tête
\begin{multicols}{2}
	
\begin{question}{neige}    
  La neige est de l'eau à l'état~:
  \begin{reponses}
    \bonne{Solide}\bareme{b=2}
    \mauvaise{Liquide}\bareme{b=-0.5}
    \mauvaise{Gazeux}\bareme{b=-0.5}
  \end{reponses}
\end{question}

\begin{questionmult}{gaz}    
  Un gaz~:
  \begin{reponses}
    \bonne{n'a pas de forme propre.}\bareme{b=1}
    \bonne{est compressible.}\bareme{b=1}
    \mauvaise{a un volume propre.}\bareme{b=-0.5}
  \end{reponses}
\end{questionmult}


\begin{question}{glacon}
	L'eau d'un glaçon exposé au Soleil un jour d'été est~:
	\begin{reponses}
		\bonne{d'abord solide puis liquide.}\bareme{b=2}
		\mauvaise{d'abord liquide puis solide.}\bareme{b=-0.5}
		\mauvaise{d'abord gazeuse puis solide.}\bareme{b=-0.5}
	\end{reponses}
\end{question}

\begin{question}{matiere}    
	la matière est constituée~:
	\begin{reponses}		
		\mauvaise{de grains de sable.}\bareme{b=-0.5}
		\bonne{d'atomes.}\bareme{b=2}
		\mauvaise{d'air.}\bareme{b=-0.5}
		\mauvaise{de quatre éléments}\bareme{b=-0.5}
	\end{reponses}
\end{question}

\begin{question}{CO2}    
	du dioxyde de carbone pur est constitué~:
	\begin{reponses}		
		\mauvaise{d'un seul type d'atomes.}\bareme{b=-0.5}
		\mauvaise{d'au moins trois types d'atomes.}\bareme{b=-0.5}
		\bonne{d'un seul type de molécules.}\bareme{b=2}
		\mauvaise{d'au moins trois types de molécules.}\bareme{b=-0.5}
	\end{reponses}
\end{question}

\begin{question}{melange}    
	Un mélange est constitué~:
	\begin{reponses}		
		\mauvaise{d'un seul type d'atomes}\bareme{b=-0.5}
		\bonne{de plusieurs espèces chimiques.}\bareme{b=2}
		\mauvaise{d'une seule espèce chimique.}\bareme{b=-0.5}
		\mauvaise{de plusieurs types d'atomes.}\bareme{b=-0.5}
		
	\end{reponses}
\end{question}

\begin{question}{liquide}    
	Au niveau microscopiques, dans un liquide, les molécules sont :
	\begin{reponses}		
		\mauvaise{proches les unes des autres et ordonnées}\bareme{b=-0.5}
		\mauvaise{éloignées les unes des autres et peuvent se déplacer.}\bareme{b=-0.5}
		\mauvaise{éloignées les unes des autres et ordonnées.}\bareme{b=-0.5}
		\bonne{proches les unes des autres et peuvent se déplacer.}\bareme{b=2}
		
	\end{reponses}
\end{question}



\begin{question}{etats}
	Quels sont les différents états physiques de la matière~?
	\begin{reponses}
		\mauvaise{Rouge, Vert et Bleu}\bareme{b=-0.5}
		\bonne{Solide, Liquide et Gazeux}\bareme{b=2}
		\mauvaise{Résistant, Souple et Dur}\bareme{b=-0.5}
	\end{reponses}
\end{question}

\begin{question}{surface}
	La surface libre d'un liquide est~:
	\begin{reponses}
		\mauvaise{parallèle au mur.}\bareme{b=-0.5}
		\mauvaise{parallèle au sol.}\bareme{b=-0.5}
		\bonne{plane et horizontale.}\bareme{b=2}
		\mauvaise{perpendiculaire au sol.}\bareme{b=-0.5}
	\end{reponses}   
	  
\end{question}


\begin{question}{volume}
	L'unité officielle de mesure du volume est le~:
	\begin{reponses}
		\mauvaise{centimètre cube.}\bareme{b=-0.5}
		\mauvaise{mètre carré.}\bareme{b=-0.5}
		\bonne{mètre cube.}\bareme{b=2}
		\mauvaise{mètre.}\bareme{b=-0.5}
	\end{reponses}   
	
\end{question}

%\begin{question}{air}
%	L'air qui nous entoure :
%	\begin{reponses}
%		\mauvaise{est composé d'un seul gaz}
%		\bonne{est une mélange de plusieurs gaz}
%		\mauvaise{ne contient pas de gaz} 
%	\end{reponses}
%\end{question}

\end{multicols} 

\clearpage    


}  

\end{document}
\grid
