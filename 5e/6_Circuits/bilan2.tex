\begin{mybilan}
	
	Un \kw{dipôle électrique}, est un composant électrique qui possède deux bornes. La pile et la lampe sont des dipôles
	
	\begin{itemize}
		\item Un circuit électrique simple est formé par une \kw{boucle} qui comporte une \kw{source d'énergie}, un \kw{interrupteur}, un \kw{dipôle récepteur} (ex : une lampe) reliés par des \kw{fils de connexion.}\pause
		
		\item Si la lampe brille, \kw{le courant électrique circule} : on dit que le circuit est \kw{fermé}.\pause
		
		\item Si la lampe est éteinte, \kw{le courant ne circule plus} : on dit que le circuit est \kw{ouvert}.\pause
		
		\item Dans un circuit électrique, on considère que \kw{le courant circule}, à l'extérieur du générateur, \kw{de la borne +, vers la borne -}.
	\end{itemize}



\end{mybilan}