\begin{myact}{3 page 154}
	\begin{enumerate}
		\item Après avoir appuyé sur le bouton TARE, la valeur 0 s'affiche sur la balance. Avant d'avoir utilisé ce bouton la balance indiquait la masse de la soucoupe. \pause
		\item Le symbole $g$ apparaît sur l'écran de la balance. Ce symbole correspond au gramme.\pause
		\item La masse de l'objet posé sur la soucoupe est de \num{7.6} $g$.\pause
		\item La masse de l'eau dans la fiole est de \num{1000} $g$.\pause
		\item Faire la tare sur une balance permet d'ignorer la masse d'un récipient.\pause
		\item On a 1 $kg$ = \num{1000}$g$.\pause
		\item La masse de  1 $L$ d'eau est de 1 $kg$.
	\end{enumerate}
\end{myact}