\subsection*{\Exo{2}{132}}

	
	\setlength{\tabcolsep}{4pt}
	\begin{tabular}{|l|c|c|c|}
		\hline
		\textbf{Caractéristiques}                                                                            & \textbf{Solides} & \textbf{Liquides} & \textbf{Gaz} \\ \hline
		Ils ont une forme propre                                                                             & x                &                   &              \\ \hline
		\begin{tabular}[c]{@{}l@{}}Ils occupent tout le volume \\ du récipient qui les contient\end{tabular} &                  &                   & x            \\ \hline
		\begin{tabular}[c]{@{}l@{}}Au repos, leur surface libre\\  est plane et horizontale\end{tabular}     &                  & x                 &              \\ \hline
		\begin{tabular}[c]{@{}l@{}}Ils prennent la forme du \\ récipient qui les contient\end{tabular}       &                  & x                 & x            \\ \hline
		On peut les saisir avec les doigts                                                                   & x                &                   &              \\ \hline
	\end{tabular}

\subsection*{\Exo{6}{133}}

\begin{enumerate}[label=\alph*)]
	\item La surface libre d'un liquide est la surface en contact avec l'air.
	\item Le rôle du fil à plomb est d'indiquer la direction verticale et celui de l'équerre est d'indiquer la direction horizontale.
	\item On en conclu que la surface libre du liquide est horizontale.
\end{enumerate}
\subsection*{\Exo{3}{132}}

\begin{enumerate}[label=\alph*)]
	\item \twoCol{\begin{enumerate}
		\item Eau pétillante
		\item Ballon
		\item Tube à dégagement
		\item Tube à essais
		\item Eau de chaux
		\item Précipité blanc
	\end{enumerate}}
	
	\item De l'eau pétillante est placée dans un ballon. Le gaz contenu dans l'eau pétillante est extrait dans un tube à dégagement. L'autre extrémité du tube à dégagement est placée dasn un tube à essais qui contient de l'eau de chaux. Un précipité blanc se forme, le gaz dissous dans l'eau pétillante est du dioxyde de carbone.
\end{enumerate}

\subsection*{\Exo{4}{132}}

\begin{enumerate}[label=\alph*)]
	\item L'eau peut contenir des gaz \mykw{dissous}.
	\item On peut extraire un gaz d'une boisson pétillante en \mykw{agitant} ou en \mykw{chauffant} le liquide.
	\item Les gaz extraits sont récupérés par \mykw{déplacement} d'eau.
	\item Le gaz dissous dans une eau pétillante est du \mykw{dioxyde de carbone}.
	\item Pour l'identifier, on utilise le test à \mykw{l'eau de chaux}. En sa présence il se forme un \mykw{précipité blanc}.
\end{enumerate}

\subsection*{\Exo{5}{133}}

\begin{enumerate}[label=\alph*)]
	\item 

\begin{tabular}{|l|c|c|}
	\hline
	\textbf{Mot}        & \textbf{\'Etat solide} & \textbf{\'Etat liquide} \\ \hline
	pluie      &               & x              \\ \hline
	nuages     &               & x              \\ \hline
	buée       &               & x              \\ \hline
	glaciers   & x             &                \\ \hline
	mer        &               & x              \\ \hline
	neige      & x             &                \\ \hline
	givre      & x             &                \\ \hline
	brouillard &               & x              \\ \hline
	rivière    &               & x             \\ \hline
\end{tabular}

\item L'état gazeux n'est pas représenté.
\item On ne peut pas le voir car la vapeur d'eau est invisible.

\end{enumerate}

\subsection*{\Exo{8}{133}}

\begin{enumerate}[label=\alph*)]
	\item Le volume d'un gaz que l'on comprime diminue.
	\item Seringues classées par ordre croissant de compression : 2 ; 1 ; 3
\end{enumerate}

\subsection*{\Exo{11}{133}}

\begin{enumerate}[label=\alph*)]
	\item On observe que l'eau de chaux se trouble.
	\item On peut conclure que l'air expiré par Marine contient du dioxyde de carbone.
\end{enumerate}

\subsection*{\Exo{14}{134}}
\begin{enumerate}[label=\alph*)]
	\item Le contenu du flacon supérieur est devenu roux.
	\item Lorsque la coupelle a été enlevée le gaz présent dans le flacon du dessous s'est étendu pour occuper l'espace du flacon du haut.
\end{enumerate}

\subsection*{\Exo{15}{134}}
\begin{enumerate}[label=\alph*)]
	\item Le ballon gonfle rapide car l'eau a été <<renforcée en gaz de la source>>.
	\item Le gaz présent dans cette eau minérale est du dioxyde de carbone.
	\item Pour le mettre en évidence on peut faire le test de reconnaissance à l'eau de chaux.
\end{enumerate}