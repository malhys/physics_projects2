\begin{myact}{3 page 92}
	\begin{enumerate}
		\item Lors d'une éclipse de Lune, la Terre se trouve entre le Soleil et la Lune.\pause
		\item La zone la plus sombre observée sur le document correspond au cône d'ombre de la Terre.\pause
		\item Une éclipse totale de Lune se produit à la pleine Lune.\pause
		\item Pour qu'il y ait une éclipse, la Lune doit passer dans le cône d'ombre de la Terre.\pause
		\item Une éclipse de Lune est visible de toute la partie de la Terre pour laquelle il fait nuit.\pause
		\item Il n'y a pas éclipse à chaque pleine Lune, car la Lune n'est pas forcément dans le cône d'ombre de la Terre.\pause
		\item Au cours de l'éclipse, la Lune prend une teinte rouge.		
	\end{enumerate}
\end{myact}