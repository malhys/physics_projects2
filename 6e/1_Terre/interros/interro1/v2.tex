%Pour chaque question, cocher la (ou les) bonne(s) réponses.
%Chaque question vaut 1 point et chaque mauvaise réponse en retire \num{0.25}.

\begin{questions}
\question[2] Combien d'étoiles y a-t-il dans le Système Solaire ?
\fillwithdottedlines{1.5cm}

\question[2] Quelle est la planète la plus proche du Soleil ?
\fillwithdottedlines{1.5cm}

\question[2] Qu'est ce qu'une planète tellurique ?\
\fillwithdottedlines{1.5cm}


\question[2] Nommer les planètes gazeuses :
\fillwithdottedlines{2cm}


\question[2] La planète Terre :
\begin{checkboxes}
	\correctchoice est la troisième planète autour du Soleil.
	\choice est au centre de l'Univers.
	\choice est au centre du Système Solaire.\\ 
\end{checkboxes}


\question[2] Quels sont les mouvements des objets célestes ?
\begin{checkboxes}
	
	\correctchoice La Terre tourne sur elle-même et autour du Soleil.
	\choice La Terre est immobile ainsi que l'ensemble des objets célestes.
	\choice La Terre est immobile et le Soleil et les planètes tournent autour.\\	 
\end{checkboxes}

\newpage

\question[2] Combien de temps met la Terre pour tourner sur elle-même ?
\fillwithdottedlines{1.5cm}

\question[2] Comment s'appelle l'étoile du Système Solaire ?
\fillwithdottedlines{1.5cm}


\question[2] Quel est le diamètre de la Terre ?\\ 
\begin{oneparcheckboxes}
	\correctchoice \num{12800} km.
	\choice \num{64000} km.
	\choice \num{128000} km.
\end{oneparcheckboxes}


\question[2] Quelle distance nous sépare du Soleil ?\\
\begin{oneparcheckboxes}
	\choice \num{150000000000} km.
	\correctchoice \num{150000000} km.
	\choice \num{150000} km.
\end{oneparcheckboxes}\\


\end{questions}

