\documentclass[xcolor={dvipsnames}]{beamer}
%\usepackage[utf8]{inputenc}
%\usetheme{Malmoe}
\usetheme{CambridgeUS}
%\usecolortheme{beaver}

%-------------------------------------------------------------------------------
%          -Packages nécessaires pour écrire en Français et en UTF8-
%-------------------------------------------------------------------------------
\usepackage[utf8]{inputenc}
\usepackage[french]{babel}
\usepackage[T1]{fontenc}
\usepackage{lmodern}
\usepackage{textcomp}

%-------------------------------------------------------------------------------

%-------------------------------------------------------------------------------
%                          -Outils de mise en forme-
%-------------------------------------------------------------------------------
\usepackage{hyperref}
\hypersetup{pdfstartview=XYZ}
\usepackage{enumerate}
\usepackage{graphicx}
%\usepackage{multicol}
%\usepackage{tabularx}

%\usepackage{anysize} %%pour pouvoir mettre les marges qu'on veut
%\marginsize{2.5cm}{2.5cm}{2.5cm}{2.5cm}

\usepackage{indentfirst} %%pour que les premier paragraphes soient aussi indentés
\usepackage{verbatim}
%\usepackage[table]{xcolor}  
%\usepackage{multirow}
\usepackage{ulem}
%-------------------------------------------------------------------------------


%-------------------------------------------------------------------------------
%                  -Nécessaires pour écrire des mathématiques-
%-------------------------------------------------------------------------------
\usepackage{amsfonts}
\usepackage{amssymb}
\usepackage{amsmath}
\usepackage{amsthm}
\usepackage{tikz}
\usepackage{xlop}
\usepackage[output-decimal-marker={,}]{siunitx}
%-------------------------------------------------------------------------------

%-------------------------------------------------------------------------------
%                  -Nécessaires pour écrire des formules chimiquess-
%-------------------------------------------------------------------------------

\usepackage[version=4]{mhchem}

%-------------------------------------------------------------------------------
%                    - Mise en forme 
%-------------------------------------------------------------------------------

\newcommand{\bu}[1]{\underline{\textbf{#1}}}


\usepackage{ifthen}


\newcommand{\ifTrue}[2]{\ifthenelse{\equal{#1}{true}}{#2}{$\qquad \qquad$}}

\newcommand{\kword}[1]{\textcolor{red}{\underline{#1}}}


%-------------------------------------------------------------------------------



%-------------------------------------------------------------------------------
%                    - Racourcis d'écriture -
%-------------------------------------------------------------------------------

% Angles orientés (couples de vecteurs)
\newcommand{\aopp}[2]{(\vec{#1}, \vec{#2})} %Les deuc vecteurs sont positifs
\newcommand{\aopn}[2]{(\vec{#1}, -\vec{#2})} %Le second vecteur est négatif
\newcommand{\aonp}[2]{(-\vec{#1}, \vec{#2})} %Le premier vecteur est négatif
\newcommand{\aonn}[2]{(-\vec{#1}, -\vec{#2})} %Les deux vecteurs sont négatifs

%Ensembles mathématiques
\newcommand{\naturels}{\mathbb{N}} %Nombres naturels
\newcommand{\relatifs}{\mathbb{Z}} %Nombres relatifs
\newcommand{\rationnels}{\mathbb{Q}} %Nombres rationnels
\newcommand{\reels}{\mathbb{R}} %Nombres réels
\newcommand{\complexes}{\mathbb{C}} %Nombres complexes


%Intégration des parenthèses aux cosinus
\newcommand{\cosP}[1]{\cos\left(#1\right)}
\newcommand{\sinP}[1]{\sin\left(#1\right)}

%Fractions
\newcommand{\myfrac}[2]{{\LARGE $\frac{#1}{#2}$}}

%Vocabulaire courrant
\newcommand{\cad}{c'est-à-dire}

%Droites
\newcommand{\dte}[1]{$(#1)$}
\newcommand{\fig}[1]{figure $#1$}
\newcommand{\sym}{symétrique}
\newcommand{\syms}{symétriques}
\newcommand{\asym}{axe de symétrie}
\newcommand{\asyms}{axes de symétrie}
\newcommand{\seg}[1]{$[#1]$}
\newcommand{\monAngle}[1]{$\widehat{#1}$}
\newcommand{\bissec}{bissectrice}
\newcommand{\mediat}{médiatrice}
\newcommand{\ddte}[1]{$[#1)$}

%Figures
\newcommand{\para}{parallélogramme}
\newcommand{\paras}{parallélogrammes}
\newcommand{\myquad}{quadrilatère}
\newcommand{\myquads}{quadrilatères}
\newcommand{\co}{côtés opposés}
\newcommand{\diag}{diagonale}
\newcommand{\diags}{diagonales}
\newcommand{\supp}{supplémentaires}
\newcommand{\car}{carré}
\newcommand{\cars}{carrés}
\newcommand{\rect}{rectangle}
\newcommand{\rects}{rectangles}
\newcommand{\los}{losange}
\newcommand{\loss}{losanges}


\newcommand{\homo}{homothétie}
\newcommand{\homos}{homothéties}




%----------------------------------------------------
% Environnements de cours
%------------------------------------------------------



%\usepackage{../../../../pas-math}
\usepackage{../../../moncours_beamer}





\graphicspath{{../img/}}

\title{Chapitre 1 : Où est située la Terre ?}
%\author{O. FINOT}\institute{Collège S$^t$ Bernard}


\AtBeginSection[]
{
	\begin{frame}
		\frametitle{}
		\tableofcontents[currentsection, hideallsubsections]
	\end{frame} 

}


%\AtBeginSubsection[]
%{
%	\begin{frame}
%		\frametitle{Sommaire}
%		\tableofcontents[currentsection, currentsubsection]
%	\end{frame} 
%}

\begin{document}

\begin{frame}
  \titlepage 
\end{frame}



\begin{frame}

La Terre se trouve dans le \mykw{système solaire}.
Le système solaire est composé d'une étoile (le Soleil) et de huit planètes. \pause De la plus proche à la plus éloignée du Soleil :



\begin{block}<2->{Planètes telluriques (rocheuses)}
	\begin{center}
		\begin{itemize}
			\item Mercure
			\item Venus
			\item Terre
			\item Mars
		\end{itemize}
	\end{center}
\end{block}

\begin{block}<3>{Planètes gazeuses}
	\begin{center}
		\begin{itemize}
			\item Jupiter
			\item Saturne
			\item Uranus
			\item Neptune
		\end{itemize}
	\end{center}
\end{block}



\end{frame}

\begin{frame}
\begin{block}{Activité}
	Compléter le schéma :
	\begin{center}
		\includegraphics[scale=0.3]{schema_systeme}
	\end{center}
\end{block}

\end{frame}


\begin{frame}

\begin{block}{Remarque}
	Moyens mnémotechniques pour retenir l'ordre des planètes :
	\begin{itemize}
		\item \textbf{M}on \textbf{V}élo \textbf{T}ourne \textbf{M}al, \textbf{J}'en \textbf{S}ouhaite \textbf{U}n \textbf{N}ouveau.
		\item \textbf{M}e \textbf{V}oici \textbf{T}oute \textbf{M}ignonne, \textbf{J}e \textbf{S}uis \textbf{U}ne \textbf{N}ébuleuse.
	\end{itemize}
\end{block}

\end{frame}

\begin{frame}
\begin{block}{Ordres de grandeurs}
	\begin{itemize}
		\item La distance qui nous sépare du Soleil est de : 150 000 000 km.
		\item Le diamètre de la Terre est de 12 800 km.
		\item Le diamètre du Soleil est de 1 400 000 km, soit 140 fois plus gros !
		\item La distance qui sépare le Soleil de Neptune est de : 
		4 500 000 000 km.
		\item Il existe d’autres planètes dans le système solaire, des planètes naines, donc plus petites et pas forcément sphériques !
		
	\end{itemize}
\end{block}

\end{frame}


\begin{frame}

\begin{block}{Mouvements de la Terre }
	\begin{itemize}
		\item La Terre tourne sur elle-même en 24 heures.
		\item Elle tourne autour du Soleil en 365 jours.
		\item Durant l'année il y a deux solstices et deux équinoxes :
			\begin{itemize}
				\item Aux équinoxes (21 mars et 23 septembre) les jours sont aussi longs que les nuits sur toute la Terre.  
				\item Au solstice d'hiver, le 22 décembre dans l'hémisphère nord, la nuit est la plus longue de l'année.
				\item Au solstice d'été, le 21 juin dans l'hémisphère nord, la nuit est la plus courte de l'année.
			\end{itemize}
	\end{itemize}
\end{block}

	\begin{center}
		\includegraphics[scale=0.2]{solstices}
	\end{center}



\end{frame}

\end{document}


\begin{frame}


\end{frame}