\documentclass[xcolor={dvipsnames}]{beamer}
%\usepackage[utf8]{inputenc}
%\usetheme{Malmoe}
\usetheme{CambridgeUS}
%\usecolortheme{beaver}

%-------------------------------------------------------------------------------
%          -Packages nécessaires pour écrire en Français et en UTF8-
%-------------------------------------------------------------------------------
\usepackage[utf8]{inputenc}
\usepackage[french]{babel}
\usepackage[T1]{fontenc}
\usepackage{lmodern}
\usepackage{textcomp}

%-------------------------------------------------------------------------------

%-------------------------------------------------------------------------------
%                          -Outils de mise en forme-
%-------------------------------------------------------------------------------
\usepackage{hyperref}
\hypersetup{pdfstartview=XYZ}
\usepackage{enumerate}
\usepackage{graphicx}
%\usepackage{multicol}
%\usepackage{tabularx}

%\usepackage{anysize} %%pour pouvoir mettre les marges qu'on veut
%\marginsize{2.5cm}{2.5cm}{2.5cm}{2.5cm}

\usepackage{indentfirst} %%pour que les premier paragraphes soient aussi indentés
\usepackage{verbatim}
%\usepackage[table]{xcolor}  
%\usepackage{multirow}
\usepackage{ulem}
%-------------------------------------------------------------------------------


%-------------------------------------------------------------------------------
%                  -Nécessaires pour écrire des mathématiques-
%-------------------------------------------------------------------------------
\usepackage{amsfonts}
\usepackage{amssymb}
\usepackage{amsmath}
\usepackage{amsthm}
\usepackage{tikz}
\usepackage{xlop}
\usepackage[output-decimal-marker={,}]{siunitx}
%-------------------------------------------------------------------------------

%-------------------------------------------------------------------------------
%                  -Nécessaires pour écrire des formules chimiquess-
%-------------------------------------------------------------------------------

\usepackage[version=4]{mhchem}

%-------------------------------------------------------------------------------
%                    - Mise en forme 
%-------------------------------------------------------------------------------

\newcommand{\bu}[1]{\underline{\textbf{#1}}}


\usepackage{ifthen}


\newcommand{\ifTrue}[2]{\ifthenelse{\equal{#1}{true}}{#2}{$\qquad \qquad$}}

\newcommand{\kword}[1]{\textcolor{red}{\underline{#1}}}


%-------------------------------------------------------------------------------



%-------------------------------------------------------------------------------
%                    - Racourcis d'écriture -
%-------------------------------------------------------------------------------

% Angles orientés (couples de vecteurs)
\newcommand{\aopp}[2]{(\vec{#1}, \vec{#2})} %Les deuc vecteurs sont positifs
\newcommand{\aopn}[2]{(\vec{#1}, -\vec{#2})} %Le second vecteur est négatif
\newcommand{\aonp}[2]{(-\vec{#1}, \vec{#2})} %Le premier vecteur est négatif
\newcommand{\aonn}[2]{(-\vec{#1}, -\vec{#2})} %Les deux vecteurs sont négatifs

%Ensembles mathématiques
\newcommand{\naturels}{\mathbb{N}} %Nombres naturels
\newcommand{\relatifs}{\mathbb{Z}} %Nombres relatifs
\newcommand{\rationnels}{\mathbb{Q}} %Nombres rationnels
\newcommand{\reels}{\mathbb{R}} %Nombres réels
\newcommand{\complexes}{\mathbb{C}} %Nombres complexes


%Intégration des parenthèses aux cosinus
\newcommand{\cosP}[1]{\cos\left(#1\right)}
\newcommand{\sinP}[1]{\sin\left(#1\right)}

%Fractions
\newcommand{\myfrac}[2]{{\LARGE $\frac{#1}{#2}$}}

%Vocabulaire courrant
\newcommand{\cad}{c'est-à-dire}

%Droites
\newcommand{\dte}[1]{$(#1)$}
\newcommand{\fig}[1]{figure $#1$}
\newcommand{\sym}{symétrique}
\newcommand{\syms}{symétriques}
\newcommand{\asym}{axe de symétrie}
\newcommand{\asyms}{axes de symétrie}
\newcommand{\seg}[1]{$[#1]$}
\newcommand{\monAngle}[1]{$\widehat{#1}$}
\newcommand{\bissec}{bissectrice}
\newcommand{\mediat}{médiatrice}
\newcommand{\ddte}[1]{$[#1)$}

%Figures
\newcommand{\para}{parallélogramme}
\newcommand{\paras}{parallélogrammes}
\newcommand{\myquad}{quadrilatère}
\newcommand{\myquads}{quadrilatères}
\newcommand{\co}{côtés opposés}
\newcommand{\diag}{diagonale}
\newcommand{\diags}{diagonales}
\newcommand{\supp}{supplémentaires}
\newcommand{\car}{carré}
\newcommand{\cars}{carrés}
\newcommand{\rect}{rectangle}
\newcommand{\rects}{rectangles}
\newcommand{\los}{losange}
\newcommand{\loss}{losanges}


\newcommand{\homo}{homothétie}
\newcommand{\homos}{homothéties}




%----------------------------------------------------
% Environnements de cours
%------------------------------------------------------



%\usepackage{../../../../pas-math}
\usepackage{../../../moncours_beamer}





\graphicspath{{../img/}}

\title{Chapitre 3 : Description de la matière}
\author{O. FINOT}\institute{Collège S$^t$ Bernard}


\AtBeginSection[]
{
	\begin{frame}
		\frametitle{}
		\tableofcontents[currentsection, hideallsubsections]
	\end{frame} 

}


%\AtBeginSubsection[]
%{
%	\begin{frame}
%		\frametitle{Sommaire}
%		\tableofcontents[currentsection, currentsubsection]
%	\end{frame} 
%}

\begin{document}

\begin{frame}
  \titlepage 
\end{frame}


\section{La diversité de la matière}

\begin{frame}



%\vspace*{1cm}

\begin{alertblock}{}
	La matière est \kw{diverse} : elle peut être vivante ou inerte, naturelle ou fabriquée. Il y a des métaux, des verres, des plastiques, de la \kw{matière minérale}, de la \kw{matière organique} sous différentes formes ...
\end{alertblock}

%\vspace*{1cm}

\begin{alertblock}<2>{}
	Un échantillon de matière peut exister sous \kw{trois états différents} :
	\begin{itemize}
		\item solide ;
		\item liquide ;
		\item gazeux.
	\end{itemize}
\end{alertblock}

\end{frame}


\begin{frame}

\begin{block}{\Large{Vocabulaire}}
	\begin{itemize}
		\item \kw{Matière minérale} : \pause  l'eau, l'air et les roches sont des matières minérales.\pause
		
		\item \kw{Matière organique} : \pause matière produite par les êtres vivants.
	\end{itemize}
\end{block}
		
\end{frame}

\section{Les propriétés de la matière}

\begin{frame}


\begin{block}{}
	\begin{itemize}
		\item La distinction entre différents matériaux peut se faire à partir de leur propriétés (densité, élasticité, conductivité électrique ou thermique, magnétisme, solubilité dans l'eau ...  )
		
		\item Un autre moyen de caractériser un échantillon de matière est de mesurer \kw{sa masse}. 
	\end{itemize}
	
		
\end{block}

\end{frame}

\section{Les mélanges}

\begin{frame}
	\begin{block}{}
		\begin{itemize}
			\item La matière qui nous entoure peut être le résultat d'un mélange de plusieurs constituants solides, liquides ou gazeux.\pause
			
			\item Réaliser des mélanges peut provoquer des transformations de la matière (dissolution, transformation chimique).\pause
			
			\item Différentes techniques existent pour séparer les constituants d'un mélange : décantation, évaporation, filtration ...
		\end{itemize}
	
	
	\end{block}

\end{frame}

\begin{frame}

\begin{block}{\Large{Vocabulaire}}
	\begin{itemize}
		\item \kw{Décantation} : \pause  technique qui consiste à laisser reposer un mélange en attendant que les constituants se séparent spontanément.
		
		\item \kw{\'Evaporation} : \pause passage progressif d'une substance de l'état liquide à l'état gazeux.
		
		\item \kw{Filtration} : \pause séparation des constituants solides et liquides d'un mélange grâce à un filtre.
	\end{itemize}
\end{block}

\end{frame}

\end{document}


\begin{frame}


\end{frame}