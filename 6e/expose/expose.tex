\documentclass[french]{article}
\usepackage[T1]{fontenc}
\usepackage[utf8]{inputenc}
\usepackage{lmodern}
\usepackage[a4paper]{geometry}
\usepackage{babel}

\newcommand{\kw}[1]{\textbf{\underline{#1}}}

\title{Exposé Sixième : l'espace}


\begin{document}

\maketitle

\section{Forme et choix du sujet}

Exposé individuel à l'écrit sur le thème de l'espace. Le sujet de cet exposé sera un \kw{corps céleste} ou un \kw{véhicule spatial} du choix de l'élève.

\section{Contenu}

L'exposé devra contenir les informations suivantes :

\subsection{Corps céleste}
	Si le sujet choisi est un corps céleste, l'exposé devra contenir les informations suivantes :
	\begin{itemize}
		\item son nom;
		\item nature;
		\item sa position dans l'espace;
		\item la distance par rapport à la Terre
		\item ses dimensions;
		\item ses mouvements.
		
	
	\end{itemize}

\subsection{Véhicule spatial}
Si le sujet choisi est un véhicule spatial, l'exposé devra contenir les informations suivantes :
	\begin{itemize}
		\item son nom;
		\item nature;
		\item son but / sa mission;
		\item sa date de lancement;
		\item la distance qu'il a parcouru depuis son lancement;
		\item ses dimensions.
		
		
	\end{itemize}

\subsection{Sources}

Quel que soit le type de sujet choisi l'exposé devra également contenir la liste des sources qui ont été utilisées.
\end{document}
