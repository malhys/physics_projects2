\documentclass[xcolor={dvipsnames}]{beamer}
%\usepackage[utf8]{inputenc}
%\usetheme{Malmoe}
\usetheme{CambridgeUS}
%\usecolortheme{albatross}

%-------------------------------------------------------------------------------
%          -Packages nécessaires pour écrire en Français et en UTF8-
%-------------------------------------------------------------------------------
\usepackage[utf8]{inputenc}
\usepackage[french]{babel}
\usepackage[T1]{fontenc}
\usepackage{lmodern}
\usepackage{textcomp}

%-------------------------------------------------------------------------------

%-------------------------------------------------------------------------------
%                          -Outils de mise en forme-
%-------------------------------------------------------------------------------
\usepackage{hyperref}
\hypersetup{pdfstartview=XYZ}
\usepackage{enumerate}
\usepackage{graphicx}
%\usepackage{multicol}
%\usepackage{tabularx}

%\usepackage{anysize} %%pour pouvoir mettre les marges qu'on veut
%\marginsize{2.5cm}{2.5cm}{2.5cm}{2.5cm}

\usepackage{indentfirst} %%pour que les premier paragraphes soient aussi indentés
\usepackage{verbatim}
%\usepackage[table]{xcolor}  
%\usepackage{multirow}
\usepackage{ulem}
%-------------------------------------------------------------------------------


%-------------------------------------------------------------------------------
%                  -Nécessaires pour écrire des mathématiques-
%-------------------------------------------------------------------------------
\usepackage{amsfonts}
\usepackage{amssymb}
\usepackage{amsmath}
\usepackage{amsthm}
\usepackage{tikz}
\usepackage{xlop}
\usepackage[output-decimal-marker={,}]{siunitx}
%-------------------------------------------------------------------------------

%-------------------------------------------------------------------------------
%                  -Nécessaires pour écrire des formules chimiquess-
%-------------------------------------------------------------------------------

\usepackage[version=4]{mhchem}

%-------------------------------------------------------------------------------
%                    - Mise en forme 
%-------------------------------------------------------------------------------

\newcommand{\bu}[1]{\underline{\textbf{#1}}}


\usepackage{ifthen}


\newcommand{\ifTrue}[2]{\ifthenelse{\equal{#1}{true}}{#2}{$\qquad \qquad$}}

\newcommand{\kword}[1]{\textcolor{red}{\underline{#1}}}


%-------------------------------------------------------------------------------



%-------------------------------------------------------------------------------
%                    - Racourcis d'écriture -
%-------------------------------------------------------------------------------

% Angles orientés (couples de vecteurs)
\newcommand{\aopp}[2]{(\vec{#1}, \vec{#2})} %Les deuc vecteurs sont positifs
\newcommand{\aopn}[2]{(\vec{#1}, -\vec{#2})} %Le second vecteur est négatif
\newcommand{\aonp}[2]{(-\vec{#1}, \vec{#2})} %Le premier vecteur est négatif
\newcommand{\aonn}[2]{(-\vec{#1}, -\vec{#2})} %Les deux vecteurs sont négatifs

%Ensembles mathématiques
\newcommand{\naturels}{\mathbb{N}} %Nombres naturels
\newcommand{\relatifs}{\mathbb{Z}} %Nombres relatifs
\newcommand{\rationnels}{\mathbb{Q}} %Nombres rationnels
\newcommand{\reels}{\mathbb{R}} %Nombres réels
\newcommand{\complexes}{\mathbb{C}} %Nombres complexes


%Intégration des parenthèses aux cosinus
\newcommand{\cosP}[1]{\cos\left(#1\right)}
\newcommand{\sinP}[1]{\sin\left(#1\right)}

%Fractions
\newcommand{\myfrac}[2]{{\LARGE $\frac{#1}{#2}$}}

%Vocabulaire courrant
\newcommand{\cad}{c'est-à-dire}

%Droites
\newcommand{\dte}[1]{$(#1)$}
\newcommand{\fig}[1]{figure $#1$}
\newcommand{\sym}{symétrique}
\newcommand{\syms}{symétriques}
\newcommand{\asym}{axe de symétrie}
\newcommand{\asyms}{axes de symétrie}
\newcommand{\seg}[1]{$[#1]$}
\newcommand{\monAngle}[1]{$\widehat{#1}$}
\newcommand{\bissec}{bissectrice}
\newcommand{\mediat}{médiatrice}
\newcommand{\ddte}[1]{$[#1)$}

%Figures
\newcommand{\para}{parallélogramme}
\newcommand{\paras}{parallélogrammes}
\newcommand{\myquad}{quadrilatère}
\newcommand{\myquads}{quadrilatères}
\newcommand{\co}{côtés opposés}
\newcommand{\diag}{diagonale}
\newcommand{\diags}{diagonales}
\newcommand{\supp}{supplémentaires}
\newcommand{\car}{carré}
\newcommand{\cars}{carrés}
\newcommand{\rect}{rectangle}
\newcommand{\rects}{rectangles}
\newcommand{\los}{losange}
\newcommand{\loss}{losanges}


\newcommand{\homo}{homothétie}
\newcommand{\homos}{homothéties}




%----------------------------------------------------
% Environnements de cours
%------------------------------------------------------



%\usepackage{../../../../pas-math}
\usepackage{../../../moncours_beamer}





\graphicspath{{../img/}}

\title{Chapitre 4 : Description du mouvement}
\author{O. FINOT}\institute{Collège S$^t$ Bernard}


\AtBeginSection[]
{
	\begin{frame}
		\frametitle{}
		\tableofcontents[currentsection, hideallsubsections]
	\end{frame} 

}


%\AtBeginSubsection[]
%{
%	\begin{frame}
%		\frametitle{Sommaire}
%		\tableofcontents[currentsection, currentsubsection]
%	\end{frame} 
%}

\begin{document}

\begin{frame}
  \titlepage 
\end{frame}


\section{Description du mouvement}

\begin{frame}
	\begin{mybilan}
	\begin{itemize}
		\item Un \kw{solide} a une \kw{forme propre} qui ne change pas, on peut le saisir.
		\item Un \kw{liquide} prend la \kw{forme du récipient} qui le contient.
		\item La surface d'un liquide en contact avec l'air est sa \kw{surface libre}.
		\item Au repos, cette surface libre est \kw{plane et horizontale}.
	\end{itemize}	   
\end{mybilan}
\end{frame}

\section{Trajectoire d'un objet en mouvement }

\begin{frame}
	\begin{mybilan}
	Pour décrire la vitesse d'un objet en mouvement, on utilise trois caractéristiques :
	\begin{itemize}
		\item la \kw{direction} (horizontale, verticale ou oblique), tangente à la trajectoire;
		
		\item le \kw{sens}, celui du mouvement (vers la gauche, vars la droite, vers le haut etc.);
		
		\item la \kw{valeur} exprimée m/s (ou km/h ou autre).
		
		Si le mouvement est uniforme, la relation \kw{$ v = \dfrac{d}{\Delta t} $}, permet de relier la vitesse de l'objet, la distance parcourue et la durée du parcours avec :
		\begin{itemize}
			\item d : distance parcourue en mètre (m)
			\item $\Delta t$ :durée du trajet en seconde (s)
			\item v : vitesse en mètre par seconde (m/s).
		\end{itemize}
	\end{itemize}



\end{mybilan}


\end{frame}

\section{Valeur de la vitesse d'un objet en mouvement}

\begin{frame}



%\vspace*{1cm}

\begin{alertblock}{}
	\begin{itemize}
		\item Il est possible de calculer la vitesse d'un objet à condition d'avoir mesuré la \kw{distance parcourue} (en kilomètres ou en mètres) et la \kw{durée du parcours} (en heures ou en secondes). 
		
		\item Les unités de mesure de la valeur de la vitesse sont généralement le \kw{kilomètre par heure} (km/h) ou le \kw{mètre par seconde} (m/s).
	\end{itemize}
	
	
	
\end{alertblock}

\begin{alertblock}<2->{}
	
	\begin{align*}
		%vitesse = distance \;  parcourue \div temps \; du  \; parcours
		vitesse\; d'un\; objet = \dfrac{distance \;  parcourue \; par \; l'objet}{temps \; du  \; parcours}
	\end{align*}
\end{alertblock}

\end{frame}
%\vspace*{1cm}


\begin{frame}
	\begin{block}{Méthode : convertir une vitesse}
		
		\begin{itemize}
			\item Il y a \num{3600} secondes dans une heure et \num{1000} mètres dans un kilomètre.	\pause
			
			\item Pour des $m/s$ en $m/h$, on multiplie par \num{3600}, puis pour passer des $m/h$ aux $km/h$ on divise par \num{1000}.\pause
			
			\item Donc pour convertir une vitesse exprimée en $m/s$ en $km/h$ il faut la multiplier par  $\num{3.6}$. 
			
			\item Pour convertir des $km/h$ en $m/s$, il suffit de diviser par $\num{3.6}$.
		\end{itemize}
	\end{block}
\end{frame}


\begin{frame}
	\begin{myex}
		On converti 9 $m/s$ en $km/h$ :
		
		\begin{equation*}
		9 \times \num{3600} = \num{32400}
		\end{equation*}
		
		9 $m/s$ est égal à \num{32400} $m/h$.
		
		\begin{equation*}
		\num{32400} \div \num{1000} = \num{32.4}
		\end{equation*}
		
		\num{32400} $m/h$ est égal à \num{32.4} $km/h$. On a donc 9 $m/s$ = \num{32.4} $km/h$.
		
	\end{myex}
\end{frame}
\begin{frame}
\begin{alertblock}{}
	La vitesse d'un objet peut rester constante ou changer au cours du temps.
	Si elle change, il y a soit \kw{accélération} soit \kw{décélération} de l'objet. 
	
\end{alertblock}


\begin{block}<2->{\Large{Vocabulaire}}
	\begin{itemize}
		\item \kw{Accélération} : \pause  augmentation de la valeur de la vitesse au cours du temps.\pause
		
		\item \kw{Décélération} : \pause diminution de la valeur de la vitesse au cours du temps.
	\end{itemize}
\end{block}
		
\end{frame}

\end{document}


\begin{frame}


\end{frame}