


\section{Mouvement}

\begin{questions}
	\question Quels sont les différents types de trajectoires pour décrire le mouvement d'un objet
	\fillwithdottedlines{2cm}
	
	\question A la fête foraine, Jules et Sophie sont sur un manège en fonctionnement, ils ne bougent pas l'un par rapport à l'autre.
	\begin{parts}
		\part Quel est le mouvement de Sophie par rapport au sol ?
		\fillwithdottedlines{2cm}
		
		\part Quel est son mouvement par rapport à Jules ?
		\fillwithdottedlines{2cm}
	\end{parts}
	
	
\end{questions}



\section{Calculs de vitesse}

\begin{questions}
	
	\question Dans la cour pendant la récréation, Eve a couru pendant 20 secondes, elle a parcouru 60 mètres. A quelle vitesse a-t-elle couru ?
	\fillwithdottedlines{3cm}
	
	
	\question Dans sa voiture Pierre fait un trajet de 220 km, il lui a fallu 2 heures. A quelle vitesse a-t-il roulé ?
	\fillwithdottedlines{3cm}
		
\end{questions}

\newpage

\section{Conversion}

Convertir :
\begin{questions}
	\question 3 $m/s$ en $km/h$ :
	\fillwithdottedlines{2cm}
	
	\question 15 $m/s$ en $km/h$ :
	\fillwithdottedlines{2cm}
	
	\question 36 $km/h$ en $m/s$ :
	\fillwithdottedlines{2cm}
	
	\question 180 $km/h$ en $m/s$ :
	\fillwithdottedlines{2cm}
\end{questions}