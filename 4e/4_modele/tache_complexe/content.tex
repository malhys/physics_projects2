	\begin{center}
		{\LARGE \textbf{Proposition de correction de l'exercice 28 p 32}}
	\end{center}
	
	Le but de l'exercice est de comprendre comment un ballon jouet fourni dans un magazine peut s'envoler.
	Le \textbf{doc. 2} présente un ballon solaire avec son enveloppe fabriquée dans une matière plastique noire capable d'absorber les rayonnements de l'énergie solaire (\textbf{doc. 3}).
	D'après le \textbf{doc. 4} plus la température de l'air augmente, et plus sa masse volumique diminue.
	Le mode d'emploi du ballon présenté \textbf{doc. 1} indique qu'il faut tout d'abord le remplir d'air puis ensuite le laisser au soleil.
	L'énergie solaire captée par le ballon noir va alors réchauffer l'air enfermé à l'intérieur. 
	L'air à l'intérieur du ballon, plus chaud que l'air ambiant aura une masse volumique plus faible.
	Dans ce cas, le ballon sera plus léger que l'air ambiant et pourra s'envoler.
	
%	D'après le \textbf{doc. 2}, la combustion du bois produit du dioxyde d e carbone ($CO_2$) qui est ensuite consommé par les arbres. Pour que l'utilisation du bois de chauffage ait un bilan carbone neutre, il faut que les arbres consomment autant de molécules de $CO_2$ que la combustion du bois de chauffage.	
%	D'après le \textbf{doc. 3}, le bois est essentiellement composé d'un dérivé du glucose ; la combustion d'une molécule de glucose produit 6 molécules de $CO_2$. 
%	D'après le \textbf{doc. 1}, la photosynthèse permet aux végétaux chlorophylliens et donc aux arbres de produire du glucose à partir de $CO_2$. Produire une molécule de glucose consomme 6 molécules de $CO_2$.
%	
%	Si un nouvel arbre est planté pour chaque arbre coupé, alors pour chaque molécule de glucose consommée, le $CO_2$ produit sera à son tour consommé pour produire une nouvelle molécule de glucose. Donc, l'utilisation de bois de chauffage produit dans ces conditions affiche un bilan carbone neutre.