\section{Comment caractériser un mouvement ?}

\begin{questions}
	\question Le mouvement du tunnelier est \underline{rectiligne} et \underline{uniforme}.
	
	\question Lors du fonctionnement du tunnelier, la roue coupante a une trajectoire \underline{circulaire}.
	
	\question Lors d'un cycle de fonctionnement du tunnelier la roue :
	\begin{enumerate}
		\item commence par démarrer, donc sa vitesse augmente ;
		\item puis elle se stabilise à vitesse constante;
		\item enfin elle ralenti pour s'arrêter.
	\end{enumerate} 

	\question La roue coupante du tunnelier a donc un mouvement :
	\begin{enumerate}
		\item d'abord circulaire accéléré;
		\item ensuite circulaire uniforme;
		\item enfin circulaire ralenti;
	\end{enumerate} 
\end{questions}