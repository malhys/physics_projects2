\begin{mybilan}
	Pour décrire la vitesse d'un objet en mouvement, on utilise trois caractéristiques :
	\begin{itemize}
		\item la \kw{direction} (horizontale, verticale ou oblique), tangente à la trajectoire;
		
		\item le \kw{sens}, celui du mouvement (vers la gauche, vars la droite, vers le haut etc.);
		
		\item la \kw{valeur} exprimée m/s (ou km/h ou autre).
		
		Si le mouvement est uniforme, la relation \kw{$ v = \dfrac{d}{\Delta t} $}, permet de relier la vitesse de l'objet, la distance parcourue et la durée du parcours avec :
		\begin{itemize}
			\item d : distance parcourue en mètre (m)
			\item $\Delta t$ :durée du trajet en seconde (s)
			\item v : vitesse en mètre par seconde (m/s).
		\end{itemize}
	\end{itemize}



\end{mybilan}

