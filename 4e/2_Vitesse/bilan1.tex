\begin{mybilan}
	\begin{itemize}
		
		\item La trajectoire d'un objet en mouvement est formée par l'ensemble des positions prises par l'objet au cours du mouvement.
		
		\begin{itemize}
			\item Si la trajectoire décrit une \kw{ligne droite} le mouvement est \kw{rectiligne}.
			\item Si elle décrit \kw{un cercle ou un arc de cercle}, le mouvement est \kw{circulaire}.
			\item Sinon il est \kw{curviligne}.
		\end{itemize}
	
		\item Un mouvement est \kw{uniforme} si la valeur de la vitesse est constante ; \kw{accéléré} si si cette valeur augmente et \kw{ralenti} si elle diminue. 
		
	\end{itemize}
\end{mybilan}