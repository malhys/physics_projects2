\section{Proxima du centaure (4 points)}

Proxima du centaure est l'étoile la plus proche de notre système solaire. Sa lumière nous parvient après avoir parcouru \num{39700} milliards de kilomètres à la vitesse de \num{300000} km/s.

\begin{questions}
	\question[1] Quelle est la distance parcourue par la vitesse en un an ?
	%\fillwithdottedlines{3cm}
	
	
	\question[1] Quelle est la durée, en année, du parcours de la lumière issue de cette étoile jusqu'à nous ?
	%\fillwithdottedlines{4cm}
	
	\question[2] Quelle serait la durée, en année, de ce parcours pour une personne marchant à 5km/h ?
	%\fillwithdottedlines{4cm}
\end{questions}