\documentclass[a4paper,11pt]{exam}
%\printanswers % pour imprimer les réponses (corrigé)
\noprintanswers % Pour ne pas imprimer les réponses (énoncé)
\addpoints % Pour compter les points
% \noaddpoints % pour ne pas compter les points
%\qformat{\textbf{\thequestion ) } }
%\qformat{\textbf{\thequestion )}} % Pour définir le style des questions (facultatif)
\usepackage{color} % définit une nouvelle couleur
\shadedsolutions % définit le style des réponses
% \framedsolutions % définit le style des réponses
\definecolor{SolutionColor}{rgb}{0.8,0.9,1} % bleu ciel
\renewcommand{\solutiontitle}{\noindent\textbf{Solution:}\par\noindent} % Définit le titre des solutions




\makeatletter

\def\maketitle{{\centering%
	\par{\huge\textbf{\@title}}%
	\par{\@date}%
	\par}}


\renewcommand{\thesubsection}{\Alph{subsection}.}   

\makeatother

\lhead{NOM Pr\'enom :}
\rhead{\textbf{Les r\'eponses doivent \^etre justifi\'ees et r\'edig\'ees}}
\cfoot{\thepage / \pageref{LastPage}}


%\usepackage{../../pas-math}
%\usepackage{../../moncours}


%\usepackage{pas-cours}
%-------------------------------------------------------------------------------
%          -Packages nécessaires pour écrire en Français et en UTF8-
%-------------------------------------------------------------------------------
\usepackage[utf8]{inputenc}
\usepackage[frenchb]{babel}
%\usepackage{numprint}
\usepackage[T1]{fontenc}
%\usepackage{lmodern}
\usepackage{textcomp}
\usepackage[french, boxed]{algorithm2e}
\usepackage{hyperref}


%-------------------------------------------------------------------------------

%-------------------------------------------------------------------------------
%                          -Outils de mise en forme-
%-------------------------------------------------------------------------------
\usepackage{hyperref}
\hypersetup{pdfstartview=XYZ}
%\usepackage{enumerate}
\usepackage{graphicx}
\usepackage{multicol}
\usepackage{tabularx}
\usepackage{multirow}
\usepackage{color}
\usepackage{eurosym}


\usepackage{anysize} %%pour pouvoir mettre les marges qu'on veut
%\marginsize{2.5cm}{2.5cm}{2.5cm}{2.5cm}

\usepackage{indentfirst} %%pour que les premier paragraphes soient aussi indentés
\usepackage{verbatim}
\usepackage{enumitem}
\usepackage{booktabs}
\usepackage[usenames,dvipsnames,svgnames,table]{xcolor}

\usepackage{variations}

%-------------------------------------------------------------------------------


%-------------------------------------------------------------------------------
%                  -Nécessaires pour écrire des mathématiques-
%-------------------------------------------------------------------------------
\usepackage{amsfonts}
\usepackage{amssymb}
\usepackage{amsmath}
\usepackage{amsthm}
\usepackage{tikz}
\usepackage{xlop}
\usepackage[output-decimal-marker={,}]{siunitx}
%-------------------------------------------------------------------------------

%-------------------------------------------------------------------------------
%                  -Nécessaires pour écrire des formules chimiquess-
%-------------------------------------------------------------------------------

\usepackage[version=4]{mhchem}

%-------------------------------------------------------------------------------
% Pour pouvoir exploiter les fichiers directement dans beamer
\newcommand{\pause}{\ }
%-------------------------------------------------------------------------------
%                    - Mise en forme avancée
%-------------------------------------------------------------------------------

\usepackage{ifthen}
\usepackage{ifmtarg}


\newcommand{\ifTrue}[2]{\ifthenelse{\equal{#1}{true}}{#2}{$\qquad \qquad$}}

%\newcommand{\kword}[1]{\textcolor{red}{\underline{#1}}}
%-------------------------------------------------------------------------------

%-------------------------------------------------------------------------------
%                     -Mise en forme d'exercices-
%-------------------------------------------------------------------------------
%\newtheoremstyle{exostyle}
%{\topsep}% espace avant
%{\topsep}% espace apres
%{}% Police utilisee par le style de thm
%{}% Indentation (vide = aucune, \parindent = indentation paragraphe)
%{\bfseries}% Police du titre de thm
%{.}% Signe de ponctuation apres le titre du thm
%{ }% Espace apres le titre du thm (\newline = linebreak)
%{\thmname{#1}\thmnumber{ #2}\thmnote{. \normalfont{\textit{#3}}}}% composants du titre du thm : \thmname = nom du thm, \thmnumber = numéro du thm, \thmnote = sous-titre du thm

%\theoremstyle{exostyle}
%\newtheorem{exercice}{Exercice}
%
%\newenvironment{questions}{
%\begin{enumerate}[\hspace{12pt}\bfseries\itshape a.]}{\end{enumerate}
%} %mettre un 1 à la place du a si on veut des numéros au lieu de lettres pour les questions 
%-------------------------------------------------------------------------------

%-------------------------------------------------------------------------------
%                    - Mise en forme de tableaux -
%-------------------------------------------------------------------------------

\renewcommand{\arraystretch}{1.7}

\setlength{\tabcolsep}{1.2cm}

%-------------------------------------------------------------------------------



%-------------------------------------------------------------------------------
%                    - Racourcis d'écriture -
%-------------------------------------------------------------------------------
%Droites
\newcommand{\dte}[1]{$(#1)$}
\newcommand{\fig}[1]{figure $#1$}
\newcommand{\sym}{symétrique}
\newcommand{\syms}{symétriques}
\newcommand{\asym}{axe de symétrie}
\newcommand{\asyms}{axes de symétrie}
\newcommand{\seg}[1]{$[#1]$}
\newcommand{\monAngle}[1]{$\widehat{#1}$}
\newcommand{\bissec}{bissectrice}
\newcommand{\mediat}{médiatrice}
\newcommand{\ddte}[1]{$[#1)$}


% Angles orientés (couples de vecteurs)
\newcommand{\aopp}[2]{(\vec{#1}, \vec{#2})} %Les deuc vecteurs sont positifs
\newcommand{\aopn}[2]{(\vec{#1}, -\vec{#2})} %Le second vecteur est négatif
\newcommand{\aonp}[2]{(-\vec{#1}, \vec{#2})} %Le premier vecteur est négatif
\newcommand{\aonn}[2]{(-\vec{#1}, -\vec{#2})} %Les deux vecteurs sont négatifs

%Ensembles mathématiques
\newcommand{\naturels}{\mathbb{N}} %Nombres naturels
\newcommand{\relatifs}{\mathbb{Z}} %Nombres relatifs
\newcommand{\rationnels}{\mathbb{Q}} %Nombres rationnels
\newcommand{\reels}{\mathbb{R}} %Nombres réels
\newcommand{\complexes}{\mathbb{C}} %Nombres complexes


%Intégration des parenthèses aux cosinus
\newcommand{\cosP}[1]{\cos\left(#1\right)}
\newcommand{\sinP}[1]{\sin\left(#1\right)}


%Probas stats
\newcommand{\stat}{statistique}
\newcommand{\stats}{statistiques}


\newcommand{\homo}{homothétie}
\newcommand{\homos}{homothéties}


\newcommand{\mycoord}[3]{(\textcolor{red}{\num{#1}} ; \textcolor{Green}{\num{#2}} ; \textcolor{blue}{\num{#3}})}
%-------------------------------------------------------------------------------

%-------------------------------------------------------------------------------
%                    - Mise en page -
%-------------------------------------------------------------------------------

\newcommand{\twoCol}[1]{\begin{multicols}{2}#1\end{multicols}}


\setenumerate[1]{font=\bfseries,label=\textit{\alph*})}
\setenumerate[2]{font=\bfseries,label=\arabic*)}


%-------------------------------------------------------------------------------
%                    - Elements cours -
%-------------------------------------------------------------------------------

%Correction d'exercice
\newcommand{\exoSec}[2]{\subsection*{Exercice #1 page #2}}
%-------------------------------------------------------------------------------
%                    - raccourcis d'écriture -
%-------------------------------------------------------------------------------

%Mise en évidence de termes clés
\newcommand{\mykw}[1]{\textcolor{red}{\underline{\textbf{#1}}}}

%Exercices
\newcommand{\exo}[2]{exercice #1 page #2}
\newcommand{\Exo}[2]{Exercice #1 page #2}

\renewcommand{\pause}{\ }


%\usepackage{fullpage}
\author{\ }
\date{18 Décembre 2018}
\title{Sciences Physiques : DS n° 2}


\begin{document}
%	\usepackage{fancyhdr}
%	
%	\pagestyle{fancy}
%	\fancyhf{}
	%\rhead{Share\LaTeX}


	\maketitle


\begin{small}
	\begin{center}
		\begin{tabular}{|@{\ }l@{}|@{\ }c@{\ }|}
			\hline
			\textbf{Compétence} & \textbf{Maitrise} \\
			\hline
			 Utiliser la relation liant vitesse, distance et durée dans le cas d’un mouvement uniforme \ \ &  \ \ \ \\
			\hline
			Vitesse : direction, sens et valeur. &  \\
			\hline			
			Mouvements uniformes et mvts dont la vitesse varie au cours du temps en direction ou en valeur.  &  \\
			\hline
			
		\end{tabular}
	\end{center}
\end{small}	
\vspace*{-0.5cm}	



%\section{Orbite de la Terre (4 points)}

La Terre tourne autour du Soleil à une distance moyenne d'environ 150 millions de kilomètres suivant une période de \num{365.25} jours. On considère que le mouvement de la Terre autour du Soleil est circulaire.

\begin{center}
	\includegraphics[scale=0.5]{terre}
\end{center}

\begin{questions}
	\question[2] Quelle est la distance parcourue par la Terre autour du Soleil pendant une année ?
	\fillwithdottedlines{3cm}
	
	\question[2] Calculer la vitesse de rotation de la Terre en km/h puis en km/s.
	\fillwithdottedlines{3cm}
\end{questions}


\section{Valeur, direction et sens (3 points)}

On a représenté ci-dessous, les vitesses de 4 objets (A, B, C et D) à un moment précis.

\begin{center}
	\includegraphics[scale=0.4]{exemples}
\end{center}

\begin{questions}
	\question \'A l'instant représenté, quels objets ont :
	\begin{parts}
		\part[1] la même direction ?
		%\fillwithdottedlines{1.5cm}
		
		\part[1] le même sens de déplacement ?
		%\fillwithdottedlines{1.5cm}
		
		\part[1] la même valeur ?
		%\fillwithdottedlines{1.5cm}
	\end{parts}

\end{questions}

\section{Représentation de la vitesse (4 points)}

\begin{questions}
	\question Représenter la vitesse d'un objet à un instant précis, dans les conditions suivantes  :
	
	\begin{parts}
		\part[2] \begin{itemize}
			\item Mouvement : horizontal de gauche à droite;
			\item Valeur de la vitesse : 25 m/s;
			\item \'Echelle choisie: 1 cm pour 10 m/s.
			
		\end{itemize}
	
%		\makeemptybox{4cm}
		
		
		\part[2] \begin{itemize}
			\item Mouvement : chute verticale d'un objet;
			\item Valeur de la vitesse : 10 m/s;
			\item \'Echelle choisie: 1 cm pour 5 m/s.
		\end{itemize}
			
		
%		\makeemptybox{4cm}
	\end{parts}

%	\question[2] \'A l'aide d'un logiciel de traitement de vidéos, on peut repérer les positions successives prises par un point d'une grande roue lors de son mouvement.
%	
%	Sans tenir compte sa valeur, représenter la vitesse aux positions 3, 12, 22 et 33.
%	
%	\begin{center}
%		\includegraphics[scale=0.35]{roue}
%	\end{center}
\end{questions}

\section{Proxima du centaure (4 points)}

Proxima du centaure est l'étoile la plus proche de notre système solaire. Sa lumière nous parvient après avoir parcouru \num{39700} milliards de kilomètres à la vitesse de \num{300000} km/s.

\begin{questions}
	\question[1] Quelle est la distance parcourue par la vitesse en un an ?
	\fillwithdottedlines{3cm}
	
	
	\question[1] Quelle est la durée, en année, du parcours de la lumière issue de cette étoile jusqu'à nous ?
	\fillwithdottedlines{4cm}
	
	\question[2] Quelle serait la durée, en année, de ce parcours pour une personne marchant à 5km/h ?
	\fillwithdottedlines{4cm}
\end{questions}
\section{Le lièvre et la tortue (14 points)}

\begin{center}
	\includegraphics[scale=1.5]{lievre}
\end{center}

<<Rien ne sert de courir ; il faut partir à point>> est une maxime tirée de la fable <<le lièvre et la tortue>> de Jean de la Fontaine (1621 - 1695).\\


Après avoir fait la sieste sous un arbre à $\num{40.0}  m$  de la ligne d'arrivée, le lièvre se réveille et aperçoit la tortue qui le précède d'une distance $d = \num{39.4} m$. Elle file vers le succès dans cette ligne droite avec une vitesse de valeur constante $v_{tortue} = \num{0.2} m/s $.

Le lièvre se met alors à courir en accélérant jusqu'à atteindre une vitesse de valeur $v_{lievre} = \num{18.0} m/s$ et il s'y maintient.
\begin{questions}
	\question[4] Caractériser le mouvement et la vitesse
	
	\begin{parts}
		\part[1] Comment qualifie-t-on le mouvement de la tortue ?
		\fillwithdottedlines{2cm}
		\part[1] Identifier et nommer les deux phases du  mouvement du lièvre.
		\fillwithdottedlines{2cm}
		\part[2]Donner les trois caractéristiques de la vitesse de la tortue.
		\fillwithdottedlines{2cm}
	\end{parts}


	\question[2] %Utiliser la formule de la vitesse
	
	\begin{parts}
		\part[1] Combien de temps faut-il à la tortue pour atteindre la ligne d'arrivée ?
		\fillwithdottedlines{2cm}
		
		\part[1] Pendant cette durée, quelle distance maximale $d_{lievre}$ parcourrait le lièvre à sa vitesse maximale ?
		\fillwithdottedlines{2cm}
	\end{parts}

	\question[4] Lors de la phase d'accélération, on peut calculer la distance qui sépare le lièvre de l'arbre avec la formule suivante ($t$ est le temps que dure la phase d'accélération):
	
	\begin{equation*}
		d_{lievre} = \num{4.5} \times t^2
	\end{equation*}

	\begin{parts}
		\part[2] En considérant que cette première phase ne dure que 2 secondes ; à quelle distance de l'arbre se trouve-t-il ?
		\fillwithdottedlines{3cm}
		
		\part[2] Montrer alors qu'il a perdu la course.
		\fillwithdottedlines{3cm}
	\end{parts}

	\question[4] Une coccinelle, qui s'était endormie au bout de l'aiguille du chronomètre, fut entrainée dans son mouvement.
	
	\begin{parts}
		\part[1] Décrire la trajectoire de la coccinelle.
		\fillwithdottedlines{2cm}
		
		\part[2] Calculer sa vitesse en cm/s, puis en m/s.
		\fillwithdottedlines{3cm}
		
		\part[1] Tracer la flèche vitesse de la coccinelle en choisissant et précisant une échelle adaptée.
		\makeemptybox{8cm}
		
		
		
	\end{parts}
\end{questions}


%\section{Distance d'arrêt (8 points)}

Manon roule en scooter à 30 km/h. Elle est attentive. Soudain elle voit un enfant surgir imprudemment sur la route à 40 m d'elle.

\begin{multicols}{2}
	\includegraphics[scale=0.3]{dist_doc1}
	
	\includegraphics[scale=0.9]{dist_doc3}
	
	
	\includegraphics[scale=0.3]{dist_doc2}		
\end{multicols}

\begin{questions}
	\question[8] En utilisant les documents ci-dessus, expliquer, en détaillant le raisonnement si Manon peut éviter l'accident.  Détailler les calculs et comment les documents sont utilisés.
	
	\fillwithdottedlines{12cm}
\end{questions}






%\section{Profils de course (4 points)}

Après un cross de 4km, \'Ethan et Amine obtiennent le <<profil>> de leur course grâce à leur montre connectées : la distance parcourue s'affiche en fonction du temps.

\begin{center}
	\includegraphics[scale=0.5]{cross}
\end{center}

\begin{questions}
	\question[1] \'Ethan et Amine ont-ils terminé la course ?
	\fillwithdottedlines{3cm}
	
	\question[1] Combien de temps ont-ils couru ?
	\fillwithdottedlines{3cm}
	
	\question[2] Lequel des deux s'est arrêté ? Expliquer la réponse.
	\fillwithdottedlines{4cm}
\end{questions}





\ \label{LastPage}

\end{document}