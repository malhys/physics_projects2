\section{Air en plongée}

Au cours d'une plongée sous-marine, Emma et Juliette utilisent chacune la même bouteille de volume 12 l.
ON a introduit dans la bouteille d'Emma 30 l d'air pris à la pression atmosphérique normale et dans celle de Juliette 50 l pris à la pression atmosphérique normale.

\begin{questions}
	\question Comment peut-on introduire dans une bouteille de 12 l, un volume d'air supérieur ?
	\fillwithdottedlines{3cm}
	
	\question Dans laquelle des deux bouteilles :
	\begin{parts}
		\part la pression d'air est-elle la plus forte ? Justfier la réponse
		.
		\fillwithdottedlines{3cm}
		\part la masse d'air est-elle la plus petite ? Justfier la réponse.
		\fillwithdottedlines{3cm}
	\end{parts}

	\question L'air de la bouteille de Juliette occupe-t-il un volume supérieur, inférieur ou égal à celui de l'air de la bouteille d'Emma ? Justifier la réponse.
	
	\fillwithdottedlines{4cm}
\end{questions}