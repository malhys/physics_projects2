\documentclass[xcolor={dvipsnames}]{beamer}
%\usepackage[utf8]{inputenc}
%\usetheme{Madrid}
\usetheme{CambridgeUS}
\usecolortheme{}

%-------------------------------------------------------------------------------
%          -Packages nécessaires pour écrire en Français et en UTF8-
%-------------------------------------------------------------------------------
\usepackage[utf8]{inputenc}
\usepackage[french]{babel}
\usepackage[T1]{fontenc}
\usepackage{lmodern}
\usepackage{textcomp}

%-------------------------------------------------------------------------------

%-------------------------------------------------------------------------------
%                          -Outils de mise en forme-
%-------------------------------------------------------------------------------
\usepackage{hyperref}
\hypersetup{pdfstartview=XYZ}
\usepackage{enumerate}
\usepackage{graphicx}
%\usepackage{multicol}
%\usepackage{tabularx}

%\usepackage{anysize} %%pour pouvoir mettre les marges qu'on veut
%\marginsize{2.5cm}{2.5cm}{2.5cm}{2.5cm}

\usepackage{indentfirst} %%pour que les premier paragraphes soient aussi indentés
\usepackage{verbatim}
%\usepackage[table]{xcolor}  
%\usepackage{multirow}
\usepackage{ulem}
%-------------------------------------------------------------------------------


%-------------------------------------------------------------------------------
%                  -Nécessaires pour écrire des mathématiques-
%-------------------------------------------------------------------------------
\usepackage{amsfonts}
\usepackage{amssymb}
\usepackage{amsmath}
\usepackage{amsthm}
\usepackage{tikz}
\usepackage{xlop}
\usepackage[output-decimal-marker={,}]{siunitx}
%-------------------------------------------------------------------------------

%-------------------------------------------------------------------------------
%                  -Nécessaires pour écrire des formules chimiquess-
%-------------------------------------------------------------------------------

\usepackage[version=4]{mhchem}

%-------------------------------------------------------------------------------
%                    - Mise en forme 
%-------------------------------------------------------------------------------

\newcommand{\bu}[1]{\underline{\textbf{#1}}}


\usepackage{ifthen}


\newcommand{\ifTrue}[2]{\ifthenelse{\equal{#1}{true}}{#2}{$\qquad \qquad$}}

\newcommand{\kword}[1]{\textcolor{red}{\underline{#1}}}


%-------------------------------------------------------------------------------



%-------------------------------------------------------------------------------
%                    - Racourcis d'écriture -
%-------------------------------------------------------------------------------

% Angles orientés (couples de vecteurs)
\newcommand{\aopp}[2]{(\vec{#1}, \vec{#2})} %Les deuc vecteurs sont positifs
\newcommand{\aopn}[2]{(\vec{#1}, -\vec{#2})} %Le second vecteur est négatif
\newcommand{\aonp}[2]{(-\vec{#1}, \vec{#2})} %Le premier vecteur est négatif
\newcommand{\aonn}[2]{(-\vec{#1}, -\vec{#2})} %Les deux vecteurs sont négatifs

%Ensembles mathématiques
\newcommand{\naturels}{\mathbb{N}} %Nombres naturels
\newcommand{\relatifs}{\mathbb{Z}} %Nombres relatifs
\newcommand{\rationnels}{\mathbb{Q}} %Nombres rationnels
\newcommand{\reels}{\mathbb{R}} %Nombres réels
\newcommand{\complexes}{\mathbb{C}} %Nombres complexes


%Intégration des parenthèses aux cosinus
\newcommand{\cosP}[1]{\cos\left(#1\right)}
\newcommand{\sinP}[1]{\sin\left(#1\right)}

%Fractions
\newcommand{\myfrac}[2]{{\LARGE $\frac{#1}{#2}$}}

%Vocabulaire courrant
\newcommand{\cad}{c'est-à-dire}

%Droites
\newcommand{\dte}[1]{$(#1)$}
\newcommand{\fig}[1]{figure $#1$}
\newcommand{\sym}{symétrique}
\newcommand{\syms}{symétriques}
\newcommand{\asym}{axe de symétrie}
\newcommand{\asyms}{axes de symétrie}
\newcommand{\seg}[1]{$[#1]$}
\newcommand{\monAngle}[1]{$\widehat{#1}$}
\newcommand{\bissec}{bissectrice}
\newcommand{\mediat}{médiatrice}
\newcommand{\ddte}[1]{$[#1)$}

%Figures
\newcommand{\para}{parallélogramme}
\newcommand{\paras}{parallélogrammes}
\newcommand{\myquad}{quadrilatère}
\newcommand{\myquads}{quadrilatères}
\newcommand{\co}{côtés opposés}
\newcommand{\diag}{diagonale}
\newcommand{\diags}{diagonales}
\newcommand{\supp}{supplémentaires}
\newcommand{\car}{carré}
\newcommand{\cars}{carrés}
\newcommand{\rect}{rectangle}
\newcommand{\rects}{rectangles}
\newcommand{\los}{losange}
\newcommand{\loss}{losanges}


\newcommand{\homo}{homothétie}
\newcommand{\homos}{homothéties}




%----------------------------------------------------
% Environnements de cours
%------------------------------------------------------



%\usepackage{../../../../pas-math}
\usepackage{../../../moncours_beamer}





\graphicspath{{../img/}}
%Quelles sont les deux sortes de sources de lumière
\title{La tension électrique}
\author{O. FINOT}\institute{Collège S$^t$ Bernard}


\AtBeginSection[]
{
	\begin{frame}
		\frametitle{}
		\tableofcontents[currentsection, hideallsubsections]
	\end{frame} 

}


%\AtBeginSubsection[]
%{
%	\begin{frame}
%		\frametitle{Sommaire}
%		\tableofcontents[currentsection, currentsubsection]
%	\end{frame} 
%}

\begin{document}

\begin{frame}
  \titlepage 
\end{frame}

\section{Mesure d'une tension électrique}



\begin{frame}
	\begin{block}{Rappel}
		Un \kw{dipôle} est un composant électrique qui possède deux bornes (piles, lampes, interrupteurs, etc .).
	\end{block}
\end{frame}



\begin{frame}

\begin{alertblock}{}
	L'unité de mesure de la tension électrique est le \kw{volt} (symbole $V$). Elle se mesure avec un \kw{voltmètre}.
	Pour mesurer la tension électrique d'une pile, on relie la \kw{borne rouge} du voltmètre à la \kw{borne $[+]$} de la pile et \kw{l'autre borne} du voltmètre à la \kw{borne $[-]$}.
	
	Pour mesurer la tension aux bornes d'un dipôle placé dans un circuit, on branche un \kw{voltmètre en dérivation} entre les bornes de ce dipôle. Il existe une \kw{tension électrique} entre les bornes d'un \kw{interrupteur ouvert} placé dans un circuit.
	La tension électrique entre les bornes d'un fil de connexion est nulle.
\end{alertblock}
\end{frame}


\section{Adaptation d'un dipôle}

\begin{frame}
\begin{alertblock}{}
		La \kw{tension nominale} notée sur un appareil électrique est une indication de fonctionnement correct.
	Si la tension électrique aux bornes d'un dipôle est \kw{proche de sa tension nominale}, alors il y a \kw{adaptation} du dipôle au générateur. 
	Il y a \kw{sous-tension} si la tension est inférieure à la tension nominale. Il y a \kw{surtension} si elle est supérieure, le dipôle risque d'être détérioré.
\end{alertblock}
\end{frame}


\section{Dipôles en série et en dérivation}


\begin{frame}

	\begin{block}

	
		\begin{itemize}
			\item Dans un \kw{circuit série} :
			\begin{itemize}
				\item la valeur de la tension entre les bornes d'un dipôle ne \kw{dépend pas de sa position} dans le circuit.
				\item la valeur de la \kw{tension $U$} aux bornes du générateur est égale à la \kw{somme des valeurs des tension $U_1$ et $U_2$} entre les bornes des dipôles : c'est la \kw{loi d'additivité des tensions}.
				

				\vspace*{-0.5cm}
				
				{\Large \textcolor{red}{\begin{equation*}
					U=U_1+U_2
				\end{equation*}}}
				\vspace*{-0.5cm}
				
			\end{itemize}
			
			\item Dans un circuit comportant des \kw{dérivations}, la valeur de la \kw{tension est la même} entre les bornes des dipôles branchés en \kw{dérivation}.
			
				\vspace*{-0.5cm}

			{\Large \textcolor{red}{\begin{equation*}
				U=U_1=U_2
			\end{equation*}}}
			\vspace*{-0.5cm}
		\end{itemize}
	\end{block}
\end{frame}

%\section{Reconnaître le dioxyde de carbone}
%
%\begin{frame}
%	\begin{myact}{4 page 127}
	\begin{enumerate}
		\item Le gaz prélevé dans la seringue a été extrait d'eau pétillante par déplacement d'eau.\pause
		\item Au début de l'expérience, la solution d'eau de chaux est incolore et transparente.\pause
		\item Après y avoir fait barboter le gaz l'eau de chaux s'est troublée.\pause
		\item Un précipité blanc s'est formé lors de cette expérience, donc le gaz dissous dans l'eau pétillante est du dioxyde de carbone.
	\end{enumerate}
\end{myact}
%\end{frame}
%
%
%\begin{frame}
%	\begin{mybilan}
	\begin{itemize}
		\item La masse d'un corps est \kw{proportionnelle} à son volume; \pause
		\item Le coefficient de proportionnalité est la \kw{masse volumique} (notée $\rho$);\pause
		\item \kw{Un litre d'eau} a une masse de \kw{1 kilogramme};\pause
		\item Une substance est \kw{plus dense} qu'une autre si, pour un même volume, sa masse est supérieure.		
	\end{itemize}
\end{mybilan}
%\end{frame}

\end{document}