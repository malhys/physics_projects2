\section{Réaliser des soudures sur les circuits (3 points)}

En électronique, pour fixer un composant sur un circuit imprimé, on fait fondre un fil d'étain (métal dont la température de fusion est 232 °C) avec un fer à souder. La goutte d'étain déposée sur le circuit refroidit, fixant ainsi le composant sur le circuit.

\begin{questions}
	\question[1] Donner le nom du changement d'état subit par l'étain lorsqu'on le chauffe au fer à souder.
	
	\question[1] Nommer le changement d'état subit par la goutte d'étain sur le circuit en refroidissant.
	
	\question[1] Justifier l'utilisation de l'étain pour effectuer les soudures plutôt que le fer dont la température de fusion est de \num{1535} °C.  
\end{questions}