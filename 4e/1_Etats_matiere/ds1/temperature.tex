\section{Qu'est ce que la température (5 points) }\label{ex:temperture}

En physique, la température d'un corps représente l'agitation des molécules qui composent ce corps : plus les molécules bougent et plus le corps est chaud.

\begin{questions}
	\question[1] Au zéro absolu [\num{-273.15} \degree C], les molécules peuvent-elles bouger ?
	\begin{solution}
		Au zéro absolu, les molécules ne peuvent pas bouger.
	\end{solution}
	
	\question[2] Expliquer pourquoi, pour une matière donnée, l'état solide est plus froid que l'état liquide, lui même plus froids que l'état gazeux.
	\begin{solution}
		On sait qu'à l'état solide les molécules sont organisées et ne peuvent pas se déplacer les unes par rapport aux autres contrairement aux états liquide et gazeux. De plus à l'état gazeux les molécules se déplacent plus qu'à l'état liquide pour occuper tout l'espace disponible. C'est pourquoi l'état solide est plus froid que l'état liquide, lui même plus froids que l'état gazeux.
	\end{solution}
	
	\question[2] À l'état solide, les molécules sont très proches les unes des autres et fortement attachées les unes aux autres. C'est la raison pour laquelle les solides ont une forme et un volume propre : les molécules ne se déplacent presque pas les unes par rapport aux autres : elles se déplacent en blocs. \\ Expliquer comment l'augmentation de température permet de passer à l'état liquide.
	\begin{solution}
		La température est liée au mouvement des molécules les unes par rapport aux autres. Si l'on augmente la température d'un corps à l'état solide ses molécules vont se détacher et se mettre en mouvement, le corps va passer à l'état liquide.
	\end{solution}
	
\end{questions}