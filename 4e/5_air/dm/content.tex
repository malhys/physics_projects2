\section{Avec mon ballon solaire, je décolle}

	Le but de l'exercice est de comprendre comment un ballon jouet fourni dans un magazine peut s'envoler.
	Le \textbf{doc. 2} présente un ballon solaire avec son enveloppe fabriquée dans une matière plastique noire capable d'absorber les rayonnements de l'énergie solaire (\textbf{doc. 3}).
	D'après le \textbf{doc. 4} plus la température de l'air augmente, et plus sa masse volumique diminue.
	Le mode d'emploi du ballon présenté \textbf{doc. 1} indique qu'il faut tout d'abord le remplir d'air puis ensuite le laisser au soleil.
	L'énergie solaire captée par le ballon noir va alors réchauffer l'air enfermé à l'intérieur. 
	L'air à l'intérieur du ballon, plus chaud que l'air ambiant aura une masse volumique plus faible.
	Dans ce cas, le ballon sera plus léger que l'air ambiant et pourra s'envoler.
	
