\section{Avec mon ballon solaire, je décolle}

	Le but de l'exercice est de comprendre comment un ballon jouet fourni dans un magazine peut s'envoler.
	Le \textbf{doc. 2} présente un ballon solaire avec son enveloppe fabriquée dans une matière plastique noire capable d'absorber les rayonnements de l'énergie solaire (\textbf{doc. 3}).
	D'après le \textbf{doc. 4} plus la température de l'air augmente, et plus sa masse volumique diminue.
	Le mode d'emploi du ballon présenté \textbf{doc. 1} indique qu'il faut tout d'abord le remplir d'air puis ensuite le laisser au soleil.
	L'énergie solaire captée par le ballon noir va alors réchauffer l'air enfermé à l'intérieur. 
	L'air à l'intérieur du ballon, plus chaud que l'air ambiant aura une masse volumique plus faible.
	Dans ce cas, le ballon sera plus léger que l'air ambiant et pourra s'envoler.
	

\section{S'envole ou s'envole pas ?}

	Pour que le ballon s'envole il doit être plus léger que l'air dans lequel il se trouve. 
	D'après le \textbf{doc. 1} la masse volumique de l'air est de \num{1.2} $km/m^3$ alors que celle de l'hélium est \num{0.2} $km/m^3$. L'hélium est donc bien plus léger que l'air, c'est pour cela que, comme on le voit sur le \textbf{doc. 3} les ballons remplis d'hélium s'envolent. 
	
	Un ballon gonflé "au souffle" est rempli d'air expiré par la personne qui a gonflé. Air expiré qui, d'après le \textbf{doc. 3} contient moins de dioxygène et plus de dioxyde de carbone que l'air inspiré. Or comme l'indique le \textbf{doc. 1} le dioxyde de carbone est plus lourd que l'air, ainsi un ballon rempli d'air expiré sera forcément plus lourd que l'air et donc incapable de s'envoler. 