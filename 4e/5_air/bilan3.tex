\begin{mybilan}
	
	\begin{itemize}
		\item La \kw{masse volumique} d'un corps, notée \kw{$\rho$}, est le rapport entre la masse \kw{m} et son volume \kw{V}. Elle est spécifique à chaque matière pour une température et une pression donnée.\pause
		
		\begin{equation*}
		\mathbf{\rho} \; \mathit{{\small (kg/m^3)}} = \frac{\mathbf{m} \; \mathit{{\small (kg)}}}{\mathbf{V} \; \mathit{{\small (m^3)}}}\pause
		\end{equation*}
		
		\item Dans des conditions normales de température et de pression, la masse volumique de l'air est égale à :
		
			\begin{equation*}
				\rho = \num{1.2} \;g/L = \num{1.2} \;kg/m^3.
			\end{equation*} 
	\end{itemize}
	

	
\end{mybilan}