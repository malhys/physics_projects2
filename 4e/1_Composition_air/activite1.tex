\begin{myact}{1 page 14}
	\begin{enumerate}
		\item L'atmosphère est la couche d'air de faible épaisseur qui entoure la Terre. \pause
		\item L'atmosphère a une épaisseur moyenne de 600 km, soit environ $\frac{1}{10}$ du rayon de la Terre. Elle est formée de 5 couches.\pause
		\item La couche d'ozone nous protège des rayons UV, elle se situe dans la stratosphère.\pause
		\item Nous vivons dans la troposphère, elle contient l'air que l'on respire.\pause
		\item La troposphère mesure en moyenne 15 km d'épaisseur soit environ $\frac{1}{40}$ de l'atmosphère et $\frac{1}{400}$ du rayon de la Terre.\pause
		\item Les autres couches de l'atmosphère ne contiennent que très peu d'air, nous ne pourrions pas y vivre. 
	\end{enumerate}
\end{myact}