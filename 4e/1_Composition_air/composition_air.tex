\documentclass[12pt,a4paper]{article}

%\usepackage[left=1.5cm,right=1.5cm,top=1cm,bottom=2cm]{geometry}
\usepackage[in, plain]{fullpage}
\usepackage{array}
%\usepackage{../../pas-math}
\usepackage{../../moncours}



%-------------------------------------------------------------------------------
%          -Packages nécessaires pour écrire en Français et en UTF8-
%-------------------------------------------------------------------------------
\usepackage[utf8]{inputenc}
\usepackage[frenchb]{babel}
%\usepackage{numprint}
\usepackage[T1]{fontenc}
%\usepackage{lmodern}
\usepackage{textcomp}
\usepackage[french, boxed]{algorithm2e}
\usepackage{hyperref}


%-------------------------------------------------------------------------------

%-------------------------------------------------------------------------------
%                          -Outils de mise en forme-
%-------------------------------------------------------------------------------
\usepackage{hyperref}
\hypersetup{pdfstartview=XYZ}
%\usepackage{enumerate}
\usepackage{graphicx}
\usepackage{multicol}
\usepackage{tabularx}
\usepackage{multirow}
\usepackage{color}
\usepackage{eurosym}


\usepackage{anysize} %%pour pouvoir mettre les marges qu'on veut
%\marginsize{2.5cm}{2.5cm}{2.5cm}{2.5cm}

\usepackage{indentfirst} %%pour que les premier paragraphes soient aussi indentés
\usepackage{verbatim}
\usepackage{enumitem}
\usepackage{booktabs}
\usepackage[usenames,dvipsnames,svgnames,table]{xcolor}

\usepackage{variations}

%-------------------------------------------------------------------------------


%-------------------------------------------------------------------------------
%                  -Nécessaires pour écrire des mathématiques-
%-------------------------------------------------------------------------------
\usepackage{amsfonts}
\usepackage{amssymb}
\usepackage{amsmath}
\usepackage{amsthm}
\usepackage{tikz}
\usepackage{xlop}
\usepackage[output-decimal-marker={,}]{siunitx}
%-------------------------------------------------------------------------------

%-------------------------------------------------------------------------------
%                  -Nécessaires pour écrire des formules chimiquess-
%-------------------------------------------------------------------------------

\usepackage[version=4]{mhchem}

%-------------------------------------------------------------------------------
% Pour pouvoir exploiter les fichiers directement dans beamer
\newcommand{\pause}{\ }
%-------------------------------------------------------------------------------
%                    - Mise en forme avancée
%-------------------------------------------------------------------------------

\usepackage{ifthen}
\usepackage{ifmtarg}


\newcommand{\ifTrue}[2]{\ifthenelse{\equal{#1}{true}}{#2}{$\qquad \qquad$}}

%\newcommand{\kword}[1]{\textcolor{red}{\underline{#1}}}
%-------------------------------------------------------------------------------

%-------------------------------------------------------------------------------
%                     -Mise en forme d'exercices-
%-------------------------------------------------------------------------------
%\newtheoremstyle{exostyle}
%{\topsep}% espace avant
%{\topsep}% espace apres
%{}% Police utilisee par le style de thm
%{}% Indentation (vide = aucune, \parindent = indentation paragraphe)
%{\bfseries}% Police du titre de thm
%{.}% Signe de ponctuation apres le titre du thm
%{ }% Espace apres le titre du thm (\newline = linebreak)
%{\thmname{#1}\thmnumber{ #2}\thmnote{. \normalfont{\textit{#3}}}}% composants du titre du thm : \thmname = nom du thm, \thmnumber = numéro du thm, \thmnote = sous-titre du thm

%\theoremstyle{exostyle}
%\newtheorem{exercice}{Exercice}
%
%\newenvironment{questions}{
%\begin{enumerate}[\hspace{12pt}\bfseries\itshape a.]}{\end{enumerate}
%} %mettre un 1 à la place du a si on veut des numéros au lieu de lettres pour les questions 
%-------------------------------------------------------------------------------

%-------------------------------------------------------------------------------
%                    - Mise en forme de tableaux -
%-------------------------------------------------------------------------------

\renewcommand{\arraystretch}{1.7}

\setlength{\tabcolsep}{1.2cm}

%-------------------------------------------------------------------------------



%-------------------------------------------------------------------------------
%                    - Racourcis d'écriture -
%-------------------------------------------------------------------------------
%Droites
\newcommand{\dte}[1]{$(#1)$}
\newcommand{\fig}[1]{figure $#1$}
\newcommand{\sym}{symétrique}
\newcommand{\syms}{symétriques}
\newcommand{\asym}{axe de symétrie}
\newcommand{\asyms}{axes de symétrie}
\newcommand{\seg}[1]{$[#1]$}
\newcommand{\monAngle}[1]{$\widehat{#1}$}
\newcommand{\bissec}{bissectrice}
\newcommand{\mediat}{médiatrice}
\newcommand{\ddte}[1]{$[#1)$}


% Angles orientés (couples de vecteurs)
\newcommand{\aopp}[2]{(\vec{#1}, \vec{#2})} %Les deuc vecteurs sont positifs
\newcommand{\aopn}[2]{(\vec{#1}, -\vec{#2})} %Le second vecteur est négatif
\newcommand{\aonp}[2]{(-\vec{#1}, \vec{#2})} %Le premier vecteur est négatif
\newcommand{\aonn}[2]{(-\vec{#1}, -\vec{#2})} %Les deux vecteurs sont négatifs

%Ensembles mathématiques
\newcommand{\naturels}{\mathbb{N}} %Nombres naturels
\newcommand{\relatifs}{\mathbb{Z}} %Nombres relatifs
\newcommand{\rationnels}{\mathbb{Q}} %Nombres rationnels
\newcommand{\reels}{\mathbb{R}} %Nombres réels
\newcommand{\complexes}{\mathbb{C}} %Nombres complexes


%Intégration des parenthèses aux cosinus
\newcommand{\cosP}[1]{\cos\left(#1\right)}
\newcommand{\sinP}[1]{\sin\left(#1\right)}


%Probas stats
\newcommand{\stat}{statistique}
\newcommand{\stats}{statistiques}


\newcommand{\homo}{homothétie}
\newcommand{\homos}{homothéties}


\newcommand{\mycoord}[3]{(\textcolor{red}{\num{#1}} ; \textcolor{Green}{\num{#2}} ; \textcolor{blue}{\num{#3}})}
%-------------------------------------------------------------------------------

%-------------------------------------------------------------------------------
%                    - Mise en page -
%-------------------------------------------------------------------------------

\newcommand{\twoCol}[1]{\begin{multicols}{2}#1\end{multicols}}


\setenumerate[1]{font=\bfseries,label=\textit{\alph*})}
\setenumerate[2]{font=\bfseries,label=\arabic*)}


%-------------------------------------------------------------------------------
%                    - Elements cours -
%-------------------------------------------------------------------------------

%Correction d'exercice
\newcommand{\exoSec}[2]{\subsection*{Exercice #1 page #2}}
%-------------------------------------------------------------------------------
%                    - raccourcis d'écriture -
%-------------------------------------------------------------------------------

%Mise en évidence de termes clés
\newcommand{\mykw}[1]{\textcolor{red}{\underline{\textbf{#1}}}}

%Exercices
\newcommand{\exo}[2]{exercice #1 page #2}
\newcommand{\Exo}[2]{Exercice #1 page #2}

\renewcommand{\pause}{\ }


\begin{document}
	
	
\chap[num=14, color=blue]{\small Quelle est la composition de l'air ?}{Olivier FINOT, \today }	

\section{L'air dans l'atmosphère}

\begin{myact}{1 page 14}
	\begin{enumerate}
		\item L'atmosphère est la couche d'air de faible épaisseur qui entoure la Terre. \pause
		\item L'atmosphère a une épaisseur moyenne de 600 km, soit environ $\frac{1}{10}$ du rayon de la Terre. Elle est formée de 5 couches.\pause
		\item La couche d'ozone nous protège des rayons UV, elle se situe dans la stratosphère.\pause
		\item Nous vivons dans la troposphère, elle contient l'air que l'on respire.\pause
		\item La troposphère mesure en moyenne 15 km d'épaisseur soit environ $\frac{1}{40}$ de l'atmosphère et $\frac{1}{400}$ du rayon de la Terre.\pause
		\item Les autres couches de l'atmosphère ne contiennent que très peu d'air, nous ne pourrions pas y vivre. 
	\end{enumerate}
\end{myact}

\begin{mybilan}
	\begin{itemize}
		\item Dans un environnement sec, un courant électrique est dangereux à partir d'une tension de 50 V.\pause
		
		\item En France, une prise électrique fournit une tension de \kw{230 V}. Il ne faut pas toucher toucher les bornes d'une prise car cela pourrait provoquer une \kw{électrisation} voire une \kw{électrocution}.\pause
				 
	\end{itemize}

\end{mybilan}

\begin{mydefs}
	\begin{itemize}
		\item \kw{\'Electrisation} : passage du courant électrique à travers le corps humain. 
		
		\item \kw{\'Electrocution} : \'electrisation qui entraine la mort.
	\end{itemize}
\end{mydefs}

%\begin{myexos}
%	\twoCol{
%	\begin{itemize}
%		\item \exo{1}{20}
%	\end{itemize}}
%\end{myexos}

\section{Composition de l'air}

\begin{myact}{2 page 125}
	\begin{enumerate}
		\item Lorsque l'eau bout, il se forme de la vapeur dans l'erlenmeyer.\pause
		\item La vapeur d'eau emprisonnée dans l'erlenmeyer occupe tout l'espace disponible.\pause
		\item Quand les deux erlenmeyers sont en communication, on voit apparaître de la buée sur la paroi, car la vapeur est montée dans le deuxième erlenmeyer.\pause
		\item Après la mise en communication, la vapeur occupe l'espace des deux erlenmeyers.\pause
		\item Lorsque l'on appuie sur le piston, le volume d'air contenu dans la seringue fermée diminue.\pause
		\item Non, la vapeur d'eau n'a pas de forme propre.\pause
		\item La vapeur d'eau est expansible car lorsque l'on ajoute le second erlenmeyer, elle l'occupe en plus du premier.\pause
		\item Lorsque l'on appuie sur le piston de la seringue fermée, le volume d'air diminue, l'air est donc compressible.
	\end{enumerate}
\end{myact}

\begin{mybilan}
	\begin{itemize}
		\item L'énergie peut être \kw{transférée} d'un objet vers un autre objet.
		
		\item Une forme d'énergie peut être \kw{convertie} en une autre forme d'énergie.
		
		
		\begin{center}
			\includegraphics[scale=0.8]{conversion}
		\end{center}
		
		\item On représente un ensemble de transferts et conversions d'énergie par une \kw{chaine énergétique}.

		\begin{center}
			\includegraphics[scale=0.5]{chaine}
		\end{center}
	\end{itemize}
\end{mybilan}



\begin{myexos}
	\twoCol{
	\begin{itemize}
		\item \exo{6}{21}
		\item \exo{9}{21} (sans question d)
		\item \exo{13}{22}
		\item \exo{17}{23}
		\item \exo{19}{23}
	\end{itemize}}
\end{myexos}

\appendix

\newpage

\section*{Correction des exercices}

\subsection*{\exo{6}{21}}

\begin{enumerate}[label=\alph*)]
	\item $V = 3 \times \num{3.5} \times \num{2.6} = \num{27.3}$, soit \num{27.3} $m^3$ ou $ \num{27300}$ L.
	\item On considère $\frac{1}{5}$ de dioxygène pour $\frac{4}{5}$ de diazote dans l'air.
	On a donc  :\\ $V_{dioxygène}=\num{27.3} \times \num{0.2} = \num{5.46} m^3$, soit \num{5460} L.\\
	Et $V_{diazote}=\num{27.3} \times \num{0.8} = \num{21.84} m^3$, soit \num{21840} L.
\end{enumerate}

\subsection*{\exo{9}{21}}
\begin{enumerate}[label=\alph*)]
	\item Les producteurs de polluants atmosphériques visibles sur ce dessin sont : les centres urbains et industriels et les automobiles.
	\item Les oxydes d'azote et de souffre peuvent de transformer en acides.
	\item On retrouve ensuite ces acides dans les pluies puis dans les sols et dans les eaux.
	\item On rejette chaque année 100 T d'oxydes d'azote\footnote{http://www.destinationsante.com/Faut-il-bruler-les-incinerateurs.html} et \num{120 000} T d'oxydes de souffre. 
\end{enumerate}
\end{document}]