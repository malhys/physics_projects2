\section{Résolution d'un problème (4 points)}

Par mégarde, on a remplacé un fusible de 5 $A$ par un fusible de 10 $A$ dans le tableau électrique de la voiture. Qu'a-t-il pu arriver à la voiture Tesla pour qu'elle prenne feu ?


	\begin{doc}[h!]
		\caption{Voiture électrique Tesla S\textregistered}
		
		\begin{center}
			\includegraphics[scale=0.35]{img/tesla}
		\end{center}
	\end{doc}
	
	\begin{doc}[h!]
		\caption{D'après l'Est Républicain, 17 aout 2016 }
		
		<< Une voiture du fabricant américain de véhicules électriques Tesla Motors (modèle S90 D) a pris feu spontanément ce lundi à Bayonne à l'occasion de journées promotionnelles.>>
	\end{doc}




\begin{doc}[h!]
	\caption{Schéma de différents fusibles à lames}
	
	\begin{center}
		\includegraphics[scale=0.3]{img/fusibles}
	\end{center}
	
	Pour protéger les circuits électriques des véhicules, on utilise des fusibles. Ceux-cis fondent en cas d'intensité trop forte : le circuit est alors ouvert.
\end{doc}

\begin{solution}
	La voiture a pris feu à cause d'une surintensité, une partie de son circuit électrique a été traversée par un courant d'une intensité supérieure à celle prévue à cause d'un mauvais fusible.
	Si le courant a une intensité supérieure à 5 $A$ le fusible est censé fondre et ouvrir le circuit pour empêcher les problèmes. Mais ce fusible a été remplacé par un de 10 $A$, il n'a donc pas fondu.
\end{solution}