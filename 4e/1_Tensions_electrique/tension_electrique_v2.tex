\documentclass[12pt,a4paper]{article}

%\usepackage[left=1.5cm,right=1.5cm,top=1cm,bottom=2cm]{geometry}
\usepackage[in, plain]{fullpage}
\usepackage{array}
%\usepackage{../../pas-math}
\usepackage{../../moncours}



%-------------------------------------------------------------------------------
%          -Packages nécessaires pour écrire en Français et en UTF8-
%-------------------------------------------------------------------------------
\usepackage[utf8]{inputenc}
\usepackage[frenchb]{babel}
%\usepackage{numprint}
\usepackage[T1]{fontenc}
%\usepackage{lmodern}
\usepackage{textcomp}
\usepackage[french, boxed]{algorithm2e}
\usepackage{hyperref}


%-------------------------------------------------------------------------------

%-------------------------------------------------------------------------------
%                          -Outils de mise en forme-
%-------------------------------------------------------------------------------
\usepackage{hyperref}
\hypersetup{pdfstartview=XYZ}
%\usepackage{enumerate}
\usepackage{graphicx}
\usepackage{multicol}
\usepackage{tabularx}
\usepackage{multirow}
\usepackage{color}
\usepackage{eurosym}


\usepackage{anysize} %%pour pouvoir mettre les marges qu'on veut
%\marginsize{2.5cm}{2.5cm}{2.5cm}{2.5cm}

\usepackage{indentfirst} %%pour que les premier paragraphes soient aussi indentés
\usepackage{verbatim}
\usepackage{enumitem}
\usepackage{booktabs}
\usepackage[usenames,dvipsnames,svgnames,table]{xcolor}

\usepackage{variations}

%-------------------------------------------------------------------------------


%-------------------------------------------------------------------------------
%                  -Nécessaires pour écrire des mathématiques-
%-------------------------------------------------------------------------------
\usepackage{amsfonts}
\usepackage{amssymb}
\usepackage{amsmath}
\usepackage{amsthm}
\usepackage{tikz}
\usepackage{xlop}
\usepackage[output-decimal-marker={,}]{siunitx}
%-------------------------------------------------------------------------------

%-------------------------------------------------------------------------------
%                  -Nécessaires pour écrire des formules chimiquess-
%-------------------------------------------------------------------------------

\usepackage[version=4]{mhchem}

%-------------------------------------------------------------------------------
% Pour pouvoir exploiter les fichiers directement dans beamer
\newcommand{\pause}{\ }
%-------------------------------------------------------------------------------
%                    - Mise en forme avancée
%-------------------------------------------------------------------------------

\usepackage{ifthen}
\usepackage{ifmtarg}


\newcommand{\ifTrue}[2]{\ifthenelse{\equal{#1}{true}}{#2}{$\qquad \qquad$}}

%\newcommand{\kword}[1]{\textcolor{red}{\underline{#1}}}
%-------------------------------------------------------------------------------

%-------------------------------------------------------------------------------
%                     -Mise en forme d'exercices-
%-------------------------------------------------------------------------------
%\newtheoremstyle{exostyle}
%{\topsep}% espace avant
%{\topsep}% espace apres
%{}% Police utilisee par le style de thm
%{}% Indentation (vide = aucune, \parindent = indentation paragraphe)
%{\bfseries}% Police du titre de thm
%{.}% Signe de ponctuation apres le titre du thm
%{ }% Espace apres le titre du thm (\newline = linebreak)
%{\thmname{#1}\thmnumber{ #2}\thmnote{. \normalfont{\textit{#3}}}}% composants du titre du thm : \thmname = nom du thm, \thmnumber = numéro du thm, \thmnote = sous-titre du thm

%\theoremstyle{exostyle}
%\newtheorem{exercice}{Exercice}
%
%\newenvironment{questions}{
%\begin{enumerate}[\hspace{12pt}\bfseries\itshape a.]}{\end{enumerate}
%} %mettre un 1 à la place du a si on veut des numéros au lieu de lettres pour les questions 
%-------------------------------------------------------------------------------

%-------------------------------------------------------------------------------
%                    - Mise en forme de tableaux -
%-------------------------------------------------------------------------------

\renewcommand{\arraystretch}{1.7}

\setlength{\tabcolsep}{1.2cm}

%-------------------------------------------------------------------------------



%-------------------------------------------------------------------------------
%                    - Racourcis d'écriture -
%-------------------------------------------------------------------------------
%Droites
\newcommand{\dte}[1]{$(#1)$}
\newcommand{\fig}[1]{figure $#1$}
\newcommand{\sym}{symétrique}
\newcommand{\syms}{symétriques}
\newcommand{\asym}{axe de symétrie}
\newcommand{\asyms}{axes de symétrie}
\newcommand{\seg}[1]{$[#1]$}
\newcommand{\monAngle}[1]{$\widehat{#1}$}
\newcommand{\bissec}{bissectrice}
\newcommand{\mediat}{médiatrice}
\newcommand{\ddte}[1]{$[#1)$}


% Angles orientés (couples de vecteurs)
\newcommand{\aopp}[2]{(\vec{#1}, \vec{#2})} %Les deuc vecteurs sont positifs
\newcommand{\aopn}[2]{(\vec{#1}, -\vec{#2})} %Le second vecteur est négatif
\newcommand{\aonp}[2]{(-\vec{#1}, \vec{#2})} %Le premier vecteur est négatif
\newcommand{\aonn}[2]{(-\vec{#1}, -\vec{#2})} %Les deux vecteurs sont négatifs

%Ensembles mathématiques
\newcommand{\naturels}{\mathbb{N}} %Nombres naturels
\newcommand{\relatifs}{\mathbb{Z}} %Nombres relatifs
\newcommand{\rationnels}{\mathbb{Q}} %Nombres rationnels
\newcommand{\reels}{\mathbb{R}} %Nombres réels
\newcommand{\complexes}{\mathbb{C}} %Nombres complexes


%Intégration des parenthèses aux cosinus
\newcommand{\cosP}[1]{\cos\left(#1\right)}
\newcommand{\sinP}[1]{\sin\left(#1\right)}


%Probas stats
\newcommand{\stat}{statistique}
\newcommand{\stats}{statistiques}


\newcommand{\homo}{homothétie}
\newcommand{\homos}{homothéties}


\newcommand{\mycoord}[3]{(\textcolor{red}{\num{#1}} ; \textcolor{Green}{\num{#2}} ; \textcolor{blue}{\num{#3}})}
%-------------------------------------------------------------------------------

%-------------------------------------------------------------------------------
%                    - Mise en page -
%-------------------------------------------------------------------------------

\newcommand{\twoCol}[1]{\begin{multicols}{2}#1\end{multicols}}


\setenumerate[1]{font=\bfseries,label=\textit{\alph*})}
\setenumerate[2]{font=\bfseries,label=\arabic*)}


%-------------------------------------------------------------------------------
%                    - Elements cours -
%-------------------------------------------------------------------------------

%Correction d'exercice
\newcommand{\exoSec}[2]{\subsection*{Exercice #1 page #2}}
%-------------------------------------------------------------------------------
%                    - raccourcis d'écriture -
%-------------------------------------------------------------------------------

%Mise en évidence de termes clés
\newcommand{\mykw}[1]{\textcolor{red}{\underline{\textbf{#1}}}}

%Exercices
\newcommand{\exo}[2]{exercice #1 page #2}
\newcommand{\Exo}[2]{Exercice #1 page #2}

\renewcommand{\pause}{\ }

%Intervalles
\newcommand{\interOO}[2]{$]$#1 , #2$[$}
\newcommand{\interOF}[2]{$]$#1 , #2$]$}
\newcommand{\interFO}[2]{$[$#1 , #2$[$}
\newcommand{\interFF}[2]{$[$#1 , #2$]$}



\begin{document}
	
	
\chap[num=14, color=blue]{Tension électrique}{Olivier FINOT, \today }	

\section{Mesure d'une tension électrique}

\begin{myrap}
	Un \kw{dipôle} est un composant électrique qui possède deux bornes (piles, lampes, interrupteurs, etc .).
\end{myrap}


\begin{mybilan}
	L'unité de mesure de la tension électrique est le \kw{volt} (symbole $V$). Elle se mesure avec un \kw{voltmètre}.
	Pour mesurer la tension électrique d'une pile, on relie la \kw{borne rouge} du voltmètre à la \kw{borne $[+]$} de la pile et l'autre borne du voltmètre à la borne $[-]$.
	
	Pour mesurer la tension aux bornes d'un dipôle placé dans un circuit, on branche un \kw{voltmètre en dérivation} entre les bornes de ce dipôle. Il existe une \kw{tension électrique} entre les bornes d'un \kw{interrupteur ouvert} placé dans un circuit.
	La tension électrique entre les bornes d'un fil de connexion est nulle.
\end{mybilan}

\begin{myexos}
	\twoCol{
	\begin{itemize}
		\item \exo{5}{97}
		\item \exo{6}{97}
		\item \exo{7}{97}
		\item \exo{8}{97}
		\item \exo{15}{98}
		\item \exo{17}{98}		
	\end{itemize}}
\end{myexos}


\section{Adaptation d'un dipôle}

\begin{mybilan}
	La \kw{tension nominale} notée sur un appareil électrique est une indication de fonctionnement correct.
	Si la tension électrique aux bornes d'un dipôle est \kw{proche de sa tension nominale}, alors il y a \kw{adaptation} du dipôle au générateur. 
	Il y a \kw{sous-tension} si la tension est inférieure à la tension nominale. Il y a \kw{surtension} si elle est supérieure, le dipôle risque d'être détérioré. 
\end{mybilan}

\begin{myex}
	\twoCol{
		\begin{itemize}
			\item \exo{14}{98}
			\item \exo{18}{99}
			\item \exo{20}{99}
		\end{itemize}
	}
\end{myex}

\section{Dipôles en série et en dérivation}


\begin{mybilan}
	\begin{itemize}
		\item Dans un \kw{circuit série} :
		\begin{itemize}
			\item la valeur de la tension entre les bornes d'un dipôle ne \kw{dépend pas de sa position} dans le circuit.
			\item la valeur de la \kw{tension $U$} aux bornes du générateur est égale à la \kw{somme des valeurs des tension $U_1$ et $U_2$} entre les bornes des dipôles : c'est la \kw{loi d'additivité des tensions}.
			
			\vspace*{-1cm}
			
			\kw{\begin{center}
				\begin{equation*}
					U=U_1+U_2
				\end{equation*}
			\end{center}}
		\end{itemize}
	
		\item Dans un circuit comportant des \kw{dérivations}, la valeur de la \kw{tension est la même} entre les bornes des dipôles branchés en \kw{dérivation}.
		
		\vspace*{-1cm}
		
		\kw{\begin{center}
				\begin{equation*}
					U=U_1=U_2
				\end{equation*}
		\end{center}}
	\end{itemize}
\end{mybilan}


\begin{myex}
	\twoCol{
		\begin{itemize}
			\item \exo{8}{110}
			\item \exo{10}{110}
			\item \exo{11}{110}
			\item \exo{12}{111}
			\item \exo{17}{111}
			\item \exo{18}{111}
			\item \exo{19}{111}
			\item \exo{20}{111}
		\end{itemize}
	}
\end{myex}

\appendix



\newpage

\section*{Correction des exercices}

\subsection*{\exo{5}{97}}

\twoCol{
\begin{itemize}
	\item 1 $mV$ $=$ \textbf{\num{0.001}} $V$
	\item 1 $V$ $=$ \textbf{\num{1000}} $mV$
	\item 1 $kV$ $=$ \textbf{\num{1000}} $V$
	\item 1 $V$ $=$ \textbf{\num{0.001}} $kV$
	\item 1 $kV$ $=$ \textbf{\num{1000000}} $mV$
\end{itemize}}

\subsection*{\exo{6}{97}}


	\begin{enumerate}
		\item Incorrect : la borne COM du voltmètre doit être reliée à la borne négative $(-)$ du générateur.
		\item Incorrect : la borne rouge du voltmètre doit être reliée à la borne positive $(+)$ du générateur.
		\item Le montage est correct.
	\end{enumerate}


\subsection*{\exo{7}{97}}
\begin{multicols}{2}
	\begin{enumerate}
		\item 
	\end{enumerate}
\end{multicols}

\subsection*{\exo{8}{97}}


	\begin{enumerate}[label=\alph*)]
		\item 
	\end{enumerate}

\subsection*{\exo{14}{98}}

\subsection*{\exo{15}{98}}

\subsection*{\exo{17}{98}}

\subsection*{\exo{18}{99}}

\subsection*{\exo{20}{99}}
\end{document}]